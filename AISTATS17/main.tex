\documentclass[twoside]{article} \usepackage{aistats2017}

% If your paper is accepted, change the options for the package
% aistats2017 as follows:
%
%\usepackage[accepted]{aistats2017}
%
% This option will print headings for the title of your paper and
% headings for the authors names, plus a copyright note at the end of
% the first column of the first page.


\usepackage{algorithm}
\usepackage{algorithmic}
\usepackage{multirow}
\usepackage{booktabs}
\usepackage{pifont} % \checkmark
\usepackage[utf8]{inputenc} % allow utf-8 input
\usepackage[T1]{fontenc}    % use 8-bit T1 fonts
\usepackage{hyperref}       % hyperlinks
\usepackage{url}            % simple URL typesetting
\usepackage{booktabs}       % professional-quality tables
\usepackage{amsfonts}       % blackboard math symbols
\usepackage{nicefrac}       % compact symbols for 1/2, etc.
\usepackage{microtype}      % microtypography
%\usepackage[english]{babel}
%\usepackage{cite}
%\usepackage{verbatim}
\usepackage{amsmath,amsthm}
\usepackage{amssymb,amsbsy,epsfig,float}
\usepackage{graphicx,wrapfig,lipsum}
%\usepackage{graphicx}
%\usepackage{multirow}
%\usepackage{algorithmicx}
%\usepackage[ruled]{algorithm}
%\usepackage{algpseudocode}
\usepackage{subfigure} 
\usepackage[makeroom]{cancel}
\usepackage{xspace}
\usepackage{mathtools}

\usepackage{color}
\usepackage[usenames,dvipsnames]{xcolor}

\usepackage[backgroundcolor = White,textwidth=\marginparwidth]{todonotes}
% To remove todo notes, simply uncomment the following line and comment out the previous one
% \usepackage[disable,backgroundcolor = White,textwidth=\marginparwidth]{todonotes}

% Comments by Csaba:
\newcommand{\todoc}[2][]{\todo[color=Apricot!20,size=\tiny,#1]{Cs: #2}}
% Comments by Manjesh:
\newcommand{\todom}[2][]{\todo[color=Cerulean!20,size=\tiny,#1]{M: #2}}
% Comments by Venkatesh:
\newcommand{\todov}[2][]{\todo[color=Purple!20,size=\tiny,#1]{V: #2}}

\newcommand{\hY}{\hat{Y}}
\newcommand{\N}{\mathbb{N}}
\newcommand{\htheta}{\hat{\theta}}
\newcommand{\iset}[1]{[#1]}
\DeclareMathOperator{\argmin}{argmin}
\newcommand{\ip}[1]{\langle #1 \rangle} % inner product
\newcommand{\SA}{\mathrm{SA}}
\newcommand{\SD}{\mathrm{SD}}
\newcommand{\WD}{\mathrm{WD}}
\newcommand{\TSA}{\Theta_{\SA}}
\newcommand{\Alg}{\mathfrak{A}}
\newcommand{\TSD}{\Theta_{\SD}}
\newcommand{\TWD}{\Theta_{\WD}}
\newcommand{\awd}{a_{\mathrm{wd}}}
\def\ddefloop#1{\ifx\ddefloop#1\else\ddef{#1}\expandafter\ddefloop\fi}

% \bA, \bB, ...
\def\ddef#1{\expandafter\def\csname b#1\endcsname{\ensuremath{\mathbf{#1}}}}
\ddefloop ABCDEFGHIJKLMNOPQRSTUVWXYZ\ddefloop

% \bbA, \bbB, ...
\def\ddef#1{\expandafter\def\csname bb#1\endcsname{\ensuremath{\mathbb{#1}}}}
\ddefloop ABCDEFGHIJKLMNOPQRSTUVWXYZ\ddefloop

% \cA, \cB, ...
\def\ddef#1{\expandafter\def\csname c#1\endcsname{\ensuremath{\mathcal{#1}}}}
\ddefloop ABCDEFGHIJKLMNOPQRSTUVWXYZ\ddefloop

% \vA, \vB, ..., \va, \vb, ...
\def\ddef#1{\expandafter\def\csname v#1\endcsname{\ensuremath{\boldsymbol{#1}}}}
\ddefloop ABCDEFGHIJKLMNOPQRSTUVWXYZabcdefghijklmnopqrstuvwxyz\ddefloop

% \valpha, \vbeta, ...,  \vGamma, \vDelta, ...,
\def\ddef#1{\expandafter\def\csname v#1\endcsname{\ensuremath{\boldsymbol{\csname #1\endcsname}}}}
\ddefloop {alpha}{beta}{gamma}{delta}{epsilon}{varepsilon}{zeta}{eta}{theta}{vartheta}{iota}{kappa}{lambda}{mu}{nu}{xi}{pi}{varpi}{rho}{varrho}{sigma}{varsigma}{tau}{upsilon}{phi}{varphi}{chi}{psi}{omega}{Gamma}{Delta}{Theta}{Lambda}{Xi}{Pi}{Sigma}{varSigma}{Upsilon}{Phi}{Psi}{Omega}\ddefloop

\newcommand{\Y}{\mathcal{Y}}
\newcommand{\A}{\mathcal{A}}
\newcommand{\EE}[1]{\mathbb{E}\left[#1\right]}
\newcommand{\EEi}[2]{\mathbb{E}_{#1}\left[#2\right]}
\newcommand{\Prob}[1]{\mathbb{P}\left(#1\right)}
\newcommand{\Regret}{\mathfrak{R}}
\newcommand{\R}{\mathbb{R}} % reals
\newcommand{\Yti}{Y_t^i}
\newcommand{\Yt}{Y_t}
\newcommand{\X}{\mathcal{X}}


\usepackage[capitalize]{cleveref}

\newcommand\numberthis{\addtocounter{equation}{1}\tag{\theequation}}

\newtheorem{thm}{Theorem}
\newtheorem{lem}{Lemma}
\newtheorem{prop}{Proposition}
\newtheorem{cor}{Corollary}
\newtheorem{ex}{Example}
\newtheorem{cond}{Condition}
\newtheorem{rem}{Remark}
\newtheorem{defi}{Definition}
\newtheorem{ass}{Assumption}



\begin{document}

% If your paper is accepted and the title of your paper is very long,
% the style will print as headings an error message. Use the following
% command to supply a shorter title of your paper so that it can be
% used as headings.
%
%\runningtitle{I use this title instead because the last one was very long}

% If your paper is accepted and the number of authors is large, the
% style will print as headings an error message. Use the following
% command to supply a shorter version of the authors names so that
% they can be used as headings (for example, use only the surnames)
%
%\runningauthor{Surname 1, Surname 2, Surname 3, ...., Surname n}

\twocolumn[

\aistatstitle{Unsupervised Sequential Sensor Acquisition}

\aistatsauthor{ Anonymous Author 1 \And Anonymous Author 2 \And Anonymous Author 3 }

\aistatsaddress{ Unknown Institution 1 \And Unknown Institution 2 \And Unknown Institution 3 } ]

\begin{abstract}
	Sequential sensor acquisition problems (SAP) arise \todom{SAP is not a standard setup, we need to introduce it first .} in many application domains including medical-diagnostics, security and surveillance. SAP architecture is organized as a cascaded network of ``intelligent'' sensors that produce decisions upon acquisition. Sensors must be acquired sequentially and comply with the architecture. Our task is to identify the sensor with optimal accuracy-cost tradeoff. We formulate SAP as a version of the stochastic partial monitoring problem with side information and {\it unusual} reward structure.  Actions correspond to choice of sensor and the chosen sensor's parents decisions are available as side information. Nevertheless, what is atypical, is that we do not observe the reward/feedback, which a learner often uses to reject suboptimal actions. Unsurprisingly, with no further assumptions, we show that no learner can achieve sublinear regret. This negative result leads us to introduce the notion of weak dominance on cascade structures. Weak dominance supposes that a child node in the cascade has higher accuracy whenever its parent's predictions are correct. \todoc{The story is a bit more complicated. The abstract will need a rewrite once we settle on the results.}
	We then empirically verify this assumption on real datasets. We show that weak dominance is a maximal learnable set in the sense that we must suffer linear regret for any non-trivial expansion of this set. Furthermore, by reducing SAP to a special case of multi-armed bandit problem with side information we show that for any instance in the weakly dominant we only suffer a sublinear regret.
	%		We propose a sensor acquisition problem (SAP) wherein sensors (and sensing tests) are organized into a cascaded architecture and the goal is to choose a test with the optimal cost-accuracy tradeoff for a given instance. We consider the case where we obtain no feedback in terms of rewards for our chosen actions apart from test observations. Absence of feedback raises fundamentally new challenges since one cannot infer potentially optimal tests. We pose the problem in terms of competitive optimality with the goal of minimizing cumulative regret against optimally chosen actions in hindsight. In this context we introduce the notion of weak dominance and show that it is necessary and sufficient for realizing sub-linear regret. Weak dominance on a cascade supposes that a child node in the cascade has higher accuracy when its parent node makes correct predictions. When weak dominance holds we show that we can reduce SAP to a corresponding multi-armed bandit problem with side observations. Empirically we verify that weak dominance holds for many datasets.
\end{abstract}


\section{Introduction}
%!TEX root =  main.tex
Sequential sensor acquisition arises in many scenarios where we have a diverse collection of sensors with differing costs and accuracy. In these applications, to minimize costs, one often chooses inexpensive sensors first; and based on their outcomes, one sequentially decides whether or not to acquire more expensive sensors. For instance, in security systems%\footnote{A similar situation arises in clinical diagnosis, where doctors order tests such as genetic markers, imaging (CT, ultrasound, elastography) and biopsy sequentially for cancer diagnosis.} 
 (see \cite{ML13_MultistageClassifier_TrapezSaligramaCastanon} and other medically oriented examples), costs can arise due to sensor availability and delay. A suite of sensors/tests including inexpensive ones such as magnetometers, video feeds, to more expensive ones such as millimeter wave imagers are employed. These sensors are typically organized in a hierarchical architecture with low-cost sensors at the top of the hierarchy. The task is to determine which sensor acquisitions lead to maximizing accuracy for the available cost-budget. 

These scenarios motivate us to propose the unsupervised sequential sensor acquisition problem (SAP). Our SAP architecture is organized as a cascaded network of intelligent sensors. The sensors when utilized to probe an instance, outputs a prediction of the underlying state of the instance (anomaly or normal, threat or no-threat etc.). Sensors are ordered with respect to increasing cost and accuracy. While the costs are assumed to be known a priori, the exact misclassification rate of a sensor is unknown. This setup is realistic in security and surveillance scenarios because sensors are often required to be deployed in new domains/environments with little or no opportunity for re-calibration. 

We assume that the scenario is played over multiple rounds with an instance associated with each round. Sensors must be acquired sequentially and comply with the cascade architecture in each round. The learner's goal is to figure out the hidden, stochastic state of the instance based on the sensor outputs. Since the learner knows that the sensors are ordered from least to most accurate he/she can use the most accurate sensor among his/her acquired sensors for prediction. Nevertheless, since the learner does not know the sensor accuracy he/she faces the dilemma of as to which sensor to use for predicting this state.

%As in problems that deal with sequential learning under uncertainty, we pose the problem as a stochastic partial monitoring problem. In particular we consider a competitor who has the benefit of hindsight and can choose an optimal collection of tests for all the examples. Our goal is to learn the action, while our learning algorithm's performance is evaluated using their cumulative regret with respect to the competitor.


%In a sequential sensor acquisition problem (SAP) \todoc{This comes a little late.}
 %the goal is to acquire the tests/sensors that achieves the optimal cost-accuracy tradeoff.
 %We assume that the sensors/tests are organized into a diagnostic cascade architecture, where the ordering is based on costs/informativity of tests. Each stage in the cascade outputs a prediction of the underlying state of the instance (disease or disease-free, threat or no-threat etc.). We suppose that the classifiers (or predictors) corresponding to each node are part of the system and produce labeled outputs. This is often the case in diagnostic systems where a test ordering is a priori known and a report is produced by a human being or an automated mechanism corresponding to different sensor measurements. Thus our task in this paper is primarily to learn a decision rule to identify the collection of tests required for an instance. 

%Our task is to identify the sensor with optimal accuracy-cost tradeoff. We assume that while the costs for each sensor is known their corresponding accuracy is unknown.  
%The learner's goal in any round is to figure out the hidden, stochastic state of the environment based on the sensor outputs $(\Yt^{k})_{k\in [K]}$.
%The dilemma of the learner is which sensors to use for predicting this state.
%The learner knows that the sensors are ordered from least to most accurate:


%that require a careful tradeoff 
%In many applications 

%medical diagnosis  and  homeland  security,  sequential decisions  are  often  warranted.   For each  instance,  an initial diagnostic test  is conducted and based on its result further tests maybe conducted in a fixed ordering, where ordering of the tests is often based on their cost.
%Given the outcomes of the test results at some stage,
%a classifier, which is part of the diagnostic architecture,
%produces a prediction of the unknown label of the instance to be classified.
%Apart from the above-mentioned natural scenarios, 
%the problem also arises in human engineered systems, such as in the context of wireless communication systems,
%where a cascade of error-correcting decoders of increasing block lengths are designed to overcome channel noise, 
%but using a larger block lengths incurs extra communication cost.

%In all these examples, tests have varying  costs for  acquisition, accounting for delay,  throughput  or  monetary  value.%
%\footnote{As described in \cite{ML13_MultistageClassifier_TrapezSaligramaCastanon} security systems utilize a suite of sensors/tests such as X-rays, millimeter wave imagers (expensive \& low-throughput), magnetometers, video, IR imagers human search.  Security systems  must  maintain  a  throughput  constraint  in  order to  keep  pace  with  arriving  traffic.   In  clinical  diagnosis, doctors  in the context of breast cancer diagnosis utilize tests such as genetic markers, imaging (CT, ultrasound, elastography) and biopsy. Sensors providing imagery are scored by humans. The different sensing  modalities  have  diverse  costs,  in  terms  of  health risks (radiation exposure) and monetary expense.}

%We model these situations

%As in sequential learning under uncertainty, we pose the problem in terms of competitive optimality. In particular we consider a competitor who has the benefit of hindsight and can choose an optimal collection of tests for all the examples. Our goal is to learn the action, while our learning algorithm's performance is evaluated using their  cumulative regret with respect to the competitor.

We frame our problem as a version of stochastic partial monitoring problem \citep{BaFoPaRaSze14} with \emph{atypical} reward structure. As is common, we pose the problem in terms of competitive optimality. We consider a competitor who can choose an optimal action with the benefit of hindsight. Our goal is to minimize cummulative regret based on learning the optimal action based on observations that are observed during multiple rounds of play.


Stochastic partial monitoring problem is itself a generalization of multi-armed bandit problems, the latter going back to \citet{Tho33}. In our context, we view sensors choices as actions. The availability of predictions of parent sensors of a chosen sensor is viewed as side observation.  Recall that in a stochastic partial monitoring problem a decision maker needs to choose the action with the lowest expected cost by repeatedly trying the actions and observing some feedback.
The decision maker lacks the knowledge of some key information, such as in our case, the misclassification
error rates of the classifiers, but had this information been available, the decision maker could calculate the
expected costs of all the actions (sensor acquisitions) and could choose the best action (sensor). The feedback received by the decision maker in a given round depends stochastically on the unknown information and the action chosen.
Bandit problems are a special case of partial monitoring, where the key missing information is the expected
cost for each action (or arm), and the feedback is simply the noisy version of the expected cost of the action chosen.
In the \emph{unsupervised} version considered here
and which we call the unsupervised \emph{sequential sensor acquisition problem} (SAP),
the learner only observes the outputs of the classifiers, but not the label to be predicted over multiple rounds
in a stochastic, stationary environment. 


%To cast our problem as a partial monitoring problem, \todoc{Do we need this? Or leave this to the reader, just saying that casting our problem as a partial monitoring problem is trivial?} 
%the key unknown information can be the misclassification error rates of the classifiers, an action is identified with 
%the subset of sensors selected, the cost of an action is the sum of the misclassification cost of the classifiers
%that uses the selected sensor subset outputs and the cost of acquiring these sensor outputs,
%while the observed feedback is the vector of predicted labels by each of the classifiers that use 
%the first, the first and second, etc., up to all sensor outputs from the sensors that were selected.
%Note that unlike in a conventional bandit problem, we do not get \emph{direct} 
%feedback of how well our action performed (either noisy or noiseless)\footnote{This problem naturally arises in the surveillance and medical domains. We can perform a battery of tests on an individual in an airport but can never be sure whether or not he/she poses a threat.}.

%Were the probability of error known for each classifier that uses an initial segment of the tests, 
%a decision maker could optimally balance the cost of erroneous decisions and that of the sensor acquisitions'.
%\todoc{This assumes a cost associated with each error; should this be noted?}
%In the learning version of the problem, the misclassification probabilities are \emph{a priori} unknown and a learner must learn the optimal balance based on some feedback available to him. 
%In the \emph{unsupervised} version considered here
%and which we call the unsupervised \emph{sequential sensor acquisition problem} (SAP),
%the learner only observes the outputs of the classifiers, but not the label to be predicted over multiple rounds
%in a stochastic, stationary environment. 

This leads us to the following question: Can a learner still achieve the optimal balance in this case?  We first show that, unsurprisingly, with no further assumptions, no learner can achieve sublinear regret. \todoc{Has regret been defined?}
This negative result leads us to introduce the notion of weak dominance on tests. It is best described as a relaxed notion of strong dominance. Strong dominance states that a sensor's predictions are almost surely correct whenever the parent nodes in the cascade are correct. \todoc{Weak dominance has not been introduced yet.} We empirically demonstrate that weak dominance appears to hold by evaluating it on several real datasets. We also show that in a sense weak dominance is fundamental, namely, without this condition there exist problem instances that result in linear regret. On the other hand whenever this condition is satisfied there exist polynomial time algorithms that lead to sublinear ($O(\sqrt{T})$) cummulative regret. 

%In particular, we reduce the SAP problem to a stochastic multi-armed bandit with side observations, 
%a problem introduced by \citet{MaSh11}.
Our proof of sublinear regret is based on reducing SAP to a version of multi-armed bandit problem (MAB) with side-observation. The latter problem has already been shown to have sub-linear regret in the literature. In our reduction, we identify sensor nodes in the cascade as the bandit arms. % are identified by the nodes of the cascade. \todoc{Should we introduce cascade formally then above?}
The payoff of an arm is given by loss from the corresponding stage, and the side observation structure is defined by the feedback graph induced by the cascade. We then formally show that there is a one-to-one mapping between algorithms for SAP and algorithms for MAB with side-observation. In particular, under weak dominance, the regret bounds for MAB with side-observation then imply corresponding regret bounds for SAP. 

%The weak dominance condition occasionally can be shown to hold by design. For example, we do show that,
%in fact, a stronger dominance condition holds in the context of the communication systems 
%and error-correcting code cascades, implying the weak dominance condition there.
%Empirically, with the help of labelled data, 
%we verify that weak dominance condition naturally holds for several real-world problems,
%including diagnosing breast-cancer and diabetes.

%We show that weak dominance is fundamental, i.e., regardless of the algorithm, if this condition is not satisfied, we are left with a linear regret. On the other hand, we develop UCB style algorithms that show that we can realize optimal regret (sub-linear regret) guarantees when the condition is satisfied. 
%Thus, we identify weak dominance as the sharp necessary and sufficient condition for the learnability of
%our sensor acquisition problem.


%Unsurprisingly, it turns out that, without any further assumptions we show that no learner can achieve sublinear regret. This negative result leads us to introduce the notion of weak dominance on cascade structures. As we explain later weak dominance supposes that a child node in the cascade has higher accuracy whenever its parent's predictions are correct.


%\todoc{Too provocative? Give a hint that we will add some conditions?}

\if0
%In a sequential sensor acquisition problem (SAP) \todoc{This comes a little late.}
% the goal is to acquire the tests/sensors that achieves the optimal cost-accuracy tradeoff.
% We assume that the sensors/tests are organized into a diagnostic cascade architecture, where the ordering is based on costs/informativity of tests. Each stage in the cascade outputs a prediction of the underlying state of the instance (disease or disease-free, threat or no-threat etc.). We suppose that the classifiers (or predictors) corresponding to each node are part of the system and produce labeled outputs. This is often the case in diagnostic systems where a test ordering is a priori known and a report is produced by a human being or an automated mechanism corresponding to different sensor measurements. Thus our task in this paper is primarily to learn a decision rule to identify the collection of tests required for an instance. 
\fi

%Our problem can be framed as a stochastic partial monitoring problem \citep{BaFoPaRaSze14},
%which itself is a generalization of multi-armed bandit problems, going back to \citet{Tho33}. 
%Recall that in a stochastic partial monitoring problem a decision maker needs to choose the action with the lowest 
%expected cost by repeatedly trying the actions and observing some feedback.
%The decision maker lacks the knowledge of some key information, such as in our case, the misclassification
%error rates of the classifiers, but had this information been available, the decision maker could calculate the
%expected costs of all the actions and could choose the best action. The feedback received by the decision
%maker in a given round depends stochastically on the unknown information and the action chosen.
%Bandit problems are a special case of partial monitoring, where the key missing information is the expected
%cost for each action (or arm), and the feedback is simply the noisy version of the expected cost of the action chosen.
%To cast our problem as a partial monitoring problem, \todoc{Do we need this? Or leave this to the reader, just saying that casting our problem as a partial monitoring problem is trivial?} 
%the key unknown information can be the misclassification error rates of the classifiers, an action is identified with 
%the subset of sensors selected, the cost of an action is the sum of the misclassification cost of the classifiers
%that uses the selected sensor subset outputs and the cost of acquiring these sensor outputs,
%while the observed feedback is the vector of predicted labels by each of the classifiers that use 
%the first, the first and second, etc., up to all sensor outputs from the sensors that were selected.
%Note that unlike in a conventional bandit problem, we do not get \emph{direct} 
%feedback of how well our action performed (either noisy or noiseless)\footnote{This problem naturally arises in the surveillance and medical domains. We can perform a battery of tests on an individual in an airport but can never be sure whether or not he/she poses a threat.}.

%Absence of reward information associated with chosen actions is fundamentally challenging since we may not be able to infer potential optimal actions. As usual, in sequential learning under uncertainty, we pose the problem in terms of competitive optimality. In particular we consider a competitor who has the benefit of hindsight and can choose an optimal collection of tests for all the examples. Our goal is to learn the action, while our learning algorithm's performance is evaluated using their  cumulative regret with respect to the competitor.
% is sub-linear (and optimal). 

%We first prove an (unsurprising) result that states 
%with no further assumptions, no learner can ``learn'', i.e., no learner can achieve sublinear regret.
%This negative result led us to introduce the notion of weak dominance on tests. 
%We show that weak dominance is fundamental, i.e., regardless of the algorithm, if this condition is not satisfied, we are left with a linear regret. On the other hand, we develop UCB style algorithms that show that we can realize optimal regret (sub-linear regret) guarantees when the condition is satisfied. 
%Thus, we identify weak dominance as the sharp necessary and sufficient condition for the learnability of
%our sensor acquisition problem.

%The weak dominance condition amounts to a stochastic ordering of the tests on the diagnostic cascade. 
%\todoc{I think I edited out ``cascade'' from the above text.}
%Conceptually, the weak dominance condition says that the child node tends to be relatively more accurate when the parent is correct. \todoc{I would prefer to be more explicit about what this condition is.}
%The weak dominance condition is a somewhat stronger condition than stochastic dominance and ensures that the test accuracy associated with a child node when conditioned on the parent being correct improves over its unconditioned accuracy. 
%Under weak dominance we show that the learner can partially infer losses of the stages. 
%In particular, we reduce the SAP problem to a stochastic multi-armed bandit with side observations, 
%a problem introduced by \citet{MaSh11}.
%In the reduction, the bandit arms are identified by the nodes of the cascade. \todoc{Should we introduce cascade formally then above?}
%The payoff of an arm is given by loss from the corresponding stage, and side observation structure is defined by the feedback graph induced by the cascade. 
%The weak dominance condition occasionally can be shown to hold by design. For example, we do show that,
%in fact, a stronger dominance condition holds in the context of the communication systems 
%and error-correcting code cascades, implying the weak dominance condition there.
%Empirically, with the help of labelled data, 
%we verify that weak dominance condition naturally holds for several real-world problems,
%including diagnosing breast-cancer and diabetes.

%     
%Several papers including \cite{AISTATS13_SupervisedSequentialLearning_TrapezSaligram},\cite{ML13_MultistageClassifier_TrapezSaligramaCastanon}\cite{ICML13_CostSensitiveTreeClassification_XuKusnerChenWeinberger} have considered the problem of learning the best cost effective predictor/classifier using supervised learning methods. The general approach in these methods is to learn a decision function by minimizing an empirical risk objective over a training set. The objective functions in these methods are inherently non-convex and the authors resort to convex relaxations and experimental validations without any theoretical guarantees. However, in many applications gathering training samples may be infeasible, and moreover the labels may not be available at all. We consider an online version of this problem where the samples arrive sequentially and a learner has to decide which sensors to apply for prediction. For each sample, the learner only observes sensor predictions and true label is not revealed. 
%
%In this work we focus on sequential predication of binary labels.  Similar to \cite{ML13_MultistageClassifier_TrapezSaligramaCastanon}, we consider that the order in which sensors are applied is fixed. Typically, the cheapest sensor, or the one with highest error rate, is used first, followed by next cheap sensor with smaller error rate and so on. The sensors thus constitute stages of a cascade, where prediction error rates decrease along the depth, while the costs increase. For each new sample, the learner applies the sensors sequentially and can stop at any stage in the cascade. The goal is to stop at a stage where expected loss is the smallest. Loss at depth $k$ is defined as total cost incurred for acquiring sensor predictions plus a penalty which is $1$ if the prediction of $k^{th}$ sensor is correct, otherwise it is zero. If the learner stops at a depth $k$, he obtains the predictions of all the first $k$ senors as feedback, but which of them are correct is unknown. We refer to this setup as the Sensor Acquisitions Problem (SAP). 
%
%The feedback in SAP do not reveal information about the losses, hence the learner cannot identify the best stage for any sample. 
%We thus focus on scenarios where feedback satisfies some stochastic ordering. Specifically, we assume that if a sensor in the cascade predicts a label correctly, any subsequent sensor also predict it correctly. We refer to this assumption as {\em dominance condition}. When it holds, the learner can partially infer losses of the stages, which, as discussed later, is sufficient to learn the best stage for a given sample. We further demonstrate that under any weaker condition the learner cannot identify the best stage. Dominance conditions holds in many scenarios includes the examples discussed at the beginning. In the wireless communication example, if an error correcting code (ex. Reed-Solomon, LDPC \cite{Book_InferenceLearning_MacKay} recovers information in a channel with certain noise level, then with more redundancy blocks in the error correction code we can certainly recover the information on the channel (though at a lower transmission rate).     
%
%Our first main contribution is to show that if the dominance condition holds the SAP problem can be reduced to 
%a stochastic multi-armed bandit with side observations,
%where bandit arms are identified with the stage of cascade,
%the payoff of an arm is given by loss from the corresponding stage, and side observation structure is defined by the feedback graph induced by the cascade. In particular, we show that the SAPregret
%of any meta-strategy is equal to its bandit-regret
%when the procedure is used to play in the corresponding
%bandit problem. As a consequence, we conclude that existing efficient
%bandit algorithms, as well as their bounds on bandit-regret,
%can be directly applied to achieve new results
%for SAP. Although the underlying
%reduction is straightforward, it gives ready policies with performance guarantees for SAP and their fundamental limitations .
%

\section{Related Work}
%
%Supervised, batch learning, the problem is well studied.

In contrast to our SAP setup there exists a wide body of literature dealing with fully supervised sensor acquisition. Like us \cite{AISTATS13_SupervisedSequentialLearning_TrapezSaligram} also deal with cascade models. However, unlike us these works focus on prediction-time cost/accuracy tradeoffs. In particular they assume that a fully labeled training dataset is provided for test-time use. This dataset has sensor feature data, sensor decisions as well as annotated ground-truth labels. The goal for the learner is to learn a policy for acquiring sensors based on training data to optimize cost/accuracy during test-time. The work of
\cite{poczos2009} decide when to quit a cascade that leads to better decisions to maximize throughput against error rates. Full feedback about classification accuracy is assumed.

\citet{ActiveClass-AIJ-s} consider the problem of PAC learning the best ``active classifier'',
a classifier that decides about what tests to take given the results of previous tests
to minimize total cost when both tests and misclassification errors are priced.
They consider the batch, supervised setting. \todoc{I suspect they assume more than this:
In our previous paper we had a sentence that said that
``their model requires knowing a model of the actions in 
advance'' (this would mean knowing the joint probabilities, I think).}

The literature of learning active classifiers is large \todoc{Actually, much work exists, need to google this}
(e.g., \citep{LCunderBudget-ECML05,ADORE-99,isukapalli01efficient-ICJAI}).


Online learning: In 
\cite{SBCA14:BanditsPaid}, the decision maker can opt to pay for additional observations of the costs associated with other arms. Unlike ours this setting is not unsupervised. In
\citet{ZBGGySz13:CostlyFeatures}, online learning with costly features and labels is studied.
In each round, learner has to decide which features to observe, where each feature costs some money. The learner can also decide not to observe the label, but the learner always has the option
to observe the label. Again this setting is not unsupervised.

Partial monitoring:
General theory of \citet{BaFoPaRaSze14} 
applies to the so-called finite problems (unknown ``key information'') is an element of the probability simplex.
\citet{AgTeAn89:pmon} considers special case when the payoff is also observed (akin to the side-observation problem of \citet{MaSh11}\cite{COLT15_OnlineLearningWithFeedback_AlonBianchiDekel},\cite{NIPS13_FromBanditsToExperts_AlonBianchiGentile}).

Structure of paper


\section{Background}
\label{sec:background}
%!TEX root =  main.tex

%The purpose of this section is to present some necessary background material that will prove to be useful later.
In this section we will introduce a number of sequential decision making problems,
namely stochastic partial monitoring, bandits and bandits with side-observations, which we will build upon later.

First, a few words about our notation: We will use upper case letters to denote random variables.
The set of real numbers is denoted by $\R$. For positive integer $n$, we let
$[n] = \{1,\dots,n\}$. % with $[n]=\emptyset$ if $n=0$.
We let $M_1(\X)$ to denote the set of probability distributions over some set $\X$.
When $\X$ is finite with a cardinality of $d \doteq |\X|$, 
$M_1(\X)$ can be identified with the $d$-dimensional probability simplex.

%\emph{Stochastic partial monitoring problem} is described by a collection of 
%
In a \emph{stochastic partial monitoring problem (SPM)} a learner interacts with a stochastic environment in a sequential manner.
In round $t=1,2,\dots$ the learner chooses an action $A_t$ from an action set $\A$, and receives a feedback $Y_t\in \Y$
from a distribution $p$ which depends on the action chosen and also on the environment instance identified
with a ``parameter'' $\theta\in\Theta$:
$Y_t \sim p(\cdot;A_t,\theta)$. 
%The learner chooses $A_t$ based on the past feedbacks $Y_1,\dots,Y_{t-1}$. 
The learner also incurs a reward $R_t$, which is a function of the action chosen and the unknown parameter $\theta$:
$R_t = r(A_t,\theta)$. 
The reward may or may not be part of the feedback for round $t$.
The learner's goal is to maximize its total expected reward.
The family of distributions $(p(\cdot;a,\theta))_{a,\theta}$ and the family of rewards $(r(a,\theta))_{a,\theta}$
and the set of possible parameters $\Theta$ are known to the learner, who uses this knowledge to judiciously choose
its next action to reduce its uncertainty about $\theta$ so that it is able to eventually converge on choosing only an 
optimal action $a^*(\theta)$, achieving the best possible reward per round, $r^*(\theta) = \max_{a\in \A} r(a,\theta)$.
The quantification of the learning speed is given by the expected regret 
$\Regret_n = n r^*(\theta) - \EE{\sum_{t=1}^n R_t}$, which, for brevity and when it does not cause confusion, 
we will just call regret.
A sublinear expected regret, i.e., $\Regret_n/n \to 0$ as $n\to \infty$ means that the learner in the long run collects
almost as much reward on expectation as if the optimal action was known to it.
%Such a learner is called Hannan consistent. \todoc{Not sure whether Hannan consistency is this, or when the random average regret converges to zero with probability one.}
In some cases it is more natural to define the problems in terms of costs as opposed to rewards;
in such cases the definition of regret is modified appropriately by flipping the sign of rewards and costs. 
%Transforming between costs and rewards is trivial by flipping the sign of the rewards and costs.

%A wide range of interesting sequential learning scenarios can be cast as partial monitoring.
\emph{Bandit Problems} are a special case of SPMs where 
%One special case is bandit problems when 
$\Y$ is the set of real numbers, $r(a,\theta)$ is the mean of distribution $p(\cdot;a,\theta)$ and thus the learner in every round the learner upon choosing an action $A_t$ receives the noisy reward $Y_t \sim p(\cdot;A_t,\theta)$ as feedback. 
%Thus, in a bandit problem in every round the learner chooses an action $A_t$ based on its past observations
%and receives the noisy reward $Y_t \sim p(\cdot;A_t,\theta)$ as feedback. 
%A bandit problem is special in that the observation $Y_t$ and the reward are directly tied.
A finite armed \emph{bandit with side-observations} \cite{MaSh11} is also a special case of SPMs, where the learner upon choosing an action $a \in \A$ receives noisy reward observations, namely, $Y_t  = (Y_{t,a})_{a\in N(A_t)},\,\,Y_{t,a} \sim p_r(\cdot;a,\theta),\,\,\EE{Y_{t,a}} = r(a,\theta)$, from a neighbor-set $\cN(a) \subset \A$, which is a priori known to the learner. 
%
%
%
%Another special case is finite-armed  \emph{bandits with side-observations} \cite{MaSh11},
%where each action $a\in \A$ is associated with a neighbor-set $\cN(a)\subset \A$ and
%the set of neighborhoods is known to the learner from the beginning.
%The learner upon choosing action $A_t\in \A$ receives noisy reward observations for each action in $\cN(A_t)$:
%$Y_t  = (Y_{t,a})_{a\in N(A_t)}$, where $Y_{t,a} \sim p_r(\cdot;a,\theta)$, and $\EE{Y_{t,a}} = r(a,\theta)$.
%(The action chosen may or may not be an element of $N(A_t)$.)
%The reader can readily verify that this problem can also be cast as a partial monitoring problem
To cast this setting as an SPM we let $\Y$ as the set $\cup_{i=0}^K \R^i$ and define 
the family of distributions $(p(\cdot;a,\theta))_{a,\theta}$ such that $Y_t \sim p(\cdot;A_t,\theta)$.
The framework of SPM is quite general and allows for parametric and non-parametric sets $\Theta$. 
%Finally, we note in passing that while we called $\Theta$ a parameter set, 
%we have not equipped $\Theta$ with any structure. As such,
%the framework is able to model both bona fide parametric settings (e.g., Bernoulli rewards) and 
%the so-called non-parametric settings. 
%For example, $K$-armed bandits with reward distributions supported
%over $[0,1]$ can be modelled by choosing $\Theta$ as the set of all $K$-tuples 
%$\theta:=(\theta_1,\dots,\theta_K)$ of distributions over $[0,1]$ and setting $p(\cdot;a,\theta) = \theta_a(\cdot)$.
In what follows we identify $\Theta$ with set of instances $(p(\cdot;a,\theta),r(a,\theta))_{\theta\in \Theta}$.
In other words we view elements of $\Theta$ as a pair $p,r$ where $p(\cdot;a)$ is a probability distribution over $\Y$ for each $a\in \A$ and $r$ is a map from $\A$ to the reals.


\section{Unsupervised Sensor Acquisition Problem}
\label{sec:Setup}
%!TEX root =  main.tex
%The sensors are differentiated in terms of their prediction efficiency and cost. 
\newcommand{\ind}[1]{\mathbb{I}\{#1\}}

\todoc[inline]{I compressed the problem spec. We don't want the reader to get bored.}
The formal problem specification of the unsupervised, stochastic, 
cascaded sensor acquisition problem 
\todoc{I added stochastic and cascaded. Later we may want to consider alternatives,
thus it will be useful to have these so that we can distinguish between the problem defined here and those
future alternatives.}
is as follows: 
A problem instance is specified by a pair $\theta = (P,c)$, where $P$ is
a distribution over the $K+1$ dimensional hypercube, and $c$ is a $K$-dimensional, nonnegative valued vector
of costs.
While $c$ is known to the learner from the start, $P$ is initially unknown.
The instance parameters specify the learner-environment interaction as follows:
In each round for $t=1,2,\dots$, 
the environment generates a $K+1$-dimensional binary vector
$Y = (Y_t,Y_t^1,\dots,Y_t^K)$ chosen at random from $P$.
Here, $Y_t^i$ is the output of sensor $i$, while $Y_t$ is a (hidden) label to be guessed by the learner.
Simultaneously, the learner chooses an index $I_t\in [K]$ and observes the sensor outputs $Y_t^1,\dots,Y_t^{I_t}$.
The sensors are known to be ordered from least accurate to most accurate, 
i.e., $\gamma_k \doteq \Prob{Y_t\ne Y_t^k}$ is decreasing with $k$ increasing.
\todoc{Add proper figure.}
Knowing this, the learner's choice of $I_t$ also indicates that he/she chooses $I_t$ to predict the unknown label $Y_t$.
Observing sensors is costly: The cost of choosing $I_t$ is $C_{I_t} \doteq c_1 + \dots + c_{I_t}$.
The total cost suffered by the learner in round $t$ is thus $C_{I_t} + \ind{Y_t \ne Y_t^{I_t}}$.
The goal of the learner is to compete with the best choice given the hindsight of the values $(\gamma_k)_k$.
The expected regret of learner up to the end of round $n$ is 
$\Regret_n =( \sum_{t=1}^n \EE{ C_{I_t} +\ind{Y_t \ne Y_t^{I_t} }} )- n \min_k (C_k + \gamma_k)$.
For future reference, we let $c(k,\theta) = \EE{ C_{k} +\ind{Y_t \ne Y_t^{k} }}  (= C_k + \gamma_k)$ and $c^*(\theta) = \min_k c(k,\theta)$. Thus, $\Regret_n =( \sum_{t=1}^n \EE{ c(I_t,\theta) }) - n c^*(\theta)$.
In what follows, we shall denote by $\cA^*(\theta)$ the set of optimal actions of $\theta$
and we let $a^*(\theta)$ denote the optimal action that has the smallest index. Thus,
in particular, $a^*(\theta) = \min \cA^*(\theta)$. 
Note that if $i<j$ are optimal actions, then any action in the interval $[i,j] (=[i,j]\cap \N)$ is also optimal.


\if0
A learner has access to $K\geq 2$ sensors that provide predictions
of an unknown label. 
 It is assumed that the sensors form a cascade (cf. \cref{wrap-fig:1}),
i.e., they are  \emph{ordered} in terms of their prediction efficiency,
later sensors are more accurate in predicting the unknown label.
However, acquiring the output of later sensor comes at a fixed cost.
The dilemma of the learner is that while he knows the ordering of the sensors,
the accuracies of the sensors are unknown.
The learner's task is to minimize the total prediction cost, which includes
both the cost of acquiring the sensor outputs and the cost incurred due to imperfect
sensor output.
The learner knows the costs, but does not know how efficient the sensors are
and learns only the output of the sensors.
Learning happens in a sequential setting, where in each round the learner can decide
sequentially (within the round) which sensor outputs to observe,
while respecting the ordering of the sensors.
The output of the last sensor selected serves as the prediction for the round.

The formal specification of the learning problem is as follows:
Learning happens sequentially.
In round $t$ ($t=1,2,\dots$), 
the environment generates 
$(Y_t,\hY_t^1,\dots,\hY_t^K)\in \{0,1\}^{K+1}$ from a distribution $P$ unknown to the learner.

Here, $Y_t$ is the unknown label of context/instance $Z_t$ to be predicted in round $t$, while $\hY_t^k$ is the output of sensor
$k$, a prediction of $Y_t$. We focus on the case where $Z_t$ is not available to the learner. The case where they are observed is briefly discussed in the supplementary. 
At the cost of $c_1+ c_2 + \dots + c_k$,
the learner can choose to acquire the outputs of the first $k$ sensors,
where $k\in [K] := \{1,\dots,K\}$. 

Here, $c_i\ge 0$ is the marginal cost of acquiring the output of sensor $i$.
The costs $c := (c_1,\dots,c_K)$ are known to the learner.
Having acquired the output of the first $k$ sensors, the learner predicts the unknown label $Y_t$ using
the output of the last sensor acquired, i.e., using $\hY_t^k$, making the learner incur the loss
\begin{align*}
L_t(k)=\mathbf{1}_{\{\hat{Y}^k_t\neq Y_t\}}+\sum_{j=1}^k c_j\,
\end{align*}
in round $t$.
The feedback of learner for this round is then $H_t(k)=(\hat{Y}^1_t,\ldots,\hat{Y}^k_t)$.

%Let $\{Z_t, Y_t\}_{{t>0}}$ denote a sequence generated according to an unknown distribution. $Z_t \in\mathcal{C} \subset  \mathcal{R}^d$, where $\mathcal{C}$ is a compact set, denotes a feature vector/context at time $t$ and $Y_t \in \{0,1\}$ its binary label. We denote output/prediction of the $i^{th}$ sensor as $\hat{Y}^i_t$ when its input is $Z_t$. The set of actions available to the learner is $\mathcal{A}=\{1,\ldots, K\}$, where  the action $k \in \mathcal{A}$ indicates acquiring predictions from sensors $1,\ldots, k$ and classifying using the prediction $\hat{Y}^k_t$. 


\begin{wrapfigure}{r}{5cm}
	\vspace{-.5cm}
	\centering
	\includegraphics[scale=.6]{../Figures/SensorCascade.pdf}
	\caption{Cascade of sensors
	}\label{wrap-fig:1}
	\vspace{-.5cm}
\end{wrapfigure} 

\if0
The prediction error rate of the $i^{th}$ sensor is denoted as $\gamma_i:=\Pr\{Y_t\neq \hat{Y}^k_t\}$. The learner incurs an extra cost of $c_k\geq 0$ to acquire output of sensor $k$ after acquiring output of sensor $k-1$. The sensor cascade is depicted in the adjacent figure. In this section we assume that the error rate does not depend on the  context, and the treatment with contextual information is given in the supplementary. 
\fi

%Let $H_t(k)$ denote the feedback observed in round $t$ from action $k$. Since we observe predictions of all the first $k$ senors by playing action $k$, we get   $H_t(k)=(\hat{Y}^1_t,\ldots,\hat{Y}^k_t)$.
%The loss incurred in each round is defined in terms of the prediction error and the total cost involved. When the learner selects action $k$, loss is the prediction error of sensor $k$ plus sum of the costs incurred along the path ($c_1,\ldots,c_k$). Let $L_t: \mathcal{A}\rightarrow \mathcal{R}_+$ denote the loss function in round $t$. Then,
%\begin{equation}
%L_t(k)=\mathbf{1}_{\{\hat{Y}^k_t\neq Y_t\}}+\sum_{j=1}^k c_j.
%\end{equation} 
We refer to the above setup as Sensor Acquisition Problem (SAP).
Based on the previous description, an instance of SAP is the tuple $\psi = (K,P,c)$, where $K\in \mathbb{N}$, $K\ge 2$,
$P$ is a distribution over $\{0,1\}^{K+1}$ and $c\in [0,\infty)^K$. 
 A policy $\pi$ on a $K$-sensor SAP problem
 is a sequence of maps, $(\pi_1, \pi_2, \cdots)$, where
 $\pi_t : \mathcal{H}_{t-1}\rightarrow [K]$ gives the action selected in round $t$
 given a history $h_{t-1}\in \mathcal{H}_{t-1}$ that consists of all actions and corresponding feedback observed before $t$. 
 Let $\Pi$ denote set of such policies. 
 For any $\pi \in \Pi$, we compare its performance to that of the single best action in hindsight 
 and define its expected regret as follows
\begin{equation}
R^\psi_T(\pi)= \mathbb{E}\left[\sum_{t=1}^T L_t(I_t)\right]-\min_{k\in A}\mathbb{E}\left[\sum_{t=1}^T L_t(k)\right],
\end{equation}
where $I_t$ denotes the action selected by $\pi_t$ in round $t$.

The goal of the learner is to learn a policy that minimizes the expected total loss, or, equivalently, to minimize the expected regret, i.e.,
\begin{equation}
\pi^*= \arg \min_{\pi \in \Pi } R^\psi_T(\pi).
\end{equation}

\noindent
{\bf Optimal action in hindsight: } For any $t$, we have 
\begin{equation}
\label{eqn:OptimalAction}
\mathbb{E}[L_t(k)]=\Pr\{Y_t\neq \hat{Y}^k_t\}+\sum_{j=1}^kc_j=\gamma_k +\sum_{j=1}^k c_j\,,
\end{equation}
where $\gamma_k=\Pr\{Y_t\neq \hat{Y}^k_t\}$ is the misclassification error rate of sensor $k$.
Let $k^*=\arg\min_{k\in [K]} \gamma_k + \sum_{i\le  k}c_i$. 
Then the optimal policy is to play action $k^*$ in each round. 
If an action $i$ is played in any round then it adds $\Delta_k:=\gamma_k + \sum_{i\le k}c_i -( \gamma_{k^*} + \sum_{i\le k^*}c_i)$ to the expected regret. 
Let $N_k(s)$ denote the number of times action $k$ 
is selected till time $s$, i.e., $N_k(s)=\sum_{t=1}^s \boldsymbol{1}_{\{I_t=k\}}$. 
Then the expected regret can be expressed as
\begin{eqnarray}
\label{eqn:ExpRegretGap}
R^\psi_T(\pi)&=& \sum_{k \in [K]}\mathbb{E}[N_k(T)]\Delta_k\,.
\end{eqnarray}\
\fi


\section{When is SAP Learnable?}
\label{sec:Learnability}
%!TEX root =  main.tex
\newcommand{\SA}{\mathrm{SA}}
\newcommand{\SD}{\mathrm{SD}}
\newcommand{\WD}{\mathrm{WD}}
\newcommand{\TSA}{\Theta_{\SA}}
\newcommand{\Alg}{\mathfrak{A}}
\newcommand{\TSD}{\Theta_{\SD}}
\newcommand{\TWD}{\Theta_{\WD}}
Let $\TSA$ be the set of all sensor acquisition problems. \todoc{Shall we define ``proper'' problems?}
Thus, $\theta = (P,c)\in \TSA$ such that if $Y\sim P$ then $\gamma_k(\theta):=\Prob{Y\ne Y^k}$ 
is a decreasing sequence.
Given a subset $\Theta\subset \TSA$, we say that $\Theta$ is \emph{learnable} 
if there exists a learning algorithm $\Alg$ such that
for any $\theta\in \Theta$, the expected regret $\EE{ \Regret_n(\Alg,\theta) }$ 
of algorithm $\Alg$ on instance $\theta$ is sublinear.
A subset $\Theta$ is said to be a maximal learnable problem class if it is learnable and for any $\Theta'\subset \TSA$ superset
of $\Theta$, $\Theta'$ is not learnable.
In this section we study two special learnable problem classes, $\TSD\subset \TWD$, where the instances in $\TSD$ are easier to identify in practice, while $\TWD$ can be seen as a maximal extension of $\TSD$.

Let us start with some definitions.
Given an instance $\theta = (P,c)\in \TSA$, we can decompose $P$ into the joint distribution $P_S$ of the sensor outputs $S = (Y^1,\dots,Y^k)$ and the conditional distribution of the state of the environment, given the sensor outputs, $P_{Y|S}$.
Specifically, letting $(Y,S)\sim P$, for $s\in \{0,1\}^K$ and $y\in \{0,1\}$, $P_S(s) = \Prob{S = s}$ and $P_{Y|S}(y|s) = \Prob{Y=y|S=s}$. We denote this by $P = P_S \otimes P_{Y|S}$.
A learner who observes the output of all sensors for long enough is able to identify $P_S$ with arbitrary precision, while $P_{Y|S}$ remains hidden from the learner. This leads to the following simple statement whose proof is left as an exercise:
\begin{proposition}
\label{prop:learnablemap}
$\Theta\subset \TSA$ is learnable if and only if there exists a map $a: M_1( \{0,1\}^K ) \to [K]$ such that 
for any $\theta= (P,c)$, if $P = P_S \otimes P_{Y|S}$ then $a(P_S)$ is an optimal action in $\theta$.
\end{proposition}

A class of sensor acquisition problems that contains instances that satisfy the so-called \emph{strong dominance} condition 
will be shown to be learnable:
\begin{definition}[Strong Dominance]
	An instance $\theta = (P,c)\in \TSA$  is said to satisfy the \emph{strong dominance property} if 
	it holds in the instance that if a sensor predicts correctly
	then all the sensors in the subsequent stages of the cascade also predict correctly, i.e., 
	for any $i\in [K]$,
	\begin{equation}
	\label{eqn:DominanceCondition}
	Y^i=Y \,\, \Rightarrow\,\, Y^{i+1}= \dots =  Y^K = Y
	\end{equation}
	almost surely (a.s.)
	where $(Y,Y^1,\dots,Y^K)\sim P$.
\end{definition}
Let $\TWD = \{ \theta\in \TSA\,:\, \theta \text{ satisfies the strong dominance condition } \}$.
\begin{thm}
The set $\TWD$ is learnable.
\end{thm}
\begin{proof}
We construct a map as required by~\cref{prop:learnablemap}.
Take an instance $\theta = (P,c)\in \TWD$ and let $P = P_S \otimes P_{Y|S}$ be its decomposition
as defined above.
Let $\gamma_i = \Prob{Y^i \ne Y}$, $(Y,Y^1,\dots,Y^K)\sim P$, $C_i = c_1+\dots+c_i$.
For identifying an optimal action in $\theta$, it clearly suffices
to know the sign of $\gamma_i + C_i - (\gamma_j +C_j)$ for all pairs $i,j\in [K]^2$.
Since $C_i - C_j$ is known, it remains to study $\gamma_i-\gamma_j$.
Without loss of generality (WLOG) let $i<j$.
Then, 
\begin{align*}
0 & \le \gamma_i  - \gamma_j = \Prob{Y^i\ne Y} - \Prob{Y^j\ne Y} \\
& = \cancel{\Prob{Y^i\ne Y, Y^i=Y^j}} + \Prob{ Y^i\ne Y, Y^i\ne Y^j } - \\
& - \left\{ 
       \cancel{\Prob{Y^j\ne Y, Y^i=Y^j}} + \Prob{ Y^j\ne Y, Y^i\ne Y^j }\right\}\\
& = \Prob{ Y^i\ne Y, Y^i \ne Y^j } + \Prob{Y^i=Y,Y^i\ne Y^j}       \\
& - \left\{ 
	  \Prob{ Y^j \ne Y, Y^i\ne Y^j } + \Prob{ Y^i=Y,Y^i\ne Y^j}
	 \right\}\\
& \stackrel{\footnotesize (a)}{=} \Prob{ Y^j \ne Y^i } -2 \Prob{ Y^j\ne Y, Y^i = Y } \,,
\numberthis
\label{eq:keyidentity}
\end{align*}
where in $(a)$ we used that $\Prob{ Y^j \ne Y, Y^i\ne Y^j } =  \Prob{ Y^j\ne Y,Y^i= Y}$ and also
$\Prob{ Y^i=Y,Y^i\ne Y^j} = \Prob{ Y^j\ne Y,Y^i= Y}$ because $Y,Y^i,Y^j$ only take on two possible values.
Now, since $\theta$ satisfies the strong dominance condition, $ \Prob{ Y^j\ne Y, Y^i = Y } = 0$.
Thus,
\begin{align*}
\gamma_i - \gamma_j = \Prob{ Y^j \ne Y^i }\,,
\end{align*}
which is a function of $P_S$ only.
Thus, a map as required by~\cref{prop:learnablemap} exists.
\end{proof}
The proof motivates the following definition:
\begin{definition}[Weak Dominance]
	An instance $\theta = (P,c)\in \TSA$  is said to satisfy the \emph{weak dominance property} if 
	it holds that for any $i<j\in [K]$,
\begin{align}\label{eq:wd}
	C_i-C_j \notin [ \Prob{ Y^j \ne Y^i } -2 \Prob{ Y^j\ne Y, Y^i = Y }, \Prob{ Y^j \ne Y^i } ]\,,
\end{align}
where $(Y,Y^1,\dots,Y^K)\sim P$ and $C_i = c_1+\dots+c_i$, $i\in [K]$.
\end{definition}
\todoc[inline]{Note that it is NOT enough to consider $j=i+1$ (i.e., immediate successors) in the definition.
This is because one may need to compare all pairs to find the optimal decision -- comparing successors is not enough.
To see why, consider an instance when $\gamma_1,\gamma_2,\gamma_3$ are given by $0.2,0.2,0$, while
$(C_1,C_2,C_3 )= (0,0.1,0.1)$.
Then, the costs of the actions are $(0.2,0.3,0.1)$. In particular, the first index where we see the total cost grow does not give
the optimal action. Similarly, the last index would not work either.
}
Let $\TWD = \{ \theta\in \TSA\,:\, \theta \text{ satisfies the weak dominance condition } \}$.
Note that $\TSD\subset \TWD$.
\begin{thm}
The set $\TWD$ is a maximal learnable set.
\end{thm}
\begin{proof}
That $\TWD$ is learnable follows from \eqref{eq:keyidentity} combined with~\eqref{eq:wd}.
In particular, fixing $i<j\in [K]$, if $C_j-C_i<\Prob{ Y^j \ne Y^i } -2 \Prob{ Y^j\ne Y, Y^i = Y }$ then
\begin{align*}
\gamma_i+C_i-(\gamma_j+C_j) 
% =    (\gamma_i-\gamma_j) - (C_j-C_i) 
 = \Prob{ Y^j \ne Y^i } -2 \Prob{ Y^j\ne Y, Y^i = Y } - (C_j-C_i)  >  0\,.
\end{align*}
On the other hand, if $C_j-C_i>\Prob{Y^j\ne Y^i}$,
\begin{align*}
\gamma_i+C_i-(\gamma_j+C_j) 
% =    (\gamma_i-\gamma_j) - (C_j-C_i) 
& = \Prob{ Y^j \ne Y^i } -2 \Prob{ Y^j\ne Y, Y^i = Y } - (C_j-C_i) \\
 & <  -2 \Prob{ Y^j\ne Y, Y^i = Y } \le 0 \,.
\end{align*}
In particular, for an instance satisfying the weak dominance condition we see that
$\gamma_i+C_i-(\gamma_j+C_j) <0$ holds if and only if $C_j-C_i>\Prob{Y^j\ne Y^i}$, a condition
that can be verified given the knowledge of $P_S$ alone (where recall that $P = P_S \otimes P_{Y|S}$).
Thus, a map as required by \cref{prop:learnablemap} exists. Note also that this map is uniquely defined for all instances
where the optimal action is unique.
That $\TWD$ is maximal follows because if we add an instance $\theta\in \TSA \setminus \TWD$ 
\todoc[inline]{There is a problem with the definition of maximality. A set can always be ``lucky''..
Non-unique optimal actions also cause problems.}
\end{proof}

\todoc[inline]{Old text}
In the SA-Problem feedback $H_t(\cdot)$ does not reveal any information about the true label $Y_t$ in any round $t$. Hence the loss values are not known, and we are in a hopeless situation where linear regret is unavoidable. In this section we explore conditions that lead to policies that are Hannan consistent \cite{Hannan1957_HannanConsistency_Hannan}, i.e, a policy $\pi\in \Pi^\psi$ such that $R_T^\psi (\pi)/T \rightarrow 0$.

To fix ideas let us consider SA-Problem with $2$ sensors. We enumerate all possible $8$ tuples $(Y, \hat{Y}^1, \hat{Y}^2)$ as shown in Table \ref{tab:SensorOutcomes}, and write probability of $i$th tuple $i=1,2,\cdots 8$ as $p_{i-1}$.  From Table \ref{tab:SensorOutcomes}, we have  $\gamma_1=p_2+p_3+p_4+p_5$ and $\gamma_2=p_1+p_3+p_4+p_6$, thus
\begin{equation}
	\gamma_1-\gamma_2 = p_2+p_5-p_1-p_6.
\end{equation}

\begin{minipage}{0.5\textwidth}
	\vspace{.5cm}
	\begin{tabular}[c]{ c|c|c|c } 
		\label{tab:SensorOutcomes}
	%			\caption{Possible binary tuples}
		$Y$ & $\hat{Y}^1$ & $\hat{Y}^2$ & $\Pr(Y, \hat{Y}^1, \hat{Y}^2)$ \\ \hline 
		$0$ & $0$ & $0$ & $p_0$ \\  \hline
		$0$ & $0$ & $1$ & $p_1$ \\  \hline
		$0$ & $1$ & $0$ & $p_2$ \\  \hline
		$0$ & $1$ & $1$ & $p_3$ \\  \hline
		$1$ & $0$ & $0$ & $p_4$ \\  \hline
		$1$ & $0$ & $1$ & $p_5$ \\  \hline
		$1$ & $1$ & $0$ & $p_6$ \\  \hline
		$1$ & $1$ & $1$ & $p_7$ \\  \hline
		
	\end{tabular}
		\vspace{.5cm}
\end{minipage}\hspace{-1.5cm}
\begin{minipage}[c]{0.6\textwidth}
		\vspace{.5cm}
		\centering
	\hspace{-5cm}		
\begin{equation}
\label{eqn:Marginals}
\Pr(\hat{Y}^1, \hat{Y}^2)=
\begin{cases}
p_1 + p_5 \mbox{  if } (\hat{Y}^1, \hat{Y}^2)=(0,1)\\
p_2 + p_6 \mbox{  if } (\hat{Y}^1, \hat{Y}^2)=(1,0)\\
p_0 + p_4 \mbox{  if } (\hat{Y}^1, \hat{Y}^2)=(0,0)\\
p_3 + p_7 \mbox{  if } (\hat{Y}^1, \hat{Y}^2)=(1,1)
\end{cases}
\end{equation}
	\vspace{.5cm}
\end{minipage}

\noindent
Since we only observe feedbacks $(\hat{Y}_t^1, \hat{Y}_t^2)$ and not the true labels $Y_t$, only marginal probabilities $\Pr(\hat{Y}^1, \hat{Y}^2)$ as given in (\ref{eqn:Marginals}) can be estimated but not $\Pr(Y, \hat{Y}^1, \hat{Y}^2)$. Thus all the decision has to be based on the marginals only. To see when SAP has a Hannan consistent policy, let us consider the following conditions.
\begin{condition}
	\label{cond:PathDominance1}
	 If sensor $1$ predicts label $1$ correctly, then sensor $2$ also predicts it correctly\footnote{Suppose we interpret label $1$ as 'threat', the condition implies that if sensor $1$ detects threat correctly, the better sensor $2$ also detects it. }, i.e.,
	 \begin{equation*}
	 \label{eqn:PathDominace1} 
	 Y_t=1 \mbox{ and } \hat{Y}_t^1=1 \implies \hat{Y}^2_t=1. 
	 \end{equation*}
\end{condition} 
\begin{condition}
		\label{cond:PathDominance2}
	If sensor $1$ predicts label $0$ correctly, then sensor $2$ also predicts it correctly, i.e.,
	\begin{equation*}
	\label{eqn:PathDominace2} 
	Y_t=0 \mbox{ and } \hat{Y}_t^1=0 \implies \hat{Y}^2_t=0. 
	\end{equation*}
\end{condition}
The following example demonstrate  marginals do not unambiguously decide optimal action under Condition \ref{cond:PathDominance1}.
Set $c=0.35$ and consider the following two cases: 1) $p_2=1/2, p_1=1/4-1/40, p_5=1/4+1/40$ and 2) $p_2=1/2, p_1=1/4-3/40,p_5=1/4+3/40$. From Condition (\ref{cond:PathDominance1}) we have $p_6=0$. Also, set $p_0=p_4=p_3=p_7=0$ in both the cases. We get $\gamma_1-\gamma_2=0.3$ in the first case,  whereas $\gamma_1-\gamma_2=0.4$ in the second case. From \ref{eqn:OptimalAction}, optimal action is $1$ in the first case, whereas it is  $2$ in the second case. However, for both the cases the marginals $\Pr(\hat{Y}^1, \hat{Y}^2)$ are the same for all pairs $(\hat{Y}^1, \hat{Y}^2)$. Since we only  observe $\Pr(\hat{Y}^1, \hat{Y}^2)$, the two cases cannot be distinguished and linear regret is unavoidable. We can argue similarly that Condition (\ref{cond:PathDominance2}) is not sufficient for sub-linear regret. 

Next, consider that both Condition (\ref{cond:PathDominance1}) and (\ref{cond:PathDominance2}) hold, i.e.,  
\begin{condition}
	\label{cond:PathDominance}
	If sensor $1$ is correct , then sensor $2$ is also correct, i.e.,
	\begin{equation*}
	\label{eqn:Dominace} 
	 \hat{Y}_t^1=Y_t \implies \hat{Y}^2_t=Y_t. 
	\end{equation*}
\end{condition}
Then, $p_1=p_6=0$ and we get $\gamma_1-\gamma_2=p_2+p_5$. Since $p_2+p_5=\Pr(\hat{Y}^1 \neq \hat{Y}^2)$, it can be estimated from observations $(\hat{Y}_t^1,\hat{Y}_t^2)$, and the optimal action can be found unambiguously. Thus Condition \ref{cond:PathDominance} gives a sufficient for existence of an Hannan consistent policy. In the following we refer to Condition (\ref{cond:PathDominance}) as strong dominance property. For the case of $K>2$ sensors, its definition is as follows: 

\begin{definition}[Strong Dominance]
	A SA-Problem is said to satisfy strong dominance property if sensor $i$ predicts correctly, then all the sensors in the subsequent stages of the cascade also predict correctly, i.e.,
	\begin{equation}
	\label{eqn:DominanceCondition}
	\hat{Y}_t^i=Y_t \rightarrow \hat{Y}_t^j \quad \forall j>i\geq 1.
	\end{equation}
\end{definition}

We will now establish necessary and sufficient conditions for SAP learnability
For notional convenience rewrite 
$\gamma_1- \gamma_2= p_1+p_2+p_5+p_6- 2(p_1+p_6):=p_{12}-2\delta$,
where $p_{12}:=\Pr(Y^1\neq Y^2)$ is the probability that sensors disagree and $\delta:=\Pr(Y^2 \neq Y | Y^1=Y)$ is the conditional probability that sensor $2$ is incorrect given that sensor $1$ is correct. We can estimate $p_{12}$ from feedback $(\hat{Y}^1_t, \hat{Y}^2_t)$, but $\delta$ cannot be estimated.
\begin{thm}
For SA-Problem with $K=2$, an Hannan consistent policy exists if and only if $c \notin [p_{12}-2\delta, p_{12}]$.
\end{thm}
{\bf Proof:} Under dominance condition $\delta=0$, thus actions $1$ is optimal if $p_{12}<c$, otherwise action $2$ is optimal. Suppose dominance condition is violated, i.e., $\delta>0$, but decisions are made assuming dominance condition holds (i.e., using estimates of $p_{12}$ only), then the optimal action is correctly identified provided $\delta$ is such that $p_{12}-2\delta < c \Rightarrow p_{12} <c$ or $p_{12}-2\delta >c \Rightarrow p_{12}>c$. Now, notice that the latter implication is always true. So, whenever action $2$ is optimal, violation of dominance condition does not miss the optimal action. However, the first implication holds if and only if $c \notin [p_{12}-2\delta, p_{12}]$.

Clearly, when $\delta$ is small Hannan consistent policy exits for a large range of $c$. For the case of $K>2$ sensors, its definition is as follows: 

\begin{definition}[Weak Dominance]
	A SA-Problem is said to satisfy weak dominance property if $c_k \notin [p_{k-1,k}-2\delta_{k-1,k}, p_{k-1,k}]$ for all $1<k<K$, where $p_{k-1,k}=\Pr(Y^{k-1}\neq Y^k)$ and $\delta_{k-1,k}=\Pr(Y^k\neq Y| Y^{k-1}=Y).$
\end{definition}

Many real world applications are designed to satisfy strong dominance property. For example, in wireless communication, increasing block length (more redundancy) improves tolerance against noise. Many practical datasets like, PIMA diabetes dataset and breast cancer dataset, conditional error probabilities are small.   (i will add numerical values)

In the following we establish that if dominance property holds efficient algorithms for a SAP problem can be derived from algorithms on a suitable stochastic multi-armed bandit problem. We first recall the stochastic multi-armed bandit setting and the relevant results. 

%\section{Problem Setup}
%\label{sec:Setup}
%We consider the problem of efficient label prediction under partial monitoring. Let $\{Y_t\}_{{t>0}}$ denote sequence of labels generated according to an unknown distribution $\mathcal{D}$.  The learner can use a `cheap' sensor (device-1) or/and a `costly' sensor (device-2) to predict the labels. In round $t$, let $\hat{Y}^1_t$ and $\hat{Y}^2_t$ denote the predictions of device-1 and device-2 respectively. We assume that device-1 has lower performance than device-2 in the sense that prediction error rate of device-1, denoted as 
%$\gamma_1:=\Pr\{Y_t\neq \hat{Y}^1_t\}$, is larger than or equal to that of device-2, denoted as $\gamma_2:=\Pr\{Y_t\neq \hat{Y}^2_t\}$ ($\gamma_1>\gamma_2$). In each round $t$, the learner can take the following actions: 
%\begin{itemize}
%	\item Action-1: use device-1.
% \item Action-2: use both the devices. 
%\end{itemize}
%For ease of notion, we denote actions by their index, and write $\mathcal{A}=\{1,2\}$ for the set of actions and use $i$ to index them. Let $F_t(i)$ denote the feedback observed in round $t$ by selecting action $i$. When $i=1$, the learner observes $\hat{Y}^1_t$, and when $i=2$, he observes both $\hat{Y}^1_t$ and $\hat{Y}^2_t$. That is, 
%\begin{equation}
%F_t(i)=\begin{cases}
%\hat{Y}_t^1 \;\;\mbox{if}\;\;i=1,\\
%\{\hat{Y}^1_t, \hat{Y}^2_t\} \;\;\mbox{if}\;\;i=2.
%\end{cases}
%\end{equation} 
%The loss incurred in each round is defined as follows. When $i=1$, the loss is $1$ unit if prediction of device-1 (observed feedback) is incorrect, otherwise loss is zero. When $i=2$, a fixed loss of $c>0$ is incurred in addition to the prediction loss of device-2, which is $1$ unit if device-2's prediction is incorrect and $0$ otherwise. Let $L_t(i)$ denote the loss in round $t$ for taking action $i\in \mathcal{A}$. Then,
%\begin{equation}
%L_t(i)=\begin{cases}
%\boldsymbol{1}_{\{\hat{Y}^1_t \neq Y_t\}} \;\;\mbox{if}\;\;i=1,\\
%\boldsymbol{1}_{\{\hat{Y}^2_t\neq Y_t\}}+c \;\;\mbox{if}\;\;i=2.
%\end{cases}
%\end{equation} 
%Let $H_{t}$ denote the set of actions played and the corresponding feedback observed till time $t$.  A policy $\pi=(\pi_1, \pi_2, \cdots)$, where  $\pi_t : H_{t-1}\rightarrow
%\mathcal{A}$, gives action selected in each round using all the feedback observed in the past. The expected regret of a policy $\pi$ that selects action $\pi_t \in\mathcal{A}$ in round $t$ over a period $T$ with respect to the best action in hindsight is given as 
%\begin{equation}
%\label{eqn:Regret2Actions}
%R_T(\pi)= \mathbb{E}\left [\sum_{t=1}^TL_t(\pi_t)\right]
%-\min_{i\in \mathcal{A}}\mathbb{E}\left [\sum_{t=1}^T L_t(i)\right ].
%\end{equation}
%The goal of the learner is to learn a policy that minimizes the maximum expected regret, i.e.,
%\begin{equation}
%\pi^*= \arg \min_{\pi \in \Pi }R_T(\pi)  ,
%\end{equation}
%where $\Pi$ denote the set of policies that maps past history to an action in $\mathcal{A}$ in each round. 
%
%\noindent
%{\bf Optimal action in hindsight: } For any $\pi \in \Pi$ and in any round $t$ we have 
%\begin{equation}
%\mathbb{E}[L_t(i)]=
%\begin{cases}
%\gamma_1\;\; \mbox{if}\;\; i=1,\\
%\gamma_2+c\;\; \mbox{if}\;\;i=2,
%\end{cases}
%\end{equation}
%Let  $i^*=\arg\min_{i \in \mathcal{A}} {E}[L_t(i)]$ denote optimal action. Then, $i^*=1$ if $\gamma_1 \leq \gamma_2 +c$, and $i^*=2$ otherwise. Let $I_t$ denote the action taken in round $t$ and $N_i(s)$ denote the number of times action $i$ is selected till time $s$, i.e., $N_i(s)=\sum_{t=1}^s \boldsymbol{1}_{\{I_t=i\}}$. The expected regret can be expressed as
%\begin{eqnarray}
%\label{eqn:ExpRegretGap}
%R_T(\pi)&=& \sum_{i=1}^{2}\mathbb{E}[N_i(T)]\Delta_i 
%\end{eqnarray}
%where $\Delta_1=\gamma_1 -\mathbb{E}[L(i^*)]$ and $
%\Delta_2=\gamma_2+c -\mathbb{E}[L(i^*)]$. Note that for all $i=1,2$, either $\Delta_i=|\gamma_1-\gamma_2-c|$ or $\Delta_i=0$.
%
%
%\noindent
%{\bf Assumptions (Dominance condition):}   Whenever device-1 makes no prediction error, device-2 is also guaranteed to make no prediction error, i.e., in every round $t$, 
%\begin{equation}
%\label{eqn:DomAssum}
%{\hat{Y}^1_t=Y_t} \implies {\hat{Y}^2_t=Y_t}.
%\end{equation}  
%\noindent
%{\bf Reduction to the apple tasting problem:}
%The feedback from action $i=1$ reveals no information about the loss incurred in that round. However feedback after action $i=2$ reveals (partial) information about the loss of both actions. Suppose feedback is such that the predictions of devices disagree, i.e., ${\hat{Y}^1_t\neq\hat{Y}^2_t}$ after action $2$.  The dominance condition then implies that the only possible events are $\hat{Y}^1_t \neq Y_t$ and $\hat{Y}^2_t=Y_t$. I.e., the true label is that predicted by device-2 and loss is zero. Suppose the predictions of the devices agree, i.e., ${\hat{Y}^1_t = \hat{Y}^2_t}$, then the dominance condition implies that either predictions of both are correct or both are incorrect. Though the true loss is not known in this case, the learner can infer some useful knowledge: in round $t$, if the prediction of both the devices agree, then the difference of loss of the actions is $L_t(2)-L_t(1)=c>0$. And if the predictions of the devices disagree, then dominance assumption implies that $L_t(1)=1$ and $L_t(2)=c$ or $L_t(2)-L_t(1)=c-1<0$. Thus, each time learner selects action $2$, he gets to know whether or not he was better off by selecting the other action. This setup is similar the standard apple tasting problem \cite{IC2000_AppleTasting_HelmboldLittlestoneLong}
%], but differs in terms of the information structure when action $2$ is played: in the apple tasting problem, playing action $2$ in a round reveals loss incurred by both the actions. Whereas, in the sensor selection problem we get only partial information on which of the two actions is better in that round. However, we will see below that the partial information is enough to distinguish the optimal arm and one can obtain performance similar to that in the apple tasting problem .



%\section{Stochastic Multi-armed Bandits with Side Observations}
%%!TEX root =  main.tex
A stochastic multi-armed bandit (MAB), denoted as $\phi:=(K, (\nu_k)_{1 \leq k \leq K})$, is a sequential learning problem where number of arms $K$ is known and each arm $i \in [K]$ gives rewards drawn according to an unknown distribution $\nu_k$. Let $X_{i,n}$ denote the random reward from arm $i$ in its $n$th play. For each arm $i\in [K]$, $\{X_{i,t}: t>0\}$ are independently and identically (i.i.d) distributed and for all $t>0$, $\{X_{i,t}, i \in [K]\}$ are independent. We note that in the standard MAB setting the learner observes only reward from the selected arm in each round and no information from the other arms is revealed. However, in many applications playing an arm reveals information about the other arms which can be exploited to improve learning performance. Let $\mathcal{N}_i$  denote neighborhood of $i$ such that playing arm $i$ reveals rewards of all arms $j \in \mathcal{N}_i$.  Given a set of neighborhood $\{\mathcal{N}_i, i\in [K]\}$, let $\phi_G:=(K, (\nu_k)_{1\leq k\leq K}, G)$ denote a MAB with side-information graph $G=(V,E)$, where $|V|=K$ and $(i,j)\in E$ if $j \in \mathcal{N}_i$. The side-observation graph is known to the learner and remains fixed during the play. To avoid cluttering, we henceforth drop subscript $G$ in $\phi_G$ and it should be clear from context if side-observations exists or not. 

Let $\Pi^{\phi}$ denote a set of polices on $\phi$ that maps the past history into an arm in each round.. If the learner knows $\{\nu_k\}_{k \in [K]}$, then the optimal policy is to play the arm with highest mean. Given a policy $\pi \in \Pi^{\phi}$, its performance is measured with respect to the optimal policy and is defined in terms of expected cumulative regret (or simply regret) as follows ( only reward from the arm played contribute to the regret and not that from the side-observations):  Let $\pi$ selects arm $i_t$ in round $t$. After $T$ rounds, its regret is 
\begin{equation}
\label{eqn:BanditRegret}
R^\phi_T(\pi)= T \mu_{i^*}- \sum_{t=1}^{T}\mu_{i_t},
\end{equation} 
where $\mu_i=\mathbb{E}[X_{i,n}]$ denotes mean of distribution  $\nu_i$ for all $i\in [K]$ and  $i^*= \arg\max_{i \in [K]} \mu_i$. Let $N^\phi_i(t)=\sum_{s=}^{t}\boldsymbol{1}\{i_s=i\}$ denote the number of pulls of arm $i$ till time $t$. Then, the regret of policy $\pi$ can be expressed 
\[R^\phi_T(\pi)=\sum_{i=1}^{K}(\mu_{i^*}-\mu_i)\mathbb{E}[N^\phi_i(T)].\]
The goal is to learn a policy that minimizes the regret.  

%MAB problems have been extensively studied in the literature. The seminal paper of Lai \& Robbins 
%\cite{AAM85_Asymptotically_LaiRobbins} showed that for any consistent policy (that plays sub-optimal arms only sup-polynomially many times in the time horizon) the regret grows logarithmically in time horizon. Specifically, for a class of parametric reward distributions, they showed that regret of any consistent policy satisfies
%\begin{equation}
%\label{eqn:MABLowerBound}
%\liminf_{n \rightarrow \infty}\frac{ R^\phi_T(\pi)}{\log T} \geq \sum_{i\neq i^*} \frac{\mu_{i^*}-\mu_i}{D(\mu_{i^*}||\mu_i)},
%\end{equation}
%where  $D(p||q)$ is the KL-divergence of $p,q \in [0\; 1]$. Further, the authors in \cite{AAM85_Asymptotically_LaiRobbins}  provided an upper confidence bound (UCB) based policy that asymptotically achieves the lower bound for a class of parmetric reward distributions.
% 
%Auer et, al. \cite{ML02_FiniteTimeAnalysis_AuerBianchiFischer} proposed an anytime policy named UCB1, that is based on the UCB strategy and showed that it is optimal on any MAB with bounded rewards. Specifically, they showed that regret of UCB1 for any $T>K$ is upper bound as 
%\begin{equation}
%\label{eqn:UCB1UpperBound}
%R^\phi_T(\mbox{UCB1})\leq \sum_{i\neq i^*} \frac{8\log n}{\mu_{i^*}-\mu_i} + (\pi^2/3 + 1)(\mu_{i^*}-\mu_i).
%\end{equation}
%Thus demonstrating the optimality of UCB1. Since the work of Auer et. al. several works have proposed improvised UCB based policies like, KL-UCB \cite{COLT11_TheKL-UCBAlgorithm_GarivierCappe}, MOSS \cite{JMLR2010_RegretBoundsAndMinimax_AudibertBubeck}.
%
%
%\subsection{MAB With Side Information}
%
%
%Let $\Pi^{\phi_G}$ denote the set of all policies on $\phi_G$ that map the past history (including the side-observations) to an action in each round. For any policy $\pi  \in \Pi^{\phi_G}$, we denote the regret over a period $T$ as $R^{\phi_G}_T(\pi)$ and is given by (\ref{eqn:BanditRegret}). Note that, in each round, 
%only reward from the arm played contribute to the regret and not that from the side-observations. 
\cite{Sigmetrics15_StochasticBanditsWithSideObservations_BuccapatnamEriyilmazShroff} establish that any policy $\pi \in \Pi^{\phi} 
$ where side observation graph is such that $i \in \mathcal{N}_i$ for all $i\in [K]$ satisfies
%\begin{equation}
%\label{RegretBandit}
%R^B_T(\pi)= \max_{1\leq i\leq K}\mathbb{E}\left[\sum_{t=1}^{T}X_{i,t}\right]- \mathbb{E}\left[\sum_{i=1}^{T}X_{k_t,N_{k_t}(t)}\right],
%\end{equation} 
\begin{equation}
	\liminf_{T \rightarrow \infty} R^{\phi}_T(\pi)/\log T \geq \eta(G)
	\end{equation}
		where $\eta(G)$ is the optimal value of the following linear optimization
	\begin{align}
	& \mbox{LP1}:\; \;\displaystyle\min_{\{w_i\}}\sum_{i \in [K]}(\mu_{i^*}- \mu_i) w_i \nonumber\\
	\label{eqn:LowerBoundLP}
	& \mbox{subjected to} \sum_{j \in \mathcal{N}_i}w_i\geq 1/D(\mu_i || \mu_{i^*}) \mbox{ and } w_i \geq 0 \mbox{  for all } i\in [K],
%	& w_i \geq 0 \mbox{ for all } i \in [K] \nonumber
	\end{align}
$D(\mu_i || \mu_{i^*})$ here denotes the Kullback-Leibler divergence between $\nu_i$ and $\nu_{i^*}$. 
When $\mathcal{N}_i=\{i\}$ for all $i\in [K]$, it reduces to the classical lower bound of $\sum_{i\neq i^*}(\mu_{i^*}- \mu_i)/D(\mu_i || \mu_{i^*})$ established in \cite{AAM85_Asymptotically_LaiRobbins}. Further, \cite{Sigmetrics15_StochasticBanditsWithSideObservations_BuccapatnamEriyilmazShroff} also gave an UCB based strategy, named UCB-LP, that explores arms at a rate in proportion to the size of their neighborhood. Specifically, UCB-LP plays arms in proportions to the values $\{z_i^*, i\in [K]\}$ computed from the following linear optimization which is a relaxation of LP1. 

	\begin{equation}
	\label{eqn:LowerRelaxedBoundLP}
	\mbox{LP2}: \displaystyle\min_{\{z_i\}}\sum_{i \in [K]} z_i 
	 \mbox{ subjected to } \sum_{j \in \mathcal{N}_i}z_i\geq 1 \mbox{ and } z_i \geq 0 \mbox{  for all } i\in [K]
	%	&  \mbox{ for all } i \in [K] \nonumber
	\end{equation}
%	\begin{align}
%	& LP2: \displaystyle\min_{\{z_i\}}\sum_{i \in [K]} z_i \nonumber\\
%	\label{eqn:LowerRelaxedBoundLP}
%	& \mbox{subjected to} \sum_{j \in \mathcal{N}_i}z_i\geq 1 \mbox{ and } z_i \geq 0 \mbox{  for all } i\in [K]
%%	&  \mbox{ for all } i \in [K] \nonumber
%	\end{align}

The regret of UCB-LP is upper bounded by 
\begin{equation}
\label{eqn:UCBLPUpperBound}
\mathcal{O}\left(\sum_{i\in [K]} z_i^* \log T\right) +\mathcal{O}(K\delta),
\end{equation}
where $\delta= \max_{i \in [K]}|\mathcal{K}_i|$ and $\{z_i^*\}$ are the optimal values of $LP2$. 
\begin{definition}[Domination number \cite{Sigmetrics15_StochasticBanditsWithSideObservations_BuccapatnamEriyilmazShroff}]
	Given a graph $G=(V,E)$, a subset $W\subset V$ is a dominant set if for each $v\in V$ there exists $u \in W$ such that $(u,v)\in E$. The size of the smallest dominant set is called weak domination number and is denoted as $\xi(G)$. 
\end{definition} 	
Since any dominating set is a feasible solution of LP2, we get $\sum_{i\in [K]}z_i^*\leq \xi(G)$, and the regret of UCB-LP is $\mathcal{O}(\xi(G)\log T)$. 
%\subsection{Special case: 1-armed bandit}
%In the MAB problem when all the arms have a fixed reward except for one, we get 1-armed bandit. 
%The learner knows the arms that give fixed reward the goal is to identify the quality of the arm that gives stochastic reward as fast as possible. A straightforward modification of UCB1 achieves optimal regret of $\Theta(\log T)$.
  

\section{Regret Equivalence}
\label{sec:Equiv}
%!TEX root =  main.tex
In this section we establish that SAP with strong dominance property is `regret equivalent' to an instance of MAB with side-information and the corresponding algorithm for MAB can be suitably imported to solve SAP efficiently.   
 \noindent
\begin{definition}[Regret Equivalence]
	Consider a SA-Poblem $\psi:=(K, P, c)$ and a bandit problem $\phi:=(N, (\nu_i)_{i \in [N]},G)$ with side-information graph $G$. We say that $\psi$ is regret-equivalent to $\phi$ if given a policy $\pi$ for $\psi$, one can come up with a policy $\pi^\prime$ that uses $\pi$, such that the regret of $\pi^\prime$ on any instance of $\phi$ is the same as the regret of $\pi$ on some corresponding instance of $\psi$, and vice versa. 
\end{definition}	
In the following we first consider SAP with $2$ sensors and then the general case with more than $2$ sensors. SAP with $2$ sensors is useful to draw comparison with the well studied apple tasting problem \cite{IC2000_AppleTasting_HelmboldLittlestoneLong} and understand role of the dominance property. 
\subsection{SAP with two sensors}
In SAP with two sensors, while action $1$ reveals no information about the loss values, under dominance property, action $2$ reveals (partial) information about the loss from both actions. To see this, let $I_t=2$. If predictions of sensors disagree, i.e., $\hat{Y}^1_t\neq\hat{Y}^2_t$, then  dominance property implies that only sensor $2$ is correct, i.e., $\hat{Y}^1_t \neq Y_t$ and $\hat{Y}^2_t=Y_t$. Hence $L_t(1)=1$ and $L_t(2)=c$. On the other hand, if predictions agree, i.e., $\hat{Y}^1_t = \hat{Y}^2_t$, then either predictions of both are correct or both are incorrect, and we can only infer that $L_t(2)-L_t(1)=c_1+c_2>0$. Thus, each time learner plays action $2$,  loss from both actions is known only when sensor output disagree, otherwise. 

\noindent
{\bf Apple tasting \cite{IC2000_AppleTasting_HelmboldLittlestoneLong}:} In this problem, a learner gets a sequence of apples some of which can be rotten. In each round, the learner can either accept or reject an apple and irrespective of his action, a penalty is incurred if the apple is rotten in that round. If an apple is rejected, learner do not get to observe its quality, and if accepted, the learner tastes the apple and knows its quality. In the latter case loss incurred is known, and the learner can also know the loss he would have incurred if he opted to reject it.  %Thus, each time the learner accepts an apple, he knows the penalty incurred for both the actions. However, no loss values is revealed if he rejects. 
The goal of the learner is to taste more good apples. 
A SAP with dominance property is thus a general version than the apple tasting problem as unlike in apple tasting problem loss value are revealed only in few instances.  We next show that SAP satisfying dominance property can be efficiently solved.  
%
%\begin{thm}
%	\label{thm:2SAPRegret}
%Assume dominance condition (\ref{eqn:DominanceCondition}) holds. Then SAP $\psi$ with $K=2$ is regret-equivalent to a stochastic $1$-armed bandit.
%\end{thm}
%%a_i\mathbb{E}[N_i(T)]$ where $\Delta_1= c- \min\{\gamma_1-\gamma_2,c\}$ and $\Delta_2= \gamma_1-\gamma_2- \min\{\gamma_1-\gamma_2,c\}$, which can also be expressed as $\Delta_1=\gamma_1 - \mathbb{E}[L(i^*)]$ and  $\Delta_2=\gamma_2+c - \mathbb{E}[L(i^*)]$, implying that the corresponding policy on the SAP also gives the same regret and vice versa. 
%The following corollary follow immediately from the regret equivalence. 
%\begin{proposition}[SAP regret lower bound]
%	Let $\pi$ be any policy on SAT with 2 sensors such that it pulls the suboptimal arm only sub polynomial many times, i.e., $\mathbb{E}[N_i(T)]=o(T^a)$ for all $a>0$ and $i\neq i^*$. Then,
%	\begin{equation}
%	\liminf_{T \rightarrow \infty} R^\psi_T(\pi)/\log T \geq \frac{|\gamma_1-\gamma_2-c|}{D(\hat{\gamma},\gamma^*)} \mbox{ where } \gamma^*=\min\{\gamma_1,\gamma_2+c\},  \hat{\gamma}=\max\{\gamma_1,\gamma_2+c\}
%	\end{equation}
%	and $D(\hat{\gamma},\gamma^*)$  is the KL-divergence between $\hat{\gamma}$ and $\gamma^*$.
%\end{proposition}
%
%\begin{proposition}[SAP regret upper bound]
%	Let $\pi^\prime$ denote a policy on a $1$-armed stochastic bandit where one arm has mean $\gamma_1-\gamma_2$ and the other gives fixed reward $c$. Then, the regret of a policy $g(\pi)$ for the SAT problem obtained according the mapping (\ref{eqn:1BanditToSAP}) is upper bounded as
%	\begin{equation}
%	R^\psi_T(g(\pi))\leq \frac{6\log T}{|\gamma_1-\gamma_2-c|} + |\gamma_1-\gamma_2-c|(1+\pi^2/3) 	\mbox{ when } \pi^\prime=\mbox{UCB1}. 
%	\end{equation}
%		\begin{equation}
%		R^\psi_T(g(\pi))\leq \frac{|\gamma_1-\gamma_2-c|\log T}{D(\hat{\gamma},\gamma^*)} + \mathcal{O}(\sqrt{\log T }) 	\mbox{ when } \pi^\prime=\mbox{KL-UCB}. 
%		\end{equation}
%\end{proposition}
\subsection{SAP with more than two actions}

\begin{wrapfigure}{r}{6cm}
	\centering
	\includegraphics[scale=.4]{../Figures/SideInfoGraph.pdf}
	\caption{Side observation graph $G_S$}
	\label{fig:SideObservationGraph]}
\end{wrapfigure} 
In SAP with two sensors, only action $2$ provides partial information about the loss of both actions. In the case with $K>2$ sensors, by playing an action $k$, partial information about the loss from actions $l<k$ can be inferred by recursively applying the dominance property to each pair of sensors.  Further, any information provided by action $k>2$ is contained in that provided by all actions $k^\prime\geq k$ as $H_t(k) \subseteq H_t(k^\prime)$. 

This information structure can be represented by a directed graph $G_S=(V,E)$, where $|V|=K$ and $E=\{(i,j): 1<i\leq j\leq K \}$. Note that $G_S$ has self loops for all nodes except for node $1$. The nodes in $G_S$ represents actions of in SAP and an edge $(i,j)\in E$ implies that actions $i$ provides information about action $j$. The side-observation graph for the SAP is shown in Figure (\ref{fig:SideObservationGraph]}).

We now have all the ingredients to relate SAP problem with MAB.
\begin{thm}
	\label{thm:K-SAPRegret}
Let the dominance condition (\ref{eqn:DominanceCondition}) holds. Then SAP is regret equivalent to a MAB with side-observation graph $G_S$. 
\end{thm}
Then, from (\ref{eqn:MABLowerBound}), we immediately obtain following regret lower bound for SAP. 
\begin{proposition}[SAP regret lower bound]
	Let $\pi$ be any policy on SAP such that it pulls the suboptimal arm only sub polynomial many times, i.e., $\mathbb{E}[N^\psi_i(T)]=o(T^a)$ for all $a>0$ and $i\neq i^*$. Then,
	\begin{equation}
	\liminf_{T \rightarrow \infty} R^\psi_T(\pi)/\log T \geq \kappa \mbox{	where  }
	\end{equation}
		\begin{align}
	\kappa=&\displaystyle\min_{\{w_i\}}\sum_{i \in [K]}(\mu_{i^*}- \mu_i) w_i \nonumber\\
	& \mbox{subjected to} \sum_{j i}w_i\geq 1/D\left(\mu_i + \sum_{j<i} c_j \| \mu_{i^*} + \sum_{j<i^*} c_j \right ) \mbox{  for all } i\in [K]\\
	& w_i \geq 0 \mbox{ for all } i \in [K]. \nonumber
	\end{align}
\end{proposition}

\begin{proposition}[K-SAT regret upper bound]
Given a SA-problem $\psi$,	there exists a policy $\pi\in \Pi^\psi$ such that
\begin{equation}
	R^\psi_T(\pi)\leq \mathcal{O}(\log T) + \mathcal{O}(K^2). 	
\end{equation}
\end{proposition}
As discussed in the proof of Theorem \ref{thm:K-SAPRegret}, using  
 UCB-LP on side-observation graph $G_S$ we can obtain a policy for SAP that maintains regret guarantee of UCB-LP which is given as $\mathcal{O}(\xi(G_S)\log T) + \mathcal{O}(K^2)$. Now the claim follows by noting that $\xi(G_S)=1$. 



%
%\section{Extension to Multi-Stage and Multi-Action setting}
% 
% \subsection{Information and Side Observations:}
% on the feedback observed, we continue to assume that the dominance condition holds across the stages, i.e., for all $1\leq i\leq K-1$
% \begin{equation}
% \label{eqn:DominanceMultiStage}
%\mbox{ if } \hat{Y}_t^i=Y_t  \implies  \hat{Y}_t^j=Y_t \mbox{ for all } i<j\leq K
% \end{equation}
% Then, each time action $i>1$ is played, we get the same type of information about the loss incurred by an action pair  $(j,j-1)$ as in the SAP problem using the pair  $\{\hat{Y}_t^j,\hat{Y}_t^{j-1}\}$ for each $1<j<i$. Thus, playing action $i>1$ provides side observations about all the actions $j\leq ii$. We refer to this setting as Multi stage Sensor Acquisition Problem (MSAP). 
% 
% The side-observation structure in the MSAP problem can be represented by a directed graph $G=(\mathcal{V}, \mathcal{E})$ where $|\mathcal{V}|=K$ and $\mathcal{E}=\{(i,j) \in \mathcal{V}\times \mathcal{V}: i\geq j, i>1\}$. Here,  $(i,j)\in \mathcal{E}$ implies that selecting $i$ provides information about the prediction loss of action $j$. The side-observation graphs is depicted in Figure (\ref{fig:SideObservationGraph]}). Note that an edge $(i,j) \in \mathcal{E}$ only implies that playing action $i$ provides some information about the losses of actions $i\leq j$, but not their true losses. In the following we establish that regret of MSAP is equivalent to a stochastic $K$-armed bandit with the same side-observation structure.
%  \subsection{Regret Equivalence}
% \begin{thm}
% 	A MSAP $(K, \mathcal{A}, (\gamma_i,c_i)_{i\in [K]}, (L(i), F(i))_{i \in \mathcal{A}} )$ with $K>2$ is regret-equivalent to a stochastic bandit problem $(N, (\nu_i)_{i \in [N]})$ with $N=K$ and side observation structure give by $G$.
% \end{thm}	
%Consider a $K$-armed stochastic bandit problem where  reward distribution $\nu_k$ has mean $\gamma_1-\gamma_k + \sum_{j=2}^k c_j$ for all $1\leq k\leq K$, and the side-observation from arms is given by graphs $G$ . Given an arbitrary policy $\pi$ for the MSAP  that uses the side-information, we obtain a  policy BanditG($\pi$) for the bandit problem as follows: if BanditG($\pi$) played arm $i\neq 1$ in the previous round, it inputs the feedback observed from all arms $i\leq j$ to $\pi$ and copies $\pi$'s choice for next action. If BanditG($\pi$)  played arm $1$ in the previous round, it simply copies $\pi$'s choice for next action. Conversely, suppose $\theta$ is an arbitrary policy for the bandit problem with side-observation structure $G$, let MSAP($\theta$) denote a policy for MSAP that consults $\theta$ as follows: if MSAP($\theta$) played action $i$ in the previous round it inputs feedback vector $(0, \boldsymbol{1}_{\{\hat{Y}_t^j\neq \hat{Y}_t^{j-1} \}}, \forall \; j\leq 1)$ to $\theta$ and copies its choice for next action. Otherwise it simply copies $\theta$'s choice for next action.
%
%We next show that regret of BanditG($\pi$) on the bandit problem is same as that of $\pi$ on MSAP, 
% and regret of MSAP($\theta$) on MSAP is same as regret of $\theta$ on the bandit problem. 
% 
%For any strategy $\pi$, the expected regret of the MSAP can be expressed as 
%\begin{eqnarray}
%R_T(\pi) &=&\sum_{i=1}^{K}\left [ \left (\gamma_{i}+\sum_{j\leq i} c_j\right )-\left (\gamma_{i^*}+\sum_{j\leq i^*} c_j\right )\right ]\mathbb{E}[N_i(T)]\\
%&=&\sum_{i=1}^{K} \left[\left (\gamma_1+c_1-\gamma_{i^*}-\sum_{j\leq i^*} c_j \right )-\left (\gamma_1+c_1- \gamma_{i}-\sum_{j\leq i} c_j \right )\right ]\mathbb{E}[N_i(T)]
%\end{eqnarray} 
%where we added and subtracted $\gamma_1+c_1$ from each term. Now, notice that this is the expected regret of the policy on a stochastic bandit where the mean rewards  are 
%\begin{equation}
%\gamma_1+c_1 - \gamma_i - \sum_{j\leq i} c_j \quad \mbox{for } i=1,2,\cdots, K. 
%\end{equation}
%\begin{remark}
%Note that the some of mean 	values $\gamma_1+c_1 - \gamma_i - \sum_{j\leq i} c_j$ need not be positive. Since most of the stochastic bandit algorithms assume that reward lie in the interval $[0,1]$, they may not be directly applicable to our setting. However, this can be over come by setting the distributions of arm $k$, $\nu_k$, to have mean $\gamma_1+c_1 - \gamma_i - \sum_{j\leq i} c_j + \sum_{k=2}^K c_k$. Note that we translated the mean of each arm by the same amount this does not change the regret value. For $k=2$, this recovers the SAT problem and Theorem \ref{thm:SATRegret} holds
%\end{remark}
%$\wedge$
%
%\begin{thm}[Upper Bound]
%	
%\end{thm}
%\begin{thm}
%	
%\end{thm}
%When we pull an arm  $i$, we observe a realization of a distribution with mean $\gamma_1 -\gamma_i$, but since the cost values are fixed, this is equivalent to observe realization of a distribution with mean $\gamma_1 +c_1-\gamma_i + \sum_{j\leq i}c_i$. Thus, the sensor selection problem with mean costs $\{\gamma_i + \sum_{j\leq i}c_i\}$ is equivalent to a stochastic multi-armed problem with mean rewards $\{\gamma_1+c_1 -\gamma_i - \sum_{j\leq i}c_i\}$. 

% \alglanguage{pseudocode}
% \begin{algorithm}[!h]
% 	\footnotesize
% 	\caption{StAT-K}
% 	\label{Algorithm:Stochastic Apple Tasting1}
% 	\begin{algorithmic}[1]
% 		\State \textbf{Input:}
% 		\State $c_k$ cost of sensor $k=2\cdots,K $
% 		\State \textbf{Initialization:}
% 		\State Play $a_K$ once, observe $X_{j,t}^i$, for all $K \geq j>i\geq 1$
% 		\State $N_{K}(1)\leftarrow 1 $, $Y_{j,1}^i\leftarrow X_{j,1}^i$ and $\hat{\mu}_{j,1}^i\leftarrow \frac{Y_{j,1}^i}{N_{K}(1)}$ for all $K \geq j > i\geq 1$
% 		\For {$t = 2,3\cdots$}
% 		\State {\bf Select the best action optimistically. How? We can use pairwise information to select the best action as follows}  
% 		\For {$j=2,3,\cdots, K$}
% 		\State $I\leftarrow 1$
% 		\If {$\hat{\mu}_{j,t-1}^I + \sqrt{\frac{6\log t}{\sum_{k\geq j} N_{k}(t-1)}} > \displaystyle \sum_{k=I+1}^{j}c_k$}  \quad //Checks if action $I$ is better than action $i$,
% 		\State $I\leftarrow j$
% 		\Else
% 		\State $I \leftarrow I$
% 		\EndIf
% 		\EndFor
% 			\State play action $a_I$ and observe $X_{I,t}^i$ for all $I>i\geq 1$,
% 			\State $N_{I}(t) \leftarrow N_{I}(t-1)+1$, $Y_{I,t}^i\leftarrow Y_{I,t-1}^j+X_{I,t}^i$ for all $ I>i \geq 1$
% 			\State $\hat{\mu}_{I,t}^i \leftarrow \frac{Y_{I,t}^i}{\sum_{k=I}^{K}N_k(t)}$ for all $ I>i \geq 1$
% 		\EndFor 
% 		
% 		\Statex
% 	\end{algorithmic}
% 	% \vspace{-0.4cm}%
% \end{algorithm}
% 
%
%\section{Algorithm}
%Let $X_{j,t}^i=\boldsymbol{1}_{\{\hat{y}^{i}_t \neq \hat{y}^j_t\}}, j>i$ denote whether the predictions of sensors $i$ and $j$ agree or not in round $t$. Note that $X_{j,t}^i$ is observed whenever learner selects action $a_k, k>j$. Let $\hat{\mu}_{j,t}^i$ denote the empirical mean of the samples $\{X_{j,t}^t\}$ observed till time $s$, given by
%\begin{equation}
%\hat{\mu}_{j,s}^i= \frac{\sum_{t=1}^s X_{j,t}^i\boldsymbol{1}_{\{I_t\geq j\}}}{\sum_{k=j}^{K}N_k(s)}.
%\end{equation}
%
%
%{\bf Remarks: We discussed the strategy in Algorithm-1 (StAT-K). Consider the case of perfect knowledge of $\gamma_i$'s and assume $j^{th}$ action is optimal. Then we know that $\gamma_i - \gamma_j> \sum_{k=i+1}^{j} c_k$ for all $i<j$ and $\gamma_j - \gamma_i < \sum_{k=j+1}^{i} c_k$ for $j<i$. However, after estimating the $\gamma_i$'s and selecting the best action optimistically as in StAT-K does not lead to similar argument, i.e.,  suppose in round $t$, $I^{th}$ action is selected by StAT-K, then the following need not hold.}
%
%\[{\hat{\mu}_{I,t-1}^j + \sqrt{\frac{6\log t}{\sum_{k\geq I} N_{k}(t-1)}} > \displaystyle \sum_{k=j}^{I}c_k} \quad for \;\;  I>j\]
%and
%
%\[{\hat{\mu}_{j,t-1}^I + \sqrt{\frac{6\log t}{\sum_{k\geq j} N_{k}(t-1)}} < \displaystyle \sum_{k=I+1}^{j}c_k} \quad for \;\; j> I\]
%



\section{Algorithms}
\label{sec:Algo}
%!TEX root =  main.tex
\newcommand{\set}{\leftarrow}
\newcommand{\hgamma}{\hat{\gamma}}
The reduction of the previous section suggests that one can play in an SAP instance by utilizing 
an algorithm developed for stochastic bandits with side-observation.
In this paper we make use of Algorithm~1 of \citet{WGySz:NIPS15}. \todoc{Note the bug in the other paper.}
While this algorithm was proposed for stochastic bandits with Gaussian side observations, 
as noted in the above paper, the algorithm is also suitable for problems where the payoff distributions are subgaussian.
As Bernoulli random variables are $\sigma^2=1/4$-subgaussian (after centering),
the algorithm is also applicable in our case.

\begin{wrapfigure}{L}{0.55\textwidth}
\vspace{-0.4cm}
\begin{minipage}{0.54\textwidth}
\begin{algorithm}[H]
\caption{} %ALG1}
\label{alg:asym}
\begin{algorithmic}[1]
\STATE Inputs: $\alpha>0$ and $\beta: \N \to [0,\infty)$.
\STATE Play action $K$ and observe the sensor outputs $Y^1,\dots,Y^K$.
\STATE Set $\hgamma \set (0,\one{Y^1\ne Y^2},\dots,\one{Y^1\ne Y^K})$.
\STATE Initialize the exploration count: $n_e \set 0$.
\STATE Initialize the allocation counts: $N_i = \one{i=K}$, $i\in [K]$.
\FOR{$t=2,3,...$}
	\IF{$\frac{N}{4\alpha \log t}\in C(\hgamma)$} \label{alg:check}
		\STATE Set $I \set \argmin_{k\in [K]} c_k(\hgamma)$. \label{alg:greedy}
%		\STATE Set $n_e(t+1)=n_e(t)$.
	\ELSE
		\IF{$\min_{i\in \iset{K}}m_i(N)<\beta(n_e)/K$} \label{alg:starve}
			\STATE Set $I = \argmin_{i\in \iset{K}}m_i(N)$. \label{alg:forced}
		\ELSE
			\STATE Set $I$ to some $i$ for which \label{alg:plan} \\
			$\qquad$ $N_i(t)< u_i^*(\hgamma)4\alpha\log t$.
		\ENDIF
		\STATE Increment exploration count: $n_e \set n_e+1$.
	\ENDIF
	\STATE Play $I$ and observe the sensor outputs $Y^1,\dots,Y^{I}$.
	\STATE For $i\in [I_t]$, set\\
	$\qquad$ $\hgamma_i \set (1-1/t) \hgamma_i + 1/t \,\one{Y^1\ne Y^i}$.
\ENDFOR
\end{algorithmic}
\end{algorithm}
\end{minipage}
\vspace{-0.3cm}
\end{wrapfigure}

For the convenience of the reader, we give the algorithm resulting from applying the reduction to Algorithm~1 
of \citet{WGySz:NIPS15} in an explicit form.
For specifying the algorithm we need some extra notation.
Given a SAP instance $\theta = (P,c)$, we let $\gamma_k = \Prob{Y\ne Y^k}$ where $(Y,Y^1,\dots,Y^K)\sim P$ and $k\in [K]$.
Since $c$ is assumed to be known, we will denote the dependence of the various quantities below
on $\gamma = (\gamma_1,\dots,\gamma_K)$ only.
Denote the expected cost of choosing action $k$ by $c_k(\gamma) = \gamma_k + \sum_{i=1}^k c_i$ 
(note that the notation is slightly changed compared to what we used before). \todoc{Perhaps unify the notation?}
Let $c^*(\gamma) = \min_{k\in [K]} c_k(\gamma)$ be the optimal expected cost and let 
$d_k(\gamma) = c_k(\gamma) - c^*(\gamma)$ be the expected excess cost of using action $k$,
$d^*(\gamma) = \min\{ d_k(\gamma)\,:\, k\in [K], d_k(\gamma)>0 \}$ be the smallest positive expected excess cost,
$d^*_k(\gamma) = \max(d^*(\gamma),d_k(\gamma))$.
Note that $d_k^*(\gamma) = d_k(\gamma)>0$ for any suboptimal
action $k$, while $d_k^*(\gamma) = d^*(\gamma)$ for the optimal action.

Define $C(\gamma) = \{ u\in [0,\infty)^K\,:\, u_1 + \dots + u_j \ge \frac{2\sigma^2}{(d_j^*(\gamma))^2}, j\in [K] \}$
as the set of ``sufficiently informative'' (fractional)  allocation counts 
(these ``counts'' guarantee that the observations are sufficient to distinguish all suboptimal actions from the optimal action; note that $\sigma^2=1/4$).
Further, let $n^*(\gamma)$
be the allocation count that minimizes the total expected excess cost over the set of sufficiently informative allocation counts:
In particular,  we let $u^*(\gamma) = \argmin_{u\in C(\gamma)} \ip{ u, d(\gamma) }$ 
with the understanding that for any optimal action $k$, $u_k^*(\gamma) = \min \{ u_k \,: u\in C(\gamma) \}$ (here, $\ip{x,y} = \sum_i x_i y_i$ is the standard inner product of vectors $x,y$).
For an allocation count $u\in [0,\infty)^K$ let $m(u)\in \N^K$ be the number observations made at each sensor:
$m_j(u) = \sum_{i=1}^j u_i$.

The idea of the algorithm shown as \cref{alg:asym} is as follows:
The algorithm keeps track of an estimate $\hgamma$ of $\gamma$, which is initialized by pulling arm $K$ as this arm
gives information about all the other arms.
In each round, the algorithm first checks whether given the current estimate $\hgamma$ and the current confidence level (where the confidence level is gradually increased over time), the current allocation count $N\in \N^K$
is sufficiently informative (cf. line \ref{alg:check}). If this holds, the action that is optimal under $\hgamma$ is chosen 
(cf. line \ref{alg:greedy}). If the check fails, we need to explore.
The idea of the exploration is that it tries to ensure that the ``optimal plan'' -- assuming $\hgamma$ is the ``correct'' parameter -- is followed (line \ref{alg:plan}). However, this is only reasonable, if all components of $\gamma$ are relatively well-estimated.
Thus, first the algorithm checks whether any of the components of $\gamma$ has a chance of being
extremely poorly estimated (line \ref{alg:starve}). Note that the requirement here is that a significant, but still altogether diminishing fraction of the \emph{exploration rounds} is spent on estimating each components: In the long run, the fraction of exploration rounds amongst all rounds itself is diminishing; hence the forced exploration of line \ref{alg:forced} overall has a small impact on the regret, while it allows to stabilize the algorithm.

\newcommand{\gap}{d}
\newcommand{\norm}[1]{\|#1\|}
For $\theta = (P,c)\in \TWD$, let $\gamma(\theta)$ be the error probabilities for the various sensors.
The following result follows from Theorem~6 of \cite{WGySz:NIPS15}:
\begin{thm}
Let $\epsilon>0$, $\alpha>2$ arbitrary and choose any non-decreasing $\beta(n)$ that satisfies $0\le \beta(n)\le n/2$ and $\beta(m+n)\le \beta(m)+\beta(n)$ for $m,n\in \N$.
Then, 
for any
$\theta = (P,c)\in \TWD$, letting $\gamma = \gamma(\theta)$
the expected regret of \cref{alg:asym} after $T$ steps satisfies 
\todoc{Actually, needs to be checked.. I also replaced $\gap_{\max}(\theta)$ with $1$.}
\begin{align*}
 R_T(\theta,c)
 & \le \big( 2K+2+4K/(\alpha-2) \big) + 4K\sum_{s=0}^T \exp \Big( -\frac{8\beta(s)\epsilon^2}{2K} \Big) \\
& + 2 \beta\Big( 4\alpha\log T\sum_{i\in \iset{K}} u_i^*(\gamma,\epsilon)+K \Big)
   + 4\alpha\log T \sum_{i\in \iset{K}} u_i^*(\gamma,\epsilon)\gap_i(\gamma) \,,
\end{align*}
where $u_i^*(\gamma,\epsilon) = \sup\{u_i^*(\gamma')\,:\, \norm{\gamma'-\gamma}_{\infty}\le \epsilon \}$.
\label{thm:alg1-ub1}
\end{thm}  
Further specifying $\beta(n)$ and using the continuity of $u^*(\cdot)$ at $\theta$, it immediately follows that Algorithm~\ref{alg:asym} achieves asymptotically optimal performance: \todoc{I just copy\&pasted this.
We don't actually have a lower bound..}
\begin{corollary}
\label{cor:alg1-asym-opt}
 Suppose the conditions of Theorem~\ref{thm:alg1-ub1} hold. Assume, furthermore, that $\beta(n)$ satisfies $\beta(n) = o(n)$ and $\sum_{s=0}^\infty \exp \left( -\frac{\beta(s)\epsilon^2}{2K\sigma^2} \right)<\infty$ for any $\epsilon>0$, then for any $\theta$ such that $u^*(\theta)$ is unique, 
\[
\limsup_{T\rightarrow \infty} R_T(\theta,c)/\log T \le 4\alpha \inf_{u\in C_\theta} \ip{u,\gap(\gamma(\theta))}\,.
\]
\end{corollary}
Note that any $\beta(n) = an^b$ with $a\in (0,\frac{1}{2}]$, $b\in (0,1)$ satisfies the requirements in Theorem~\ref{thm:alg1-ub1} and Corollary~\ref{cor:alg1-asym-opt}.
%Also note that the algorithms presented in \cite{CaKvLe12,BuErSh14} do not achieve this asymptotic bound.





\clearpage

\section{Experiments}
\label{sec:Experiments}
In this section we apply bandit algorithms on SA-problem and evaluate its performance on synthetic and real datasets. For synthetic example, we consider data transmission over a binary symmetric channel, and for real world examples, we use diabetes (PIMA indiana) and heart disease (Clevland) from UCI dataset. The experiments are set up as follows:

{\bf Synthetic:} We consider data transmission over two binary symmetric channels (BSCs). Channel $i=1,2$ flips input bit with probability $p_i$ where $p_1> p_2$. Transmission over channel $2$ costs $ c \in (0,1] $ units per bit. Input bits are generated with uniform probability. We set $p_1=.2$ and $p_2=.1$. For this example, $p_{12}=.26$ and $\delta_{12}=.8$. Thus, it satisfies weak dominance assumption for all $ 0<c <0 .1$ and $0 .26 \leq c < 1$.

We obtain a sensor acquisition setup from the datasets as follows: in each dataset, attributes related to physical observations are cheap and that obtained from medical tests are costly. Two svm classifiers (linear, $C=.01$) are trained for each dataset, one using only cheap attributes, and the other using all attributes. These classifiers then form sensors of a two stage SAP where classifier trained with cheap features is the first stage. Cost $c$ is defined as total cost of all attributes and total cost of cheap attributes multiplied by a scaling factor $\lambda$ (trade-off parmaeter for accuracy and costs). Specific details for each dataset is given below and the various error probabilities are listed in Table (\ref{tab:ErrorTable}).    

{\bf PIMA indians diabetes} dataset consists of $768$ instances and has $8$ attributes. The labels identify if the instances are diabetic or not. $6$ of the attributes (age, sex, triceps, etc.) obtained from physical observations, are cheap, and $2$ attributes (glucose and insulin) require expensive tests. First sensor of SAP is trained with $6$ cheap attributes and costs \$$6$. Total cost for all attributes is \$$30$. We set $c= 24\lambda$.

{\bf Heart disease} dataset consists of $297$ instance (without missing values) and has $13$ attributes. $5$ class labels $(0,1,2,3,4)$ are mapped to binary values by taking value $0$ as `absence' of disease and values $(1,2,3,4)$ as `presence' of disease. First senor of SAP is trained with $7$ attributes that costs  \$$1$ each. Total cost of all attributes is \$$568$. We set $c= 561\lambda$.


\centering
\begin{tabular}[c]{c|c|c|c|c } 
	\label{tab:ErrorTable}
	%			\caption{Possible binary tuples}
	
	dataset & $\gamma_1$ & $\gamma_2$ & $p_{12}$ & $\delta_{12}$\\ \hline 
	synthetic & .2 & .1 & .26 & .08\\  \hline
	diabetic & $0.3836 $ & $0.2808$ & $0.1986$ & 0.0778\\  \hline
	heart & $0.3051$ & $0.1695$ & $0.2373$ & 0.0732\\  \hline
\end{tabular}



\begin{figure}[!h]
	\centering
	\includegraphics[scale=.4]{../Simulations/RegVsCost.jpg}
\end{figure}

\begin{figure}[!h]
	\centering
	\includegraphics[scale=.4]{../Simulations/Synthetic/BSC.jpg}
\end{figure}


\section{Conclusions}
\label{sec:Conclu}


We need to conclude soon.


\newpage
123
\newpage


%\bibliographystyle{biblio}
\bibliographystyle{IEEEtran}
\bibliography{bandits1}


\newpage
\section{Appendix}
%!TEX root =  main.tex
\clearpage
\onecolumn
\appendix

\section{Stochastic Partial Monitoring Problem}
\label{sec:spm}
In \cref{sec:Setup} it was mentioned that our problem is a special case stochastic partial monitoring (SPM).
The purpose of this short section is to formally define SPM problems.
In an SPM a learner interacts with a stochastic environment in a sequential manner.
In round $t=1,2,\dots$ the learner chooses an action $A_t$ from an action set $\A$, and receives a feedback $Y_t\in \Y$
from a distribution $p$ which depends on the action chosen and also on the environment instance identified
with a ``parameter'' $\theta\in\Theta$:
$Y_t \sim p(\cdot;A_t,\theta)$. 
%The learner chooses $A_t$ based on the past feedbacks $Y_1,\dots,Y_{t-1}$. 
The learner also incurs a reward $R_t$, which is a function of the action chosen and the unknown parameter $\theta$:
$R_t = r(A_t,\theta)$. 
The reward may or may not be part of the feedback for round $t$.
The learner's goal is to maximize its total expected reward.
The family of distributions $(p(\cdot;a,\theta))_{a,\theta}$ and the family of rewards $(r(a,\theta))_{a,\theta}$
and the set of possible parameters $\Theta$ are known to the learner, who uses this knowledge to judiciously choose
its next action to reduce its uncertainty about $\theta$ so that it is able to eventually converge on choosing only an 
optimal action $a^*(\theta)$, achieving the best possible reward per round, $r^*(\theta) = \max_{a\in \A} r(a,\theta)$.  Bandit problems are a special case of SPMs where $\Y$ is the set of real numbers, $r(a,\theta)$ is the mean of distribution $p(\cdot;a,\theta)$.

%%%%%%%%%%%%%%%%%%%%%%%%%%%%%%%%%%%%%%%%%%%%%%%%
\section{Proofs for \cref{sec:Learnability}}
%%%%%%%%%%%%%%%%%%%%%%%%%%%%%%%%%%%%%%%%%%%%%%%%

\label{sec:apxlearnability}
Here we provide the missing proof for \cref{sec:Learnability}.
For the convenience of the reader the statements of the various propositions are repeated.
We start with proofs related to Strong Dominance.
%%%%%%%%%%%%%%%%%%%%%%%%%%%%%%%%%%%%%%%%%%%%%%%%
\subsection{Proofs Related to Strong Dominance}
%\section{Proof of \cref{prop:learnablemap}}
%%%%%%%%%%%%%%%%%%%%%%%%%%%%%%%%%%%%%%%%%%%%%%%%
\propLearnablemap*
%%%%%%%%%%%%%%%%%%%%%%%%%%%%%%%%%%%%%%%%%%%%%%%%
\begin{proof}
	$\Rightarrow$: Let $\Alg$ be an algorithm that achieves sublinear regret
	and pick  an instance  $\theta \in\Theta$. Let $P = P_S \otimes P_{Y|S}$.
	The regret $\Regret_n(\Alg,\theta)$ of $\Alg$ on instance $\theta$ can be written in the form
	\begin{align*}
	\Regret_n(\Alg,\theta) = \sum_{k\in [K]} \EEi{P_S}{N_k(n)} \Delta_k(\theta)\,,
	\end{align*}
	where $N_k(n)$ is the number of times action $k$ is chosen by $\Alg$ during the $n$ rounds while
	$\Alg$ interacts with $\theta$, $\Delta_k(\theta) = c(k,\theta) - c^*(\theta)$ is the immediate regret
	and $\EEi{P_S}{\cdot}$ denotes the expectation under the distribution induced by $P_S$.
	In particular, $N_k(n)$ hides dependence on the iid sequence $Y_1,\dots,Y_n \sim P_S$ 
	that we are taking the expectation over here. 
	Since the regret is sublinear, for any $k$ suboptimal action, $\EEi{P_S}{N_k(n)} = o(n)$. 
	Define $a(P_S) = \min \{ k\in [K]\,;\, \EEi{P_S}{N_k(n)} = \Omega(n) \,\}$. Then, $a$ is well-defined as the distribution of $N_k(n)$ for any $k$ depends only on $P_S$ (and $c$). Furthermore, $a(P_S)$ selects an optimal action.
	
	$\Leftarrow$: Let $a$ be the map in the statement and let $f:\N_+\to\N_+$ be such that $1\le f(n)\le n$ for any  $n\in \N$,
	$f(n)/\log(n) \to \infty$ as $n\to \infty$ and $f(n)/n \to 0$ as $n\to \infty$ (say, $f(n) = \lceil \sqrt{n} \rceil$).
	Consider the algorithm that chooses $I_t = K$ for the first $f(n)$ steps, after which it estimates $\hat{P}_S$ by
	frequency counting and then uses $I_t = a(\hat{P}_S)$ in the remaining $n-f(n)$ trials. 
	Pick any $\theta \in \Theta$ so that $\theta = P_S \otimes P_{Y|S}$. 
	Note that by Hoeffding's inequality, 
	$\sup_{y\in \{0,1\}^K} |\hat{P}_S(y)  - P_S(y)| \le \sqrt{\frac{K\log(4n)}{2f(n)}}$ holds with probability $1-1/n$.
	Let $n_0$ be the first index such that for any $n\ge n_0$,
	$\sqrt{\frac{K\log(4n)}{2f(n)}}\le \Delta^*(\theta) \defeq \min_{k:\Delta_k(\theta)>0} \Delta_k(\theta)$.
	Such an index $n_0$ exists by our assumptions that $f$ grows faster than $n \mapsto \log(n)$.
	For $n\ge n_0$, the expected regret of $\Alg$ is at most $n \times 1/n + f(n) (1-1/n) \le 1+f(n) = o(n)$.
\end{proof}


Next, we present the proof of \cref{prop:gammadiff}. We will need this proposition in the proof of 
\cref{thm:tsdlearnable}.
%%%%%%%%%%%%%%%%%%%%%%%%%%%%%%%%%%%%%%%%%%%%%%%%
\propGammadiff*
%%%%%%%%%%%%%%%%%%%%%%%%%%%%%%%%%%%%%%%%%%%%%%%%
\begin{proof}
Recall that $\gamma_i = \Prob{Y^i \ne Y}$, $(Y,Y^1,\dots,Y^K)\sim \theta$. We have
\if0	
	We construct a map as required by~\cref{prop:learnablemap}.
	Take an instance $\theta \in \TWD$ and let $\theta = P_S \otimes P_{Y|S}$ be its decomposition
	as defined above.
	For identifying an optimal action in $\theta$, it clearly suffices
	to know the sign of $\gamma_i + C_i - (\gamma_j +C_j)$ for all pairs $i,j\in [K]^2$.
	Since $C_i - C_j$ is known, it remains to study $\gamma_i-\gamma_j$.
	Without loss of generality (WLOG) let $i<j$.
	Then, 
\fi	
	\begin{align*}
	\MoveEqLeft 
	%0  \le 
	\gamma_i  - \gamma_j = \Prob{Y^i\ne Y} - \Prob{Y^j\ne Y} \\
	& = \cancel{\Prob{Y^i\ne Y, Y^i=Y^j}} + \Prob{ Y^i\ne Y, Y^i\ne Y^j } 
	 - \left\{ 
	\cancel{\Prob{Y^j\ne Y, Y^i=Y^j}} + \Prob{ Y^j\ne Y, Y^i\ne Y^j }\right\}\\
	& = \Prob{ Y^i\ne Y, Y^i \ne Y^j } + \Prob{Y^i=Y,Y^i\ne Y^j}       
	 - \left\{ 
	\Prob{ Y^j \ne Y, Y^i\ne Y^j } + \Prob{ Y^i=Y,Y^i\ne Y^j}
	\right\}\\
	& \stackrel{\footnotesize (a)}{=} \Prob{ Y^j \ne Y^i } -2 \Prob{ Y^j\ne Y, Y^i = Y }\,,
	\end{align*}
	where in $(a)$ we used that $\Prob{ Y^j \ne Y, Y^i\ne Y^j } =  \Prob{ Y^j\ne Y,Y^i= Y}$ and also
	$\Prob{ Y^i=Y,Y^i\ne Y^j} = \Prob{ Y^j\ne Y,Y^i= Y}$
	which hold because $Y,Y^i,Y^j$ only take on two possible values.
\end{proof}



With this, we are ready to give the proof of \cref{thm:tsdlearnable}:
%%%%%%%%%%%%%%%%%%%%%%%%%%%%%%%%%%%%%%%%%%%%%%%%
\thmTSDLearnable*
%%%%%%%%%%%%%%%%%%%%%%%%%%%%%%%%%%%%%%%%%%%%%%%%
\begin{proof}[Proof of \cref{thm:tsdlearnable}]
	We construct a map as required by~\cref{prop:learnablemap}.
	Take an instance $\theta \in \TSD$ and let $\theta = P_S \otimes P_{Y|S}$ be its decomposition as before.
	Let $\gamma_i = \Prob{Y^i \ne Y}$, $(Y,Y^1,\dots,Y^K)\sim \theta$, $C_i = c_1+\dots+c_i$.
	For identifying an optimal action in $\theta$, it clearly suffices
	to know the sign of $\gamma_i + C_i - (\gamma_j +C_j) = \gamma_i-\gamma_j + (C_i-C_j)$ for all pairs $i,j\in [K]^2$.
	Without loss of generality (WLOG) let $i<j$. By \cref{prop:gammadiff},
	$\gamma_i - \gamma_j = \Prob{ Y^i \ne Y^j } -2 \Prob{ Y^j\ne Y, Y^i = Y }$.
	Now, since $\theta$ satisfies the strong dominance condition, $ \Prob{ Y^j\ne Y, Y^i = Y } = 0$.
	Thus, $\gamma_i - \gamma_j = \Prob{ Y^i \ne Y^j }$
	which is a function of $P_S$ only.
	Since $(C_i)_i$ are known, a map as required by~\cref{prop:learnablemap} exists.
\end{proof}

Let us now turn to proofs and statements related to Weak Dominance.
%%%%%%%%%%%%%%%%%%%%%%%%%%%%%%%%%%%%%%%%%%%%%%%%
\subsection{Proofs and Statements Related to Weak Dominance}
%%%%%%%%%%%%%%%%%%%%%%%%%%%%%%%%%%%%%%%%%%%%%%%%
We start with two statements that prepare for the definition of weak dominance.
%%%%%%%%%%%%%%%%%%%%%%%%%%%%%%%%%%%%%%%%%%%%%%%%
\propIlej*
%%%%%%%%%%%%%%%%%%%%%%%%%%%%%%%%%%%%%%%%%%%%%%%%
\begin{proof}
	\noindent $\Rightarrow$: From the premise, it follows that $C_j - C_i \ge \gamma_i - \gamma_j$.
	Thus, by~\eqref{eq:cond1}, $C_j - C_i \ge \Prob{Y^i\ne Y^j}$.
	\noindent $\Leftarrow$: We have $C_j - C_i \ge \Prob{Y^i \ne Y^j} \ge \gamma_i -\gamma_j$, where the last
	inequality is by \cref{cor:gammadiff}.
	\end{proof}
	
%%%%%%%%%%%%%%%%%%%%%%%%%%%%%%%%%%%%%%%%%%%%%%%%
\propJlei*
%%%%%%%%%%%%%%%%%%%%%%%%%%%%%%%%%%%%%%%%%%%%%%%%
	\begin{proof}
		\noindent $\Rightarrow$: The condition $\gamma_i + C_i \le \gamma_j + C_j$ implies that $\gamma_j -\gamma_i \ge C_i - C_j$.
		By \cref{cor:gammadiff} we get $\Prob{Y^i \ne Y^j} \ge C_i - C_j$.
		\noindent $\Leftarrow$: Let $C_i - C_j \le \Prob{Y^i \ne Y^j}$. Then, by \eqref{eq:cond2}, $C_i - C_j \le \gamma_j - \gamma_i$.
	\end{proof}

Recall the definition of $\awd$:
\defAwd*

This definition makes sense:
%%%%%%%%%%%%%%%%%%%%%%%%%%%%%%%%%%%%%%%%%%%%%%%%
\begin{prop}
	\label{prop:awdwelldef}
	The action-selector $\awd$ is sound over $\TWD$:
	For any $\theta \in \TWD$ with $\theta = P_S\otimes P_{Y|S}$, $\awd(P_S)$ is well-defined, i.e., the domain
	of $\awd$ includes all of $\TWD$.
\end{prop}
%%%%%%%%%%%%%%%%%%%%%%%%%%%%%%%%%%%%%%%%%%%%%%%%

\begin{proof}
	Let $\theta\in \TWD$, $i = a^*(\theta)$. 
	It suffices to show that $i$ satisfies both \eqref{eq:wd1} and \eqref{eq:wd2}.
	Obviously, \eqref{eq:wd2} holds by the definition of $\TWD$.
	Thus, the only question is whether \eqref{eq:wd1} also holds.
	We prove this by contradiction:
	In particular if \eqref{eq:wd1} does not hold then for some $j<i$, $C_i-C_j \ge \Prob{Y^i \ne Y^j}$. 
	Then, by \cref{cor:gammadiff}, $\Prob{ Y^i \ne Y^j} \ge \gamma_j - \gamma_i$, hence $\gamma_j + C_j \le \gamma_i + C_i$, which contradicts the definition of $i$, thus finishing the proof. 
	\end{proof}
We can now prove that $\awd$ is sound:
\begin{prop}
	\label{prop:awdsound}
	The map $\awd$ is sound over $\TWD$: In particular, for any
	$\theta\in \TWD$ with $\theta = P_S\otimes P_{Y|S}$, $\awd(P_S)= a^*(\theta)$.
\end{prop}
\begin{proof}
	Take any $\theta\in \TWD$ with $\theta = P_S\otimes P_{Y|S}$, $i = \awd(P_S)$, $j = a^*(\theta)$.
	If $i=j$, there is nothing to be proven. Hence, first assume that $j>i$. Then, by~\eqref{eq:wd2}, $C_j - C_i \ge \Prob{Y^i \ne Y^j}$.
	By \cref{cor:gammadiff}, $\Prob{Y^i \ne Y^j } \ge \gamma_i - \gamma_j$. Combining these two inequalities we get that
	$\gamma_i + C_i \le \gamma_j + C_j$, which contradicts with the definition of $j$.
	Now, assume that $j<i$. Then, by~\eqref{eq:wd}, $C_i - C_j \ge \Prob{Y^i \ne Y^j}$.
	However, by~\eqref{eq:wd1}, $C_i -C_j < \Prob{Y^i \ne Y^j}$, thus $j<i$ cannot hold either and we must have $i=j$.
	\end{proof}
A corollary of the previous result is that $\TWD$ is also learnable:
\begin{thm}\label{cor:twdlearnable}
	The set $\TWD$ is learnable.
\end{thm}
%\section{Proof of Corollary \ref{cor:twdlearnable}}
\begin{proof}
	By \cref{prop:awdwelldef}, $\awd$ is well-defined over $\TWD$, while by \cref{prop:awdsound}, $\awd$ is sound over $\TWD$.
	%By \cref{prop:learnablemap}, $\TWD$ is learnable, as witnessed by $\awd$. \todoc{We should add definitions for these concepts..
	%namely, $\awd$ well-defined over $\TWD$, $\awd$ sound over $\TWD$, etc.}\todom[]{Csaba, you did this already!!}
\end{proof}

Now, we will prove that $\awd$ is the only sound action selector over $\TWD$.
\begin{prop}
	\label{prop:awdcorrectimplieswd}
	Let $\theta \in \TSA$ and $\theta = P_S\otimes P_{Y|S}$ be such that $\awd$ is defined for $P_S$
	and $\awd(P_S) = a^*(\theta)$. Then $\theta \in \TWD$.
\end{prop}
The proof follows from the definitions. An immediate corollary of the previous proposition is as follows:
\begin{cor}\label{cor:awdoutsideincorrect}
	Let $\theta \in \TSA$ and $\theta = P_S \otimes P_{Y|S}$. 
	Assume that $\awd$ is defined for $P_S$ and $\theta \not\in \TWD$. Then $\awd(P_S) \ne a^*(\theta)$.
\end{cor}

The next result states that $\awd$ is essentially the only sound action selector map defined for
all instances derived from instances of $\TWD$:

%\section{Proof of Proposition \ref{prop:awdunique}}
\begin{thm}\label{prop:awdunique}
	Take any action selector map $a: M_1( \{0,1\}^K ) \to [K]$ which is sound over $\TWD$.
	Then, for any $P_S$ such that $\theta = P_S\otimes P_{Y|S}\in \TWD$ with some $P_{Y|S}$,
	$a(P_S) = \awd(P_S)$.
\end{thm}
\begin{proof}
	Pick any $\theta = P_S\otimes P_{Y|S}\in \TWD$. If $A^*(\theta)$ is a singleton, then clearly $a(P_S) = \awd(P_S)$ since both are sound over $\TWD$.
	Hence, assume that $A^*(\theta)$ is not a singleton.
	Let $i = a^*(\theta) = \min A^*(\theta)$ and let $j = \min A^*(\theta) \setminus \{ i \}$.
	We argue that $P_{Y|S}$ can be changed so that on the new instance $i$ is still an optimal action, while
	$j$ is not an optimal action, while the new instance $\theta' = P_S \otimes P_{Y|S}'$ is in $\TWD$.
	
	The modification is as follows:
	Consider any $y^{-j} \defeq (y^1,\dots,y^{j-1},y^{j+1},\dots,y^K)\in \{0,1\}^{K-1}$.
	For $y,y^j\in \{0,1\}$, define 
	$q(y|y^j) = P_{Y|S}(y|y^1, \dots, y^{j-1}, y^j, y^{j+1},\dots, y^K)$
	and similarly let
	$q'(y|y^j) = P_{Y|S}'(y|y^1, \dots, y^{j-1}, y^j, y^{j+1},\dots, y^K)$
	Then, we let $q'(0|0) = 0$ and $q'(0|1) = q(0|0) + q(0|1)$,
	while we let  $q'(1|1) = 0$ and $q'(1|0) = q(1|1) + q(1|0)$.
	This makes $P_{Y|S}'$ well-defined ($P_{Y|S}'(\cdot|y^1,\dots,y^K)$ is a distribution for any $y^1,\dots,y^K$).
	Further, we claim that the transformation has the property that 
	it leaves $\gamma_p$ unchanged for $p\ne j$, while $\gamma_j$ is guaranteed to decrease.
	To see why $\gamma_p$ is left unchanged for $p\ne j$ note that
	$\gamma_p = \sum_{y^p}  P_{Y^p}(y^p) P_{Y|Y^p}(1-y^p|y^p)$.
	Clearly, $P_{Y^p}$ is left unchanged.
	Introducing $y^{-k}$ to denote a tuple where the $k$th component is left out,
	$P_{Y|Y^p}(1-y^p|y^p) = \sum_{y^{-p,-j}} P_{Y|Y^1,\dots,Y^K}( 1-y^p | y^1,\dots, y^{j-1}, 0, y^{j+1}, \dots, y^K )
	+P_{Y|Y^1,\dots,Y^K}( 1-y^p | y^1,\dots, y^{j-1}, 1, y^{j+1}, \dots, y^K )$
	and by definition,
	\begin{align*}
	& P_{Y|Y^1,\dots,Y^K}( 1-y^p | y^1,\dots, y^{j-1}, 0, y^{j+1}, \dots, y^K )\\
	&\quad +P_{Y|Y^1,\dots,Y^K}( 1-y^p | y^1,\dots, y^{j-1}, 1, y^{j+1}, \dots, y^K )\\
	&
	=
	P_{Y|Y^1,\dots,Y^K}'( 1-y^p | y^1,\dots, y^{j-1}, 0, y^{j+1}, \dots, y^K )\\
	&\quad+P_{Y|Y^1,\dots,Y^K}'( 1-y^p | y^1,\dots, y^{j-1}, 1, y^{j+1}, \dots, y^K )\,,
	\end{align*}
	where the equality holds because ``$q'(y|0)+q'(y|1) = q(y|0) + q(y|1)$''.
	Thus, $P_{Y|Y^p}(1-y^p|y^p) = P_{Y|Y^p}'(1-y^p|y^p)$ as claimed.
	That $\gamma_j$ is non-increasing follows with an analogue calculation.
	In fact, this shows that $\gamma_j$ is strictly decreased
	if for any $(y^1,\dots,y^{j-1},y^{j+1},\dots,y^K)\in \{0,1\}^{K-1}$, either $q(0|0)$ or $q(1|1)$ was positive.
	If these are never positive, this means that $\gamma_j=1$. 
	But then $j$ cannot be optimal since $c_j>0$.
	Since $j$ was optimal, $\gamma_j$ is guaranteed to decrease.
	
	Finally, it is clear that the new instance is still in $\TWD$ since  $a^*(\theta)$ is left unchanged.
\end{proof}
The next result shows that
the set $\TWD$ is essentially a maximal learnable set in $\mathrm{dom}(\awd)$:

\begin{thm}
	\label{thm:MaxLearnability}
	Let $a: M_1(\{0,1\}^K) \to [K]$ be an action selector map
	such that $a$ is sound over the instances of $\TWD$.
	Then there is no instance $\theta = P_S\otimes P_{Y|S} \in \TSA\setminus \TWD$ such that 
	$P_S\in \mathrm{dom}(\awd)$, the optimal action of $\theta$ is unique
	%\todoc{It would be nice to remove this uniqueness assumption, but I don't see how this could be made to work.}
	and $a(P_S) = a^*(\theta)$.
\end{thm}
%\section{Proof of Theorem \ref{thm:MaxLearnability}}
\begin{proof}
	Let $a$ as in the theorem statement. By~\cref{prop:awdunique}, $\awd$ is the unique sound action-selector map over $\TWD$.
	Thus, for any $\theta = P_S\otimes P_{Y|S}\in \TWD$, $\awd(P_S) = a(P_S)$.
	Hence, the result follows from \cref{cor:awdoutsideincorrect}.
\end{proof}

Note that $\mathrm{dom}(\awd)\setminus \{ P_S \,:\, \exists P_{Y|S} \textrm{ s.t. } P_S \otimes P_{Y|S} \in \TWD \} \ne \emptyset$, i.e., the theorem statement is non-vacuous.
In particular, for $K=2$, consider $(Y,Y^1,Y^2)$ such that $Y$ and $Y^1$ are independent and $Y^2 = 1-Y^1$, we can see that the resulting instance gives rise to $P_S$ which is in the domain of $\awd$ for any $c\in \R_+^K$ (because here $\gamma_1 = \gamma_2 = 1/2$, thus $\gamma_1 - \gamma_2 = 0$ while $\Prob{Y^1\ne Y^2}=1$).
While $\TWD$ is learnable, it is not uniformly learnable, i.e., the minimax regret $\Regret_n^*(\TWD) = \inf_{\Alg} \sup_{\theta\in \TWD} \Regret_n(\Alg,\theta)$ over $\TWD$ grows linearly:

We close by showing that while $\TWD$ is learnable and maximal, the price is that it is not uniformly learnable:
%\section{Proof of Theorem \ref{thm:nonunif}}
\begin{thm}
	\label{thm:nonunif}
	$\TWD$ is not uniformly learnable:
	$\Regret_n^*(\TWD) = \Omega(n)$.
\end{thm}

\begin{proof}
We first consider the case when $K=2$ and arbitrarily choose $C_2 - C_1 = 1/4$. 
%\todoc{The theorem statement should be refined or this text..}
We will consider two instances, $\theta,\theta'\in \TWD$ such that for instance $\theta$, 
action $k=1$ is optimal with an action gap of $c(2,\theta) - c(1,\theta) = 1/4$ between the cost of the second and the first
action,  while for instance $\theta'$, $k=2$ is the optimal action and the action gap is $c(1,\theta) - c(2,\theta) = \epsilon$
where $0<\epsilon<3/8$.
Further, the entries in $P_S(\theta)$ and $P_S(\theta')$ differ by at most $\epsilon$. 
From this, a standard reasoning gives that no algorithm can achieve sublinear minimax regret over $\TWD$ because any
algorithm is only able to identify $P_S$. 
%\todoc{Add notation of $P_S(\theta)$ early on. Probably a good idea to add $P_S(\Theta)$ as a notation too for the ``projection'' of $\Theta$ to $P_S$. Also, we should probably remove $c$ from the instance definition;
%	in every case we are reasoning for a fixed $c$, hence it is superfluous to keep $c$ in the instance definition.}\todom{In the beginning, it is now mentioned that c is fixed and known. It is removed from definition of problem instance now. Will introduce the projection notation in the next round of edits.}

The constructions of $\theta$ and $\theta'$ are shown in \cref{tab:nonunif}:
The entry in a cell gives the probability of the event as specified by the column and row labels.
For example, in instance $\theta$, $3/8$ is the probability of $Y=Y^1=Y^2$, 
while the probability of $Y^1=Y\ne Y^2$ is $1/8$. Note that the cells with zero actually 
correspond to impossible events, i.e., these cannot be assigned a positive probability.
The rationale of a redundant (and hence sparse) table is so that probabilities of certain events of interest, such as $Y^1\ne Y^2$ are easier to determine based on the table. The reader should also verify that the positive probabilities correspond to events that are possible.
\bgroup
\def\arraystretch{1.5}
\begin{table}[]
	\centering
	%\hspace*{-0.1in}
	\begin{tabular}{|c|c|c|c|}
		\hline
		\multicolumn{2}{|c|}{Instance $\theta$}  & $Y^1=Y^2$     & $Y^1\ne Y^2$  \\ \hline
		\multirow{2}{*}{$Y^1= Y$}   & $Y^2= Y$   & $\frac{3}{8}$ & $0$           \\ \cline{2-4} 
		& $Y^2\ne Y$ & $0$ & $\frac{1}{8}$ \\ \hline
		\multirow{2}{*}{$Y^1\ne Y$} & $Y^2= Y$   & $0$ & $\frac{1}{8}$           \\ \cline{2-4} 
		& $Y^2\ne Y$ & $\frac{3}{8}$ & $0$ \\ \hline
	\end{tabular}
	\hspace*{0.5in}
	\begin{tabular}{|c|c|c|c|}
		\hline
		\multicolumn{2}{|c|}{Instance $\theta'$}  & $Y^1=Y^2$     & $Y^1\ne Y^2$  \\ \hline
		\multirow{2}{*}{$Y^1= Y$}   & $Y^2= Y$   & $\frac{3}{8}-\epsilon$ & $0$           \\ \cline{2-4} 
		& $Y^2\ne Y$ & $0$ & $0$ \\ \hline
		\multirow{2}{*}{$Y^1\ne Y$} & $Y^2= Y$   & $0$ & $\frac{2}{8}+\epsilon$           \\ \cline{2-4} 
		& $Y^2\ne Y$ & $\frac{3}{8}$ & $0$ \\ \hline
	\end{tabular}
	\vspace*{0.1in}
	\caption{The construction of two problem instances for the proof of \cref{thm:nonunif}.}
	\label{tab:nonunif}
\end{table}
\egroup

We need to verify the following:
{\em (i)} $\theta,\theta'\in \TWD$;
{\em (ii)} the optimality of the respective actions in the respective instances;
{\em (iii)} the claim concerning the size of the action gaps;
{\em (iv)} that $P_S(\theta)$ and $P_S(\theta')$ are close.
Details of the calculations to support {\em (i)}--{\em (iii)} can be found in \cref{tab:nonunif2}.
The row marked by $(*)$ supports that the instances are proper USS instances.
In the row marked by $(**)$, there is no requirement for $\theta'$ because 
in $\theta'$ action two is optimal, and hence there is no action with larger index 
than the optimal action, hence $\theta'\in \TWD$ automatically holds.

\def\arraystretch{1.5}
\begin{table}[]
	\centering
	%\hspace*{-0.1in}
	\begin{tabular}{|c|c|c|}
		\hline
		& $\theta$                & $\theta'$ \\ \hline
		$\gamma_1 = \Prob{Y^1\ne Y}$ & $\frac{1}{4}$           & $\frac{5}{8}+\epsilon$ \\ \hline
		$\gamma_2 = \Prob{Y^2\ne Y}$ & $\frac{1}{4}$           & $\frac{3}{8}$ \\ \hline
		$\gamma_2 \le \gamma_1 \mbox{}^{(*)}$        & \checkmark           & \checkmark \\ \hline
		$c(1,\cdot)$                                 & $\frac{1}{4}$           & $\frac{5}{8}+\epsilon$ \\ \hline
		$c(2,\cdot)$                                 & $\frac{2}{4}$           & $\frac{5}{8}$ \\ \hline
		$a^*(\cdot)$                                 & $k=1$                   & $k=2$ \\ \hline
		$\Prob{Y^1\ne Y^2}$                   & $\frac{1}{4}$         & $\frac{1}{4}+\epsilon$ \\ \hline
		$\theta \in \TWD  \mbox{}^{(**)}$                        & $\frac{1}{4}\ge \frac14$ \checkmark & \checkmark \\ \hline
		$|c(1,\cdot)-\c(2,\cdot)|$              & $\frac{1}{4}$         & $\epsilon$ \\ \hline
	\end{tabular}
	\vspace*{0.1in}
	\caption{Calculations for the proof of \cref{thm:nonunif}.}
	\label{tab:nonunif2}
\end{table}

To verify the closeness of $P_S(\theta)$ and $P_S(\theta')$ we actually 
would need to first specify $P_S$ (the tables do not fully specify these).
However, it is clear the only restriction we put on $P_S$ is the value of $\Prob{Y^1\ne Y^2}$ (and
that of $\Prob{Y^1=Y^2}$) and these values are within an $\epsilon$ distance of each other.
Hence, $P_S$ can also be specified to satisfy this. In particular, one possibility for $P$ and $P_S$ are given in \cref{tab:nonunif3}.
\bgroup
%\egroup
%
%\bgroup
%\def\arraystretch{1.5}
\begin{table}[]
	\centering
	%\hspace*{-0.1in}
	\begin{tabular}{|c|c|c||c|c|}
		\hline
		$Y^1$ & $Y^2$ & $Y$ & $\theta$ & $\theta'$ \\ \hline\hline
		$0$ & $0$ & $0$ & $\frac38$ & $\frac38-\epsilon$ \\ \hline
		$0$ & $0$ & $1$ & $\frac38$ & $\frac38-\epsilon$ \\ \hline
		$0$ & $1$ & $0$ & $0         $ & $0                       $ \\ \hline
		$0$ & $1$ & $1$ & $0         $ & $0                       $ \\ \hline
		$1$ & $0$ & $0$ & $\frac18$ & $\frac28+\epsilon$ \\ \hline
		$1$ & $0$ & $1$ & $\frac18$ & $0                       $ \\ \hline
		$1$ & $1$ & $0$ & $0         $ & $0                       $ \\ \hline
		$1$ & $1$ & $1$ & $0         $ & $0                       $ \\ \hline
	\end{tabular}
	\mbox{}
	\hspace*{1.5in}
%	\hspace*{0.5in}
	\mbox{}
	\begin{tabular}{|c|c||c|c|}
		\hline
		$Y^1$ & $Y^2$ &$\theta$ & $\theta'$ \\ \hline\hline
		$0$ & $0$ & $\frac68$ & $\frac68-\epsilon$ \\ \hline
		$0$ & $1$ & $0         $ & $0                       $ \\ \hline
		$1$ & $0$ & $\frac28$ & $\frac28+\epsilon$ \\ \hline
		$1$ & $1$ & $0         $ & $0                       $ \\ \hline
	\end{tabular}
	
	\vspace*{0.1in}
	\caption{Probability distributions for instances $\theta$ and $\theta'$. On the left are shown the joint
		probability distributions, while on the right are shown their marginals
		for the sensors.}
	\label{tab:nonunif3}
\end{table}
\egroup


\end{proof}

\section{Proofs for \cref{sec:Equiv}}
\propEquivalence*
\begin{proof}First note that the mapping of the policies is such that number of pull of arm $k$ after $n$ rounds by policy $\pi$ on problem instance $f(\theta)$ is the same as the number of pulls of arm $k$ by $\pi^\prime$ on problem instance $\theta$. Recall that mean value of arm $k$ in problem instance $\theta$ $ is \gamma_k +C_k$ and that of corresponding arm in problem instance $f(\theta)$ is $\gamma_1-(\gamma_i+C_i)$. We have
	\begin{align*}
	\Regret_n(\pi^\prime,\theta) = \sum_{k\in [K]} \EEi{P_S}{N_k(n)}(\gamma_k+C_k-\gamma_{k^*}-C_{k^*}) \,,
	\end{align*}
	and
	\begin{align*}
	 \Regret_n(\pi,f(\theta))
	&= \sum_{k\in [K]} \EEi{P_S}{N_k(n)}\left (\max_{i \in [K]}\{\gamma_1-\gamma_i- C_i\}-(\gamma_1-\gamma_k- C_k)\right) \\
	&= \sum_{k\in [K]} \EEi{P_S}{N_k(n)}\left (\gamma_k+ C_k - \min_{i \in [K]}\{\gamma_i+ C_i\}\right) \\
	&=\Regret_n(\pi^\prime,\theta).
	\end{align*}
	\end{proof}
%\section{Proof of Theorem \ref{thm:K-SAPRegret}}
%Consider a $K$-armed stochastic bandit problem where reward distribution $\nu_i$ has mean  $\gamma_1-\gamma_i- \sum_{j< i}c_j$ for all $i >1$ and arm $1$ gives a fixed reward of value $0$. The arms have side-observation structure defined by graph $G_S$.  
%Given an arbitrary policy $\pi=(\pi_1, \pi_2, \cdots \pi_t )$ for the SAP, we obtain a policy for the bandit problem with side observation graph $G_S$ from $\pi$ as follows: Let $H_{t-1}$ denote the history, consisting of all arms played and the corresponding rewards, available to policy $\pi_{t-1}$ till time $t-2$. In round $t-1$, let $a_{t-1}$ denote the arm selected by the bandit policy,  $r_{t-1}$ the corresponding reward and $o_{t-1}$ the side-observation defined by graph $G_S$. Then, the next action $a_t$ is obtained as follows:
%\begin{equation}
%\label{eqn:SAPtoKBandit}
%a_t=
%\begin{cases}
%\pi_t(H_{t-1}\cup \{1, \emptyset
%\}) \mbox{ if } a_{t-1}= \mbox{arm 1}	\\
%\pi_t(H_{t-1} \cup \{i, r_{t-1}\cup o_{t-1}\}) \mbox{ if } a_{t-1}= \mbox{arm i}
%\end{cases}
%\end{equation}
%\noindent
%Conversely, let $\pi^\prime=\{\pi^\prime_1, \pi^\prime_2,\cdots\}$ denote an arbitrary policy for the $K$-armed bandit problem with side-observation graph. we obtain a policy the SAP as follows: Let $H^\prime_{t-1}$ denote the history, consisting of all actions played and feedback, available to policy $\pi^\prime_{t-1}$ till time $t-2$. Let $a^\prime_{t-1}$ denote the action selected by the SAP policy in round $t-1$ and observed feedback $F_t$. Then, the next action $a^\prime_t$ is obtained as follows:
%\begin{equation}
%\label{eqn:KBanditToSAP}
%a^\prime_t=
%\begin{cases}
%\pi^\prime_t(H^\prime_{t-1} \cup \{1, 0
%\}) \mbox{ if } a^\prime_{t-1}= \mbox{action 1}	\\
%\pi^\prime_t(H^\prime_{t-1} \cup \{i, \boldsymbol{1}\{\hat{Y}_t^1\neq \hat{Y}_t^2\}\cdots \boldsymbol{1}\{\hat{Y}_t^1\neq \hat{Y}_t^i\}\}) \mbox{ if } a_{t-1}= \mbox{action i}.
%\end{cases}
%\end{equation}
%We next show that regret of a policy $\pi$ on the SAP problem is same as that of the policy derived from it for the $K$-armed bandit problem with side information graph $G_S$, 
%and regret of $\pi^\prime$ on the $K$-armed bandit with side-observation graph  $G_S$ is same as that of the policy derived from it for the SAP.
%
%Given a policy $\pi$ for the SAP problem let $f_1(\pi)$ denote the policy obtained by the mapping defined in (\ref{eqn:SAPtoKBandit}). The regret of policy $\pi$ that plays actions $i$, $N_i^\psi(T)$ times is given by 
%\begin{eqnarray}
%R^\psi_T(\pi) &=&\sum_{i=1}^{K}\left [ \left (\gamma_{i}+\sum_{j< i} c_j\right )-\left (\gamma_{i^*}+\sum_{j < i^*} c_j\right )\right ]\mathbb{E}[N^\psi_i(T)]\\
%\end{eqnarray}
%Now, regret of regret policy $f_1(\pi)$ on the $K$-armed bandit problem with side-observation graph $G_S$ 
%\begin{equation}
%R^{\phi}_T(f_1(\pi))=\sum_{i=1}^{K} \left[\left (\gamma_1-\gamma_{i^*}-\sum_{j <i^*} c_j \right )-\left (\gamma_1- \gamma_{i}-\sum_{j < i} c_j \right )\right ]\mathbb{E}[N^{\phi}_i(T)],
%\end{equation}
%where $N^{\phi}_i(T)$ is the number of times arm $i$ is pulled by policy $f_1(\pi)$. Since the mapping is such that $N^{\phi}_i(T)=N^{\psi}_i(T)$, 
%$R^\phi_T(f_1(\pi))$ is same as $R^\psi_T(\pi)$. Further, given a policy $\pi^\prime$ on $\psi$ we can obtain a policy $f_2(\psi)$ for $\psi$ as defined in (\ref{eqn:KBanditToSAP}) and we can argue similarly that they are regret equivalent. This concludes the proof. 
%
%
%\section{Extension to context based prediction}
%\label{sec:Contextual}
%In this section we consider that the prediction errors depend on the context $X_t$, and in each round the learner can decide which action to apply based on $X_t$. Let  $\gamma_i(X_t)=\Pr\{\hat{Y}^1_t \neq \hat{Y}^2_t| X_t\}$ for all $i \in [K]$. We refer to this setting as  Contextual Sensor Acquisition Problem (CSAP) and denote it as $\psi_c=(K, \mathcal{A}, \mathcal{C}, (\gamma_i,c_i)_{i\in [K]})$. 
%
%Given $x \in \mathcal{C}$, let $L_t(a|x)$ denote the loss from action $a\in \mathcal{A}$ in round $t$. A policy on $\phi^c$ maps past history and current contextual information to an action. Let $\Pi^{\psi_c}$ denote set of policies on $\psi_c$ and for any policy $\pi \in \Pi^{\psi_c}$, let $\pi(x_t)$ denote the action selected when the context is $x_t$. For any sequence $\{x_t,y_t\}_{t>0}$, the regret of a policy $\pi$ is defined as:
%\begin{equation}
%	R^{\phi_c}_T(\pi)= \sum_{t=1}^{T} \mathbb{E}\left [L_t(\pi(x_t)|x_t)\right ]-\sum_{t=1}^{T}\min_{a \in \mathcal{A}} \mathbb{E} \left [ L_t(a|x_t)\right ]. 
%\end{equation}
%As earlier, the goal is to learn a policy that minimizes the expected regret, i.e., $\pi^*= \arg \min_{\pi \in \Pi^{\psi_c}} \mathbb{E}[R^{\psi_c}_T(\pi)].$
%
%In this section we focus on CSA-problem with two sensors and assume that sensor predictions errors are linear in the context. Specifically, we assume that there exists $\theta_1, \theta_2 \in \mathcal{R}^d$ such that $\gamma_1(x)=x^\prime\theta_1$ and $\gamma_2(x)+c=x^\prime\theta_2$ for all $x \in \mathcal{C}$, were $x^\prime$ denotes the transpose of $x$. By default all vectors are column vectors. In the following we establish that CSAP is regret equivalent to a stochastic liner bandits with varying decision sets. We first recall the stochastic linear bandit setup and relevant results. 
%
%\subsection{Background on Stochastic Linear Bandits}
%In round $t$, the learner is given a decision set $D_t \subset \mathcal{R}^d$ from which he has to choose an action. For a choice $x_t \in D_t$, the learner receives a reward $r_t=x_t^\prime\theta^* + \epsilon_t$, where $\theta^* \in \mathcal{R}^d$ is unknown and $\epsilon_t$ is random noise of zero mean. The learner's goal  is to maximize the expected accumulated reward $\mathbb{E}\left[\sum_{t=1}^{T} r_t \right]$ over a period $T$. If the leaner knows $\theta^*$, his optimal strategy is to select $x_t^*=\arg \max_{x \in D_t} x^\prime \theta^* $ in round $t$. The performance of any policy $\pi$ that selects action $x_t$ at time $t$ is 
%measured with respect to the optimal policy and is given by the expected regret  as follows
%\begin{equation}
%\label{eqn:LinearBanditRegret}
%R^L_T(\pi)= \sum (x_t^*)^\prime \theta^* - \sum x_t^\prime \theta^* .
%\end{equation}
%The above setting, where actions sets can change in every round, is introduced in\cite{NIPS2011_ImprovedAlgorithms_AbbasiPalSzepes} and is a more general setting than that studied in \cite{COLT08_StochasticLinearOptimization_DaniHayesKakad,MOR11_LinearlyParametrized_RusmevichientongTsitsiklis} where decision set is fixed. Further, the above setting also specializes the contextual bandit studied in \cite{WWW10_Contextaulbandits_LiChuWei}. The authors in \cite{NIPS2011_ImprovedAlgorithms_AbbasiPalSzepes} developed an
%`optimism in the face of uncertainty linear bandit algorithm' (OFUL) that achieves $\mathcal{O}(d \sqrt{T})$ regret with high probability when the random noise is $R$-sub-Gaussian for some finite $R$. The performance of OFUL is significantly better than $ConfidenceBall_2$ \cite{COLT08_StochasticLinearOptimization_DaniHayesKakad}, $UncertainityEllipsoid$ \cite{MOR11_LinearlyParametrized_RusmevichientongTsitsiklis}
%and $LinUCB$ \cite{WWW10_Contextaulbandits_LiChuWei}. 
%
%
%\begin{thm}
%	\label{thm:2CSAPRegret}
%Consider a CSA-problem with $K=2$ sensors. Let $\mathcal{C}$ be a bounded set and $\gamma_i(x)+c_i=x^\prime\theta_i$ for $i=1,2$ for all $x \in \mathcal{C}$. Assume $x^\prime \theta_1, x^\prime \theta_2 \in [0\; 1]$ for all $x \in \mathcal{C}$. Then, equivalent to a stochastic linear bandit. 
%\end{thm}
%
%
%\subsection{Proof of Theorem \ref{thm:2CSAPRegret}}
%Let $\{x_t,y_t\}_{t\geq 0}$ be an arbitrary sequence of context-label pairs. Consider a stochastic linear bandit where $D_t=\{0, x_t\} $ is a decision set in round $t$. 
%From the previous section, we know that given a context $x$, action $1$ is optimal if $\gamma_1(x)-\gamma_2(x) -c< 0$, otherwise  action $2$ is optimal. Let $\theta:=\theta_1-\theta_2$, then it boils down to check if $x^\prime\theta-c<0$ for each context $x\in \mathcal{C}$. 
%
%For all $t$, define $\epsilon_t= \boldsymbol{1}\{\hat{Y}^1_t \neq \hat{Y}^2_t\}-x_t^\prime\theta$. Note that $\epsilon_t \in [0 \;1]$ for all $t$, and since sensors do not have memory, they are conditionally independent given past contexts. Thus, $\{\epsilon_t\}_{t>0}$ are conditionally $R$-sub-Gaussian for some finite $R$.  
%
%Given a policy $\pi$ on a linear bandit we obtain next to play for the CSAP as follows: For each round $t$ define $a_t \in \mathcal{C}$ and $r_t \in \{0,1\}$ such that $a_t=0$ and $r_t=0$ if action $1$ is played in that round, otherwise set $a_t=x_t$ and $r_t=\boldsymbol{1}\{\hat{y}^1_t \neq \hat{y}^1_t \}$. Let $\mathcal{H}_{t}=\{(a_1, r_1)\cdots (a_{t-1},r_{t-1})\}$ denote the past actions and corresponding rewards observed till time $t-1$. In round $t$, after observing context $x_t$, we transfer  $((a_{t-1},r_{t-1}), D_t)$, where  $D_t=\{0,x_t\}$. If $\pi$ outputs $0 \in D_t$ as the optimal choice, we play action $1$, otherwise we play action $2$.
%
%Conversely,   suppose $\pi^\prime$ denote a policy for the CSAP problem we select action to play from decision set $D_t=\{0,x_t\}$ as follows.  For each round $t$ define $a^\prime_t \in {1,2}$ and $r^\prime_t \in \mathcal{R}$ such that $a^\prime_t=1$ and $r^\prime_t=\emptyset$ if $0$ is played otherwise set $a^\prime_t=2$ and $r^\prime_t=x_t^\prime\theta^* +\epsilon_t$ if $x_t$ is played.  Let $\mathcal{H}^\prime_{t}=\{(a^\prime_1, r^\prime_1)\cdots (a^\prime_{t-1},r^\prime_{t-1})\}$ denote the past actions and corresponding rewards observed till time $t-1$. In round $t$, after observing set  $D_t$, we transfer  $((a^\prime_{t-1},r^\prime_{t-1}), x_t)$ to policy $\pi^\prime$. If $\pi$ outputs action $1$ as the optimal choice, we play action $0$, otherwise we play $x_t$. 



\end{document}
