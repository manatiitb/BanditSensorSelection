%!TEX root =  main.tex
\newcommand{\set}{\leftarrow}
\newcommand{\hgamma}{\hat{\gamma}}
The reduction of the previous section suggests that one can  utilize 
an algorithm developed for stochastic bandits with side-observation to solve a SAP satisfying SD property.
In this paper we make use of Algorithm~1 of \cite{WGySz:NIPS15}. \todoc{Note the bug in the other paper.}
While this algorithm was proposed for stochastic bandits with Gaussian side observations, 
as noted in the above paper, the algorithm is also suitable for problems where the payoff distributions are sub-Gaussian.
As Bernoulli random variables are $\sigma^2=1/4$-sub-Gaussian (after centering),
the algorithm is also applicable in our case.


	\begin{minipage}{0.48\textwidth}
		\begin{algorithm}[H]
			\caption{Algorithm for SAP with SD property} 
			\label{alg:asym}
			\begin{algorithmic}[1]
				\STATE Play action $K$ and observe  $Y^1,\dots,Y^K$.
				\STATE Set $\hgamma_{i}^1 \set \one{Y^1\ne Y^i}$ for all $i\in [K]$.
				\STATE Initialize the exploration count: $n_e \set 0$.
				\STATE Initialize the allocation counts: $N_K(1) \set 1$.
				\FOR{$t=2,3,...$}
				\IF{$\frac{N(t-1)}{4\alpha \log t}\in C(\hgamma^{t-1})$} \label{alg:check}
				\STATE Set $I_t \set \argmin_{k\in [K]} c(k,\hgamma^{t-1})$. \label{alg:greedy}
				%		\STATE Set $n_e(t+1)=n_e(t)$.
				\ELSE
				\IF{$N_K(t-1)<\beta(n_e)/K$} \label{alg:starve}
				\STATE Set $I_t =K$. \label{alg:forced}
				\ELSE
				\STATE Set $I_t$ to some $i$ for which \label{alg:plan} \\
				$\quad$ $N_i(t-1)< u_i^*(\hgamma^{t-1})4\alpha\log t$.
				\ENDIF
				\STATE Increment exploration count: $n_e \set n_e+1$.
				\ENDIF
				\STATE Play $I_t$ and observe  $Y^1,\dots,Y^{I_t}$.
				\STATE For $i\in [I_t]$, set\\
				$\quad$ $\hgamma_i^t \set (1-1/t) \hgamma_i^{t-1} + 1/t \,\one{Y^1\ne Y^i}$.
				\ENDFOR
			\end{algorithmic}
		\end{algorithm}
	\end{minipage}


For the convenience of the reader, we give the algorithm resulting from applying the reduction to Algorithm~1 
of \cite{WGySz:NIPS15} in an explicit form.
For specifying the algorithm we need some extra notation.
Recall that given a SAP instance $\theta = (P,c)$, we let $\gamma_k = \Prob{Y\ne Y^k}$ where $(Y,Y^1,\dots,Y^K)\sim P$ and $k\in [K]$. Let $k^*=\arg\min_k \gamma_k +C_k$ denote the optimal action and $\Delta_k(\theta)=\gamma_k+C_k-\gamma_{k^*}+C_{k^*}$ the sub-optimality gap of arm $k$. Further, let $\Delta^*(\theta) = \min\{\Delta_k(\theta), k\neq k^* \}$ denote the smallest positive sub-optimality gap and define $\Delta_k^*(\theta) =\max\{\Delta_k(\theta), \Delta^*(\theta)\}$.

Since cost vector $c$ is fixed, in the following we use parameter $\gamma$ in place of $\theta$ to denote the problem instance.
%For an instance $\gamma$, let $\pi(1,\gamma)$ denote the index of the optimal action $(\pi(1,\gamma) = \min_j C_j+\gamma_j)$, $\pi(2,\gamma)$ denote the index of the second best action $(\pi(2,\gamma) = \min_{j\ne \pi(1,\gamma)} C_j +\gamma_j)$ and more generally let $\pi(k,\gamma)$ be the index of the $k$th best action.
A (fractional) allocation count $u\in [0,\infty)^K$ determines for each action $i$ how many times the
action is selected.
Thanks to the cascade structure, using an action $i$ implies observing the output of all the sensors with index $j$ less than equal to $i$. Hence, a sensor $j$ gets observed $u_j+u_{j+1}+\dots+u_K$ times.
We call an allocation count ``sufficiently informative'' if (with some level of confidence)
it holds that {\em (i)} for each suboptimal choice, 
the number of observations for the corresponding sensor is sufficiently large to distinguish
it from the optimal choice; and  {\em (ii)}  the optimal choice is also distinguishable from the second best choice.
We collect these counts into the set $C(\gamma)$ for a given parameter $\gamma$:
$C(\gamma) = \{ u\in [0,\infty)^K\,:\, 
u_j+u_{j+1}+\dots+u_K
\ge \frac{2\sigma^2}{(\Delta_j^*(\theta))^2}, j\in [K] \}$
(note that $\sigma^2=1/4$).
Further, let $u^*(\gamma)$
be the allocation count that minimizes the total expected excess cost over the set of sufficiently informative allocation counts:
In particular,  we let $u^*(\gamma) = \argmin_{u\in C(\gamma)} \ip{ u, \Delta(\theta) }$ 
with the understanding that for any optimal action $k$, $u_k^*(\gamma) = \min \{ u_k \,: u\in C(\gamma) \}$ (here, $\ip{x,y} = \sum_i x_i y_i$ is the standard inner product of vectors $x,y$).
For an allocation count $u\in [0,\infty)^K$ let $m(u)\in \N^K$ denote total sensor observations, where $m_j(u) = \sum_{i=1}^j u_i$ corresponds to observations of sensor $j$.

The idea of the algorithm shown as \cref{alg:asym} is as follows:
The algorithm keeps track of an estimate $\hgamma^{t}:=(\hgamma^t_i)_{i \in [K]}$ of $\gamma$ in each round, which is initialized by pulling arm $K$ as this arm
gives information about all the other arms.
In each round, the algorithm first checks whether given the current estimate $\hgamma^t$ and the current confidence level (where the confidence level is gradually increased over time), the current allocation count $N(t)\in \N^K$
is sufficiently informative (cf. line \ref{alg:check}). If this holds, the action that is optimal under $\hgamma(t)$ is chosen 
(cf. line \ref{alg:greedy}). If the check fails, we need to explore.
The idea of the exploration is that it tries to ensure that the ``optimal plan'' -- assuming $\hgamma$ is the ``correct'' parameter -- is followed (line \ref{alg:plan}). However, this is only reasonable, if all components of $\gamma$ are relatively well-estimated.
Thus, first the algorithm checks whether any of the components of $\gamma$ has a chance of being
extremely poorly estimated (line \ref{alg:starve}). Note that the requirement here is that a significant, but still altogether diminishing fraction of the \emph{exploration rounds} is spent on estimating each components: In the long run, the fraction of exploration rounds amongst all rounds itself is diminishing; hence the forced exploration of line \ref{alg:forced} overall has a small impact on the regret, while it allows to stabilize the algorithm.

\newcommand{\gap}{d}
\newcommand{\norm}[1]{\|#1\|}
For $\theta \in \TSD$, let $\gamma(\theta)$ be the error probabilities for the various sensors.
The following result follows from Theorem~6 of \cite{WGySz:NIPS15}:
\begin{thm}
Let $\epsilon>0$, $\alpha>2$ arbitrary and choose any non-decreasing $\beta(n)$ that satisfies $0\le \beta(n)\le n/2$ and $\beta(m+n)\le \beta(m)+\beta(n)$ for $m,n\in \N$.
Then, 
for any
$\theta\in \TSD$, letting $\gamma = \gamma(\theta)$
the expected regret of \cref{alg:asym} after $T$ steps satisfies 
\todoc{Actually, needs to be checked.. I also replaced $\gap_{\max}(\theta)$ with $1$.}
\begin{align*}
&R_T(\theta)
  \le  \Big( 2K+2+\frac{4K}{\alpha-2} \Big) 
  +  4K\sum_{s=0}^T \exp{\Big( \frac{-8\beta(s)\epsilon^2}{2K} \Big)}~~~~~ \\
 &\hspace{-.3cm} + 2 \beta\Big( 4\alpha\log T\sum_{i\in \iset{K}} u_i^*(\gamma,\epsilon)+K \Big)
  +  4\alpha\log T \sum_{i\in \iset{K}} u_i^*(\gamma,\epsilon)\gap_i(\gamma) \,,
\end{align*}
where $u_i^*(\gamma,\epsilon) = \sup\{u_i^*(\gamma')\,:\, \norm{\gamma'-\gamma}_{\infty}\le \epsilon \}$.
\label{thm:alg1-ub1}
\end{thm}  
Further specifying $\beta(n)$ and using the continuity of $u^*(\cdot)$ at $\theta$, it immediately follows that Algorithm~\ref{alg:asym} achieves asymptotically optimal performance: \todoc{I just copy\&pasted this.
We don't actually have a lower bound..}
\begin{cor}
\label{cor:alg1-asym-opt}
 Suppose the conditions of Theorem~\ref{thm:alg1-ub1} hold. Assume, furthermore, that $\beta(n)$ satisfies $\beta(n) = o(n)$ and $\sum_{s=0}^\infty \exp \left( -\frac{\beta(s)\epsilon^2}{2K\sigma^2} \right)<\infty$ for any $\epsilon>0$, then for any $\theta$ such that $u^*(\theta)$ is unique, 
\[
\limsup_{T\rightarrow \infty} R_T(\theta,c)/\log T \le 4\alpha \inf_{u\in C_\theta} \ip{u,\gap(\gamma(\theta))}\,.
\]
\end{cor}
Note that any $\beta(n) = an^b$ with $a\in (0,\frac{1}{2}]$, $b\in (0,1)$ satisfies the requirements in Theorem~\ref{thm:alg1-ub1} and Corollary~\ref{cor:alg1-asym-opt}.
%Also note that the algorithms presented in \cite{CaKvLe12,BuErSh14} do not achieve this asymptotic bound.

The algorithm in \ref{alg:asym} only estimates the disagreements $\P\{Y^1 \neq Y^j\}$ for all $j \in [K]$ which suffices to identify the optimal arm when the SAP satisfies the SD property (see Sec \ref{sec:Equiv}). Clearly, one can estimate pairwise disagreements probabilities $\P\{Y^i \neq Y^j\}$ for $i\neq j$ and use them to order the arms. We next develop an heuristic algorithm that uses this information and works for SAP satisfying WD property. 

\subsection{Weak Dominance Case}
%A natural algorithm n alternative algorithm is the elimination algorithm 
While the reduction scheme described above is optimal \todov{need to present a lower bound for reduction} when tests satisfy SD property, it is sub-optimal for tests satisfying weak-dominance property. We describe a variant of UCB algorithm in this context. For any suboptimal arm $j < i^*$,  $C_{i^*}-C_j \leq \P\{Y^{i^*} \neq Y^j\}$ holds, and when the WD property holds, we further have $C_j -C_{i^*} \geq \P\{Y^i \neq Y^j\}$ for all $j>i^*$. Thus, given the disagreement probabilities $\P\{Y^i \neq Y^j\}$ for all $i \neq j$, the set $\{i \in [K-1]: C_j - C_i \geq \P\{Y^i \neq Y^j\} \mbox{ for all } j>i\}$ includes the optimal arm. We use this idea in Algorithm \ref{alg:UCB} to identify the optimal arm when an instance of SAP satisfies the WD property. We will experimentally validate its performance  on real datasets in the next section.
\begin{minipage}{0.48\textwidth}
		\begin{algorithm}[H]
			\caption{Algorithm for SAP with WD property} %ALG1}
			\label{alg:UCB}
			\begin{algorithmic}[1]
				\STATE Play action $K$ and observe $Y^1,\dots,Y^K$
				\STATE Set $\hgamma_{ij}^1 \set \one{Y^i\ne Y^j}$ for all $i,j\in [K]$ and $i < j$.
				\STATE $n_i(1)\leftarrow \one{i=K} \; \forall i\in [K]$.
				\FOR{$t=2,3,...$}
				\STATE $U_{ij}^t=\hgamma_{ij}^{t-1} + \sqrt{\frac{1.5\log(t)}{n_j(t-1)}}$  $\;\forall \; i,j \in [K]$ and $i<j$ \label{algo:UCB}
				\STATE $S_t=\{i \in [K-1]: C_j-C_i \geq U_{ij}^t \;\forall \;   j > i \}$ \label{algo:sort}
				\STATE Set $I_t= \arg \min S_t \cup \{K\} $
				\STATE Play $I_t$ and observe $Y^1,\dots,Y^{I_t}$.
				 \FOR {$i\in [I_t]$}
				\STATE $n_i(t) \set n_i(t-1)+1$\\
				 \STATE $\hgamma_{ij}^t \set \left (1-\frac{1}{n_j(t)}\right )
				 \hgamma_{ij}^{t-1} + \frac{1}{n_j(t)} \,\one{Y^j\ne Y^i} \forall \; i<j \leq I_t$ \label{algo:Update}
				\ENDFOR
				\ENDFOR
			\end{algorithmic}
		\end{algorithm}
	\end{minipage}
	
	\vspace{.2cm}
<<<<<<< HEAD
The algorithm works as follows. It keeps track of $\hgamma^t_{ij}$ for all $i,j \in [K]$ and $i\neq j$ in each round, where $\hgamma_{ij}$ is an estimate of the probability $\P\{Y^i \neq Y^j\}$. In the first round, the algorithm plays arm $K$ and initializes its values. In each subsequent round, the algorithm computes the upper confidence value of $\hgamma^t_{ij}$ denoted as $U^t_{ij}$ (\ref{algo:UCB}) for all pairs $(i,j)$ and orders the arms: $i$ is considered better than arm $j$ if $C_j-C_i \geq U^t_{ij}$. Specifically, the algorithm plays an arm $i$ that satisfies $C_j-C_i \geq U^t_{ij}$ for all $j>i$ \ref{algo:sort}. If no such arm is found, then it plays arm $K$.  $n_j(t), j\in [K] $ counts the total number of observation of pairs $(Y^i, Y^j)$, for all $i<j$, till round $t$ and uses it to update the estimates $\hgamma_{ij}^t$ (\ref{algo:Update}).    % \rodos{Should we mention that due to lack of uniform learnability we do not present theoretical results here}
=======
The algorithm works as follows. It keeps track of $\hgamma^t_{ij}$ for all $i,j \in [K]$ and $i\neq j$ in each round, where $\hgamma_{ij}$ is an estimate of the probability $\P\{Y^i \neq Y^j\}$. In the first round, the algorithm plays arm $K$ and initializes its values. In each subsequent round, the algorithm computes the upper confidence value of $\hgamma^t_{ij}$ denoted as $U^t_{ij}$ (\ref{algo:UCB}) for all pairs $(i,j)$ and orders the arms: $i$ is considered better than arm $j$ if $C_j-C_i \geq U^t_{ij}$. Specifically, the algorithm plays an arm $i$ that satisfies $C_j-C_i \geq U^t_{ij}$ for all $j>i$ \ref{algo:sort}. If no such arm is found, then it plays arm $K$.  $n_j(t), j\in [K] $ counts the total number of observation of pairs $(Y^i, Y^j)$, for all $i<j$, till round $t$ and uses it to update the estimates $\hgamma_{ij}^t$ (\ref{algo:Update}).     \todov{Should we mention that due to lack of uniform learnability we do not present theoretical results here}
>>>>>>> 8a1525e9dc35ddceba91ca8cb176be45d921db8d
