%!TEX root =  main.tex
Let $\TSA$ be the set of all stochastic, cascaded sensor acquisition problems.
Thus, $\theta = (P,c)\in \TSA$ such that if $Y\sim P$ then $\gamma_k(\theta):=\Prob{Y\ne Y^k}$ 
is a decreasing sequence.
Given a subset $\Theta\subset \TSA$, we say that $\Theta$ is \emph{learnable} 
if there exists a learning algorithm $\Alg$ such that
for any $\theta\in \Theta$, the expected regret $\EE{ \Regret_n(\Alg,\theta) }$ 
of algorithm $\Alg$ on instance $\theta$ is sublinear.
A subset $\Theta$ is said to be a maximal learnable problem class if it is learnable and for any $\Theta'\subset \TSA$ superset
of $\Theta$, $\Theta'$ is not learnable.
In this section we study two special learnable problem classes, $\TSD\subset \TWD$, where the regularity properties of the instances in $\TSD$ are more intuitive, while $\TWD$ can be seen as a maximal extension of $\TSD$.

Let us start with some definitions.
Given an instance $\theta = (P,c)\in \TSA$, we can decompose $P$ into the joint distribution $P_S$ of the sensor outputs $S = (Y^1,\dots,Y^k)$ and the conditional distribution of the state of the environment, given the sensor outputs, $P_{Y|S}$.
Specifically, letting $(Y,S)\sim P$, for $s\in \{0,1\}^K$ and $y\in \{0,1\}$, $P_S(s) = \Prob{S = s}$ and $P_{Y|S}(y|s) = \Prob{Y=y|S=s}$. We denote this by $P = P_S \otimes P_{Y|S}$.
A learner who observes the output of all sensors for long enough is able to identify 
$P_S$ with arbitrary precision, while $P_{Y|S}$ remains hidden from the learner.
%A problem set $\Theta$ is said to be complete if $\{P_S\,:\, \exists P\in \Theta \text{ s.t. } P = P_S \otimes P_{Y|S} \}=M_1( \{0,1\}^K )$, i.e., all distributions over the sensor outputs are possible under some problem instance in $\Theta$.
 This leads to the following statement:
\begin{prop}
\label{prop:learnablemap}
%Let $\Theta \subset \TSA$ be complete. Then,
A subset $\Theta\subset \TSA$ is learnable 
if and only if there exists a map $a: M_1( \{0,1\}^K )\times \R_+^K \to [K]$ such that 
for any $\theta= (P,c)\in \Theta$ 
with decomposition $P = P_S \otimes P_{Y|S}$, $a(P_S)$ is an optimal action in $\theta$.
\end{prop}
\begin{proof}
$\Rightarrow$: Let $\Alg$ be an algorithm that achieves sublinear regret
and pick  an instance  $\theta = (P,c)\in\Theta$. Let $P = P_S \otimes P_{Y|S}$.
The regret $\Regret_n(\Alg,\theta)$ of $\Alg$ on instance $\theta$ can be written in the form
\begin{align*}
\Regret_n(\Alg,\theta) = \sum_{k\in [K]} \EEi{P_S}{N_k(n)} \Delta_k(\theta)\,,
\end{align*}
where $N_k(n)$ is the number of times action $k$ is chosen by $\Alg$ during the $n$ rounds while
$\Alg$ interacts with $\theta$, $\Delta_k(\theta) = c(k,\theta) - c^*(\theta)$ is the immediate regret
and $\EEi{P_S}{\cdot}$ denotes the expectation under the distribution induced by $P_S$.
In particular, $N_k(n)$ hides dependence on the iid sequence $Y_1,\dots,Y_n \sim P_S$ 
that we are taking the expectation over here. 
Since the regret is sublinear, for any $k$ suboptimal action, $\EEi{P_S}{N_k(n)} = o(n)$. 
Define $a(P_S,c) = \min \{ k\in [K]\,;\, \EEi{P_S}{N_k(n)} = \Omega(n) \,\}$. Then, $a$ is well-defined as the distribution of $N_k(n)$ for any $k$ depends only on $P_S$ and $c$. Furthermore, $a(P_S,c)$ selects an optimal action.

$\Leftarrow$: Let $a$ be the map in the statement and let $f:\N_+\to\N_+$ be such that $1\le f(n)\le n$ for any  $n\in \N$,
$f(n)/\log(n) \to n$ as $n\to \infty$ and $f(n)/n \to 0$ as $n\to \infty$ (say, $f(n) = \lceil \sqrt{n} \rceil$).
Consider the algorithm that chooses $I_t = K$ for the first $f(n)$ steps, after which it estimates $\hat{P}_S$ by
frequency counting and then uses $I_t = a(\hat{P}_S,c)$ in the remaining $n-f(n)$ trials. 
Pick any $\theta = (P,c)\in \Theta$ so that $P = P_S \otimes P_{Y|S}$. 
Note that by Hoeffding's inequality, 
$\sup_{y\in \{0,1\}^K} |\hat{P}_S(y)  - P_S(y)| \le \sqrt{\frac{K\log(4n)}{2f(n)}}$ holds with probability $1-1/n$.
Let $n_0$ be the first index such that for any $n\ge n_0$,
$\sqrt{\frac{K\log(4n)}{2f(n)}}\le \Delta^*(\theta) \doteq \min_{k:\Delta_k(\theta)>0} \Delta_k(\theta)$.
Such an index $n_0$ exists by our assumptions that $f$ grows faster than $n \mapsto \log(n)$.
For $n\ge n_0$, the expected regret of $\Alg$ is at most $n \times 1/n + f(n) (1-1/n) \le 1+f(n) = o(n)$.
\end{proof}

An action selection map  $a: M_1( \{0,1\}^K )\times \R_+^K \to [K]$ is said to be \emph{sound} for an instance 
$\theta\in \TSA$ with $\theta = (P_S\otimes P_{Y|S},c)$ if $a(P_S,c)$ selects an optimal action in $\theta$.
With this terminology, the previous proposition says that a set of instances $\Theta$ is learnable if and only if there exists a
sound action selection map for all the instances in $\Theta$.

A class of sensor acquisition problems that contains instances that satisfy the so-called \emph{strong dominance} condition 
will be shown to be learnable:
\begin{defi}[Strong Dominance]
	An instance $\theta = (P,c)\in \TSA$  is said to satisfy the \emph{strong dominance property} if 
	it holds in the instance that if a sensor predicts correctly
	then all the sensors in the subsequent stages of the cascade also predict correctly, i.e., 
	for any $i\in [K]$,
	\begin{equation}
	\label{eqn:DominanceCondition}
	Y^i=Y \,\, \Rightarrow\,\, Y^{i+1}= \dots =  Y^K = Y
	\end{equation}
	almost surely (a.s.)
	where $(Y,Y^1,\dots,Y^K)\sim P$.
\end{defi}
Let $\TSD = \{ \theta\in \TSA\,:\, \theta \text{ satisfies the strong dominance condition } \}$.
%\begin{thm}

Let $\TSD = \{ \theta\in \TSA\,:\, \theta \text{ satisfies the strong dominance condition } \}$.
\begin{thm}
\label{thm:tsdlearnable}
The set $\TSD$ is learnable.
\end{thm}
We start with a proposition that will be useful beyond the proof of this result.
In this proposition, $\gamma_i = \gamma_i(\theta)$ for $\theta=(P,c)\in \TSA$ and $(Y,Y^1,\dots,Y^K) \sim P$.
\begin{prop}\label{prop:gammadiff}
For any $i,j\in [K]$, $\gamma_i - \gamma_j = \Prob{Y^i\ne Y^j} - 2\Prob{ Y^j \ne Y, Y^i=Y}$.
\end{prop}
\begin{proof}

We construct a map as required by~\cref{prop:learnablemap}.
Take an instance $\theta = (P,c)\in \TWD$ and let $P = P_S \otimes P_{Y|S}$ be its decomposition
as defined above.
Let $\gamma_i = \Prob{Y^i \ne Y}$, $(Y,Y^1,\dots,Y^K)\sim P$.
For identifying an optimal action in $\theta$, it clearly suffices
to know the sign of $\gamma_i + C_i - (\gamma_j +C_j)$ for all pairs $i,j\in [K]^2$.
Since $C_i - C_j$ is known, it remains to study $\gamma_i-\gamma_j$.
Without loss of generality (WLOG) let $i<j$.
Then, 
=======
We have:
>>>>>>> 7ec48a671514c7e5a1d1b3c7c8ff259af505fdaa
\begin{align*}
\MoveEqLeft 0  \le \gamma_i  - \gamma_j = \Prob{Y^i\ne Y} - \Prob{Y^j\ne Y} \\
& = \cancel{\Prob{Y^i\ne Y, Y^i=Y^j}} + \Prob{ Y^i\ne Y, Y^i\ne Y^j } - \\
& - \left\{ 
       \cancel{\Prob{Y^j\ne Y, Y^i=Y^j}} + \Prob{ Y^j\ne Y, Y^i\ne Y^j }\right\}\\
& = \Prob{ Y^i\ne Y, Y^i \ne Y^j } + \Prob{Y^i=Y,Y^i\ne Y^j}       \\
& - \left\{ 
	  \Prob{ Y^j \ne Y, Y^i\ne Y^j } + \Prob{ Y^i=Y,Y^i\ne Y^j}
	 \right\}\\
& \stackrel{\footnotesize (a)}{=} \Prob{ Y^j \ne Y^i } -2 \Prob{ Y^j\ne Y, Y^i = Y }\,,
\end{align*}
where in $(a)$ we used that $\Prob{ Y^j \ne Y, Y^i\ne Y^j } =  \Prob{ Y^j\ne Y,Y^i= Y}$ and also
$\Prob{ Y^i=Y,Y^i\ne Y^j} = \Prob{ Y^j\ne Y,Y^i= Y}$
which hold because $Y,Y^i,Y^j$ only take on two possible values.
\end{proof}
\begin{proof}[Proof of \cref{thm:tsdlearnable}]
We construct a map as required by~\cref{prop:learnablemap}.
Take an instance $\theta = (P,c)\in \TSD$ and let $P = P_S \otimes P_{Y|S}$ be its decomposition as before.
Let $\gamma_i = \Prob{Y^i \ne Y}$, $(Y,Y^1,\dots,Y^K)\sim P$, $C_i = c_1+\dots+c_i$.
For identifying an optimal action in $\theta$, it clearly suffices
to know the sign of $\gamma_i + C_i - (\gamma_j +C_j) = \gamma_i-\gamma_j + (C_i-C_j)$ for all pairs $i,j\in [K]^2$.
Without loss of generality (WLOG) let $i<j$. By \cref{prop:gammadiff},
$\gamma_i - \gamma_j = \Prob{ Y^i \ne Y^j } -2 \Prob{ Y^j\ne Y, Y^i = Y }$.
Now, since $\theta$ satisfies the strong dominance condition, $ \Prob{ Y^j\ne Y, Y^i = Y } = 0$.
Thus, $\gamma_i - \gamma_j = \Prob{ Y^i \ne Y^j }$
which is a function of $P_S$ only.
Since $(C_i)_i$ are known, a map as required by~\cref{prop:learnablemap} exists.
\end{proof}
The proof motivates the definition of weak dominance, a concept that we develop next through a series of smaller
propositions. In these propositions, as before $(Y,Y^1,\dots,Y^K) \sim P$ where $P\in M_1(\{0,1\}^{K+1})$,
 $\gamma_i = \Prob{Y^i \ne Y}$, $i\in [K]$, and $C_i = c_1 + \cdots + c_i$.
We start with a corollary of \cref{prop:gammadiff}
\begin{cor}
\label{cor:gammadiff}
Let $i<j$. Then $0\le \gamma_i -\gamma_j \le \Prob{Y^i\ne Y^j}$.
\end{cor}
\begin{prop}
\label{prop:ilej}
Let $i<j$. Assume 
\begin{align}
\label{eq:cond1}
C_j - C_i \not\in [\gamma_i - \gamma_j, \Prob{Y^i\ne Y^j} )\,.
\end{align}
Then $\gamma_i + C_i \le \gamma_j + C_j$ if and only if $C_j - C_i \ge \Prob{Y^i\ne Y^j}$.
\end{prop}
\begin{proof}
\noindent $\Rightarrow$: From the premise, it follows that $C_j - C_i \ge \gamma_i - \gamma_j$.
Thus, by~\eqref{eq:cond1}, $C_j - C_i \ge \Prob{Y^i\ne Y^j}$.
\noindent $\Leftarrow$: We have $C_j - C_i \ge \Prob{Y^i \ne Y^j} \ge \gamma_i -\gamma_j$, where the last
inequality is by \cref{cor:gammadiff}.
\end{proof}
\begin{prop}
\label{prop:jlei}
Let $j<i$. Assume
\begin{align}
\label{eq:cond2}
C_i - C_j \not\in (\gamma_j - \gamma_i, \Prob{Y^i \ne Y^j} ]\,.
\end{align}
Then, $\gamma_i + C_i \le \gamma_j + C_j$ if and only if $C_i - C_j \le \Prob{Y^i \ne Y^j}$.
\end{prop}
\begin{proof}
\noindent $\Rightarrow$: The condition $\gamma_i + C_i \le \gamma_j + C_j$ implies that $\gamma_j -\gamma_i \ge C_i - C_j$.
By \cref{cor:gammadiff} we get $\Prob{Y^i \ne Y^j} \ge C_i - C_j$.
\noindent $\Leftarrow$: Let $C_i - C_j \le \Prob{Y^i \ne Y^j}$. Then, by \eqref{eq:cond2}, $C_i - C_j \le \gamma_j - \gamma_i$.
\end{proof}
These results motivate the following definition:
\begin{defi}[Weak Dominance]
	An instance $\theta = (P,c)\in \TSA$  is said to satisfy the \emph{weak dominance property} if 
	for $i = a^*(\theta)$,
	\begin{align}
	\label{eq:wd} \forall j>i\,\,: \,\, C_j - C_i \ge \Prob{Y^i\ne Y^j}\,.
	\end{align}
We denote the set of all instances in $\TSA$ that satisfies this condition by $\TWD$.	
\end{defi}
Note that $\TSD\subset \TWD$ since for any $\theta\in \TSD$, any $j>i = a^*(\theta)$, on the one hand $C_j - C_i \ge \gamma_i - \gamma_j$, while on the other hand, by the strong dominance property, $\Prob{Y^i\ne Y^j} = \gamma_i - \gamma_j$.
We propose the following action selector $\awd: M_1(\{0,1\}^K) \times \R_+^K \to [K]$:
\begin{defi}\label{def:awd}
For $(P_S,c) \in M_1(\{0,1\}^K) \times \R_+^K$ let $\awd(P_S,c)$ denote the smallest index $i\in [K]$ such that
\begin{subequations}
\begin{align}
\forall j<i \,\,:\,\, C_i - C_j < \Prob{ Y^i \ne Y^j }\,, \label{eq:wd1}\\ 
\forall j>i \,\,:\,\, C_j - C_i \ge \Prob{ Y^i \ne Y^j }\,, \label{eq:wd2}
\end{align}
\end{subequations}
where $C_i = c_1+\cdots + c_i$, $i\in [K]$ and $(Y^1,\dots,Y^K) \sim P_S$.
(If no such index exists, $\awd$ is undefined, i.e., $\awd$ is a partial function.)
\end{defi}
\begin{prop}
\label{prop:awdwelldef}
For any $\theta = (P,c)\in \TWD$ with $P = P_S\otimes P_{Y|S}$, $\awd(P_S,c)$ is well-defined.
\end{prop}
\begin{proof}
Let $\theta\in \TWD$, $i = a^*(\theta)$. Obviously, \eqref{eq:wd2} holds by the definition of $\TWD$.
Thus, the only question is whether \eqref{eq:wd1} also holds.
We prove this by contadiction:
Thus, assume that \eqref{eq:wd1} does not hold, i.e., for some $j<i$, $C_i-C_j \ge \Prob{Y^i \ne Y^j}$. Then, by \cref{cor:gammadiff}, $\Prob{ Y^i \ne Y^j} \ge \gamma_j - \gamma_i$, hence $\gamma_j + C_j \le \gamma_i + C_i$, which contradicts the definition of $i$, thus finishing the proof.
\end{proof}
\begin{prop}
\label{prop:awdsound}
The map $\awd$ is sound over $\TWD$: In particular, for any
$\theta = (P,c)\in \TWD$ with $P = P_S\otimes P_{Y|S}$, $\awd(P_S,c)= a^*(\theta)$.
\end{prop}
\begin{proof}
Take any $\theta\in \TWD$ and let $\theta = (P,c)$ with $P = P_S\otimes P_{Y|S}$, $i = \awd(P_S,c)$, $j = a^*(\theta)$.
If $i=j$, there is nothing to be proven. Hence, first assume that $j>i$. Then, by~\eqref{eq:wd2}, $C_j - C_i \ge \Prob{Y^i \ne Y^j}$.
By \cref{cor:gammadiff}, $\Prob{Y^i \ne Y^j } \ge \gamma_i - \gamma_j$. Combining these two inequalities we get that
$\gamma_i + C_i \le \gamma_j + C_j$, which contradicts with the definition of $j$.
Now, assume that $j<i$. Then, by~\eqref{eq:wd}, $C_i - C_j \ge \Prob{Y^i \ne Y^j}$.
However, by~\eqref{eq:wd1}, $C_i -C_j < \Prob{Y^i \ne Y^j}$, thus $j<i$ cannot hold either and we must have $i=j$.
\end{proof}
\begin{cor}\label{cor:twdlearnable}
The set $\TWD$ is learnable.
\end{cor}
\begin{proof}
By \cref{prop:awdwelldef}, $\awd$ is well-defined over $\TWD$, while by \cref{prop:awdsound}, $\awd$ is sound over $\TWD$.
By \cref{prop:learnablemap}, $\TWD$ is learnable, as witnessed by $\awd$. \todoc{We should add definitions for these concepts..
namely, $\awd$ well-defined over $\TWD$, $\awd$ sound over $\TWD$, etc.}
\end{proof}
\begin{prop}
\label{prop:awdcorrectimplieswd}
Let $\theta \in \TSA$, $\theta = (P,c)$, $P = P_S\otimes P_{Y|S}$ be such that $\awd$ is defined for $P_S,c$
and $\awd(P_S,c) = a^*(\theta)$. Then $\theta \in \TWD$.
\end{prop}
\begin{proof}
Immediate from the definitions.
\end{proof}
An immediate corollary of the previous proposition is as follows:
\begin{cor}\label{cor:awdoutsideincorrect}
Let $\theta \in \TSA$, $\theta = (P,c)$, $P = P_S \otimes P_{Y|S}$. 
Assume that $\awd$ is defined for $(P_S,c)$ and $\theta \not\in \TWD$. Then $\awd(P_S,c) \ne a^*(\theta)$.
\end{cor}
The next proposition states that $\awd$ is essentially the only sound action selector map defined for
 all instances derived from instances of $\TWD$:
\begin{prop}\label{prop:awdunique}
Take any action selector map $a: M_1( \{0,1\}^K ) \times \R_+^K \to [K]$ which is sound over $\TWD$.
Then, for any $(P_S,c)$ such that $\theta = (P_S\otimes P_{Y|S},c)\in \TWD$ with some $P_{Y|S}$,
 $a(P_S,c) = \awd(P_S,c)$.
\end{prop}
\begin{proof}
Pick any $\theta = (P_S\otimes P_{Y|S},c)\in \TWD$. If $A^*(\theta)$ is a singleton, then clearly $a(P_S,c) = \awd(P_S,c)$ since both are sound over $\TWD$.
Hence, assume that $A^*(\theta)$ is not a singleton.
Let $i = a^*(\theta) = \min A^*(\theta)$ and let $j = \min A^*(\theta) \setminus \{ i \}$.
We argue that $P_{Y|S}$ can be changed so that on the new instance $i$ is still an optimal action, while
$j$ is not an optimal action, while the new instance $\theta' = (P_S \otimes P_{Y|S}', c)$ is in $\TWD$.

The modification is as follows:
Consider any $y^{-j} \doteq (y^1,\dots,y^{j-1},y^{j+1},\dots,y^K)\in \{0,1\}^{K-1}$.
For $y,y^j\in \{0,1\}$, define 
$q(y|y^j) = P_{Y|S}(y|y^1, \dots, y^{j-1}, y^j, y^{j+1},\dots, y^K)$
and similarly let
$q'(y|y^j) = P_{Y|S}'(y|y^1, \dots, y^{j-1}, y^j, y^{j+1},\dots, y^K)$
Then, we let $q'(0|0) = 0$ and $q'(0|1) = q(0|0) + q(0|1)$,
while we let  $q'(1|1) = 0$ and $q'(1|0) = q(1|1) + q(1|0)$.
This makes $P_{Y|S}'$ well-defined ($P_{Y|S}'(\cdot|y^1,\dots,y^K)$ is a distribution for any $y^1,\dots,y^K$).
Further, we claim that the transformation has the property that 
it leaves $\gamma_p$ unchanged for $p\ne j$, while $\gamma_j$ is guaranteed to decrease.
To see why $\gamma_p$ is left unchanged for $p\ne j$ note that
$\gamma_p = \sum_{y^p}  P_{Y^p}(y^p) P_{Y|Y^p}(1-y^p|y^p)$.
Clearly, $P_{Y^p}$ is left unchanged.
Introducing $y^{-k}$ to denote a tuple where the $k$th component is left out,
$P_{Y|Y^p}(1-y^p|y^p) = \sum_{y^{-p,-j}} P_{Y|Y^1,\dots,Y^K}( 1-y^p | y^1,\dots, y^{j-1}, 0, y^{j+1}, \dots, y^K )
+P_{Y|Y^1,\dots,Y^K}( 1-y^p | y^1,\dots, y^{j-1}, 1, y^{j+1}, \dots, y^K )$
and by definition,
\begin{align*}
\MoveEqLeft P_{Y|Y^1,\dots,Y^K}( 1-y^p | y^1,\dots, y^{j-1}, 0, y^{j+1}, \dots, y^K )\\
&\quad +P_{Y|Y^1,\dots,Y^K}( 1-y^p | y^1,\dots, y^{j-1}, 1, y^{j+1}, \dots, y^K )\\
&
=
P_{Y|Y^1,\dots,Y^K}'( 1-y^p | y^1,\dots, y^{j-1}, 0, y^{j+1}, \dots, y^K )\\
&\quad+P_{Y|Y^1,\dots,Y^K}'( 1-y^p | y^1,\dots, y^{j-1}, 1, y^{j+1}, \dots, y^K )\,,
\end{align*}
where the equality holds because ``$q'(y|0)+q'(y|1) = q(y|0) + q(y|1)$''.
Thus, $P_{Y|Y^p}(1-y^p|y^p) = P_{Y|Y^p}'(1-y^p|y^p)$ as claimed.
That $\gamma_j$ is non-increasing follows with an analogue calculation.
In fact, this shows that $\gamma_j$ is strictly decreased
if for any $(y^1,\dots,y^{j-1},y^{j+1},\dots,y^K)\in \{0,1\}^{K-1}$, either $q(0|0)$ or $q(1|1)$ was positive.
If these are never positive, this means that $\gamma_j=1$. 
But then $j$ cannot be optimal since $c_j>0$.
Since $j$ was optimal, $\gamma_j$ is guaranteed to decrease.

Finally, it is clear that the new instance is still in $\TWD$ since  $a^*(\theta)$ is left unchanged.
\end{proof}
The next result shows that
the set $\TWD$ is essentially a maximal learnable set in $\mathrm{dom}(\awd)$:
\begin{thm}
Let $a: M_1(\{0,1\}^K) \times \R_+^K \to [K]$ be an action selector map
such that $a$ is sound over the instances of $\TWD$.
Then there is no instance $\theta = (P_S\otimes P_{Y|S},c)\in \TSA\setminus \TWD$ such that 
$(P_S,c)\in \mathrm{dom}(\awd)$, the optimal action of $\theta$ is unique
\todoc{It would be nice to remove this uniqueness assumption, but I don't see how this could be made to work.}
 and $a(P_S,c) = a^*(\theta)$.
\end{thm}
Note that $\mathrm{dom}(\awd)\setminus \{ (P_S,c) \,:\, \exists P_{Y|S} \textrm{ s.t. } (P_S \otimes P_{Y|S},c) \in \TWD \} \ne \emptyset$, i.e., the theorem statement is non-vacuous.
In particular, for $K=2$, consider $(Y,Y^1,Y^2)$ such that $Y$ and $Y^1$ are independent and $Y^2 = 1-Y^1$, we can see that the resulting instance gives rise to $P_S$ which is in the domain of $\awd$ for any $c\in \R_+^K$ (because here $\gamma_1 = \gamma_2 = 1/2$, thus $\gamma_1 - \gamma_2 = 0$ while $\Prob{Y^1\ne Y^2}=1$).
\begin{proof}
Let $a$ as in the theorem statement. By~\cref{prop:awdunique}, $\awd$ is the unique sound action-selector map over $\TWD$.
Thus, for any $\theta = (P_S\otimes P_{Y|S},c)\in \TWD$, $\awd(P_S,c) = a(P_S,c)$.
Hence, the result follows from \cref{cor:awdoutsideincorrect}.
\end{proof}
While $\TWD$ is learnable, it is not uniformly learnable, i.e., the minimax regret $\Regret_n^*(\TWD) = \inf_{\Alg} \sup_{\theta\in \TWD} \Regret_n(\Alg,\theta)$ over $\TWD$ grows linearly:
\begin{thm}
\label{thm:nonunif}
$\TWD$ is not uniformly learnable:
$\Regret_n^*(\TWD) = \Omega(n)$.
\end{thm}
\begin{proof}
We first consider the case when $K=2$ and arbitrarily choose $C_2 - C_1 = 1/4$. 
\todoc{The theorem statement should be refined or this text..}
We will consider two instances, $\theta,\theta'\in \TWD$ such that for instance $\theta$, 
action $k=1$ is optimal with an action gap of $c(2,\theta) - c(1,\theta) = 1/4$ between the cost of the second and the first
action,  while for instance $\theta'$, $k=2$ is the optimal action and the action gap is $c(1,\theta) - c(2,\theta) = \epsilon$
where $0<\epsilon<3/8$.
Further, the entries in $P_S(\theta)$ and $P_S(\theta')$ differ by at most $\epsilon$. 
From this, a standard reasoning gives that no algorithm can achieve sublinear minimax regret over $\TWD$ because any
algorithm is only able to identify $P_S$. \todoc{Add notation of $P_S(\theta)$ early on. Probably a good idea to add $P_S(\Theta)$ as a notation too for the ``projection'' of $\Theta$ to $P_S$. Also, we should probably remove $c$ from the instance definition;
in every case we are reasoning for a fixed $c$, hence it is superfluous to keep $c$ in the instance definition.}

The constructions of $\theta$ and $\theta'$ are shown in \cref{tab:nonunif}:
The entry in a cell gives the probability of the event as specified by the column and row labels.
For example, in instance $\theta$, $3/8$ is the probability of $Y=Y^1=Y^2$, 
while the probability of $Y^1=Y\ne Y^2$ is $1/8$. Note that the cells with zero actually 
correspond to impossible events, i.e., these cannot be assigned a positive probability.
The rationale of a redundant (and hence sparse) table is so that probabilities of certain events of interest, such as $Y^1\ne Y^2$ are easier to determine based on the table. The reader should also verify that the positive probabilities correspond to events that are possible.
\bgroup
\def\arraystretch{1.5}
\begin{table}[]
\centering
%\hspace*{-0.1in}
\begin{tabular}{|c|c|c|c|}
\hline
\multicolumn{2}{|c|}{Instance $\theta$}  & $Y^1=Y^2$     & $Y^1\ne Y^2$  \\ \hline
\multirow{2}{*}{$Y^1= Y$}   & $Y^2= Y$   & $\frac{3}{8}$ & $0$           \\ \cline{2-4} 
                            & $Y^2\ne Y$ & $0$ & $\frac{1}{8}$ \\ \hline
\multirow{2}{*}{$Y^1\ne Y$} & $Y^2= Y$   & $0$ & $\frac{1}{8}$           \\ \cline{2-4} 
                            & $Y^2\ne Y$ & $\frac{3}{8}$ & $0$ \\ \hline
\end{tabular}
\begin{tabular}{|c|c|c|c|}
\hline
\multicolumn{2}{|c|}{Instance $\theta'$}  & $Y^1=Y^2$     & $Y^1\ne Y^2$  \\ \hline
\multirow{2}{*}{$Y^1= Y$}   & $Y^2= Y$   & $\frac{3}{8}-\epsilon$ & $0$           \\ \cline{2-4} 
                            & $Y^2\ne Y$ & $0$ & $0$ \\ \hline
\multirow{2}{*}{$Y^1\ne Y$} & $Y^2= Y$   & $0$ & $\frac{2}{8}+\epsilon$           \\ \cline{2-4} 
                            & $Y^2\ne Y$ & $\frac{3}{8}$ & $0$ \\ \hline
\end{tabular}
\vspace*{0.1in}
\caption{The construction of two problem instances for the proof of \cref{thm:nonunif}.}
\label{tab:nonunif}
\end{table}
\egroup

We need to verify the following:
{\em (i)} $\theta,\theta'\in \TWD$;
{\em (ii)} the optimality of the respective actions in the respective instances;
{\em (iii)} the claim concerning the size of the action gaps;
{\em (iv)} that $P_S(\theta)$ and $P_S(\theta')$ are close.
Details of the calculations to support {\em (i)}--{\em (iii)} can be found in \cref{tab:nonunif2}.
The row marked by $(*)$ supports that the instances are proper SAP instances.
In the row marked by $(**)$, there is no requirement for $\theta'$ because 
in $\theta'$ action two is optimal, and hence there is no action with larger index 
than the optimal action, hence $\theta'\in \TWD$ automatically holds.
To verify the closeness of $P_S(\theta)$ and $P_S(\theta')$ we actually 
would need to first specify $P_S$ (the tables do not fully specify these).
However, it is clear the only restriction we put on $P_S$ is the value of $\Prob{Y^1\ne Y^2}$ (and
that of $\Prob{Y^1=Y^2}$) and these values are within an $\epsilon$ distance of each other.
Hence, $P_S$ can also be specified to satisfy this. In particular, one possibility for $P$ and $P_S$ are given in \cref{tab:nonunif3}.
\bgroup
\def\arraystretch{1.5}
\begin{table}[]
\centering
%\hspace*{-0.1in}
\begin{tabular}{|c|c|c|}
\hline
                                                    & $\theta$                & $\theta'$ \\ \hline
$\gamma_1 = \Prob{Y^1\ne Y}$ & $\frac{1}{4}$           & $\frac{5}{8}+\epsilon$ \\ \hline
$\gamma_2 = \Prob{Y^2\ne Y}$ & $\frac{1}{4}$           & $\frac{3}{8}$ \\ \hline
$\gamma_2 \le \gamma_1 \mbox{}^{(*)}$        & \checkmark           & \checkmark \\ \hline
$c(1,\cdot)$                                 & $\frac{1}{4}$           & $\frac{5}{8}+\epsilon$ \\ \hline
$c(2,\cdot)$                                 & $\frac{2}{4}$           & $\frac{5}{8}$ \\ \hline
$a^*(\cdot)$                                 & $k=1$                   & $k=2$ \\ \hline
$\Prob{Y^1\ne Y^2}$                   & $\frac{1}{4}$         & $\frac{1}{4}+\epsilon$ \\ \hline
$\theta \in \TWD  \mbox{}^{(**)}$                        & $\frac{1}{4}\ge \frac14$ \checkmark & \checkmark \\ \hline
$|c(1,\cdot)-\c(2,\cdot)|$              & $\frac{1}{4}$         & $\epsilon$ \\ \hline
\end{tabular}
\vspace*{0.1in}
\caption{Calculations for the proof of \cref{thm:nonunif}.}
\label{tab:nonunif2}
\end{table}
\egroup

\bgroup
\def\arraystretch{1.5}
\begin{table}[]
\centering
%\hspace*{-0.1in}
\begin{tabular}{|c|c|c||c|c|}
\hline
$Y^1$ & $Y^2$ & $Y$ & $\theta$ & $\theta'$ \\ \hline\hline
$0$ & $0$ & $0$ & $\frac38$ & $\frac38-\epsilon$ \\ \hline
$0$ & $0$ & $1$ & $\frac38$ & $\frac38-\epsilon$ \\ \hline
$0$ & $1$ & $0$ & $0         $ & $0                       $ \\ \hline
$0$ & $1$ & $1$ & $0         $ & $0                       $ \\ \hline
$1$ & $0$ & $0$ & $\frac18$ & $\frac28+\epsilon$ \\ \hline
$1$ & $0$ & $1$ & $\frac18$ & $0                       $ \\ \hline
$1$ & $1$ & $0$ & $0         $ & $0                       $ \\ \hline
$1$ & $1$ & $1$ & $0         $ & $0                       $ \\ \hline
\end{tabular}
\mbox{}
\vspace*{2in}
\hspace*{0.5in}
\mbox{}
\begin{tabular}{|c|c||c|c|}
\hline
$Y^1$ & $Y^2$ &$\theta$ & $\theta'$ \\ \hline\hline
$0$ & $0$ & $\frac68$ & $\frac68-\epsilon$ \\ \hline
$0$ & $1$ & $0         $ & $0                       $ \\ \hline
$1$ & $0$ & $\frac28$ & $\frac28+\epsilon$ \\ \hline
$1$ & $1$ & $0         $ & $0                       $ \\ \hline
\end{tabular}

\vspace*{0.1in}
\caption{Probability distributions for instances $\theta$ and $\theta'$. On the left are shown the joint
probability distributions, while on the right are shown their marginals
for the sensors.}
\label{tab:nonunif3}
\end{table}
\egroup


\end{proof}
