%!TEX root =  main.tex
\newcommand{\set}{\leftarrow}
\newcommand{\hgamma}{\hat{\gamma}}
The reduction of the previous section suggests that one can play in an SAP instance by utilizing 
an algorithm developed for stochastic bandits with side-observation.
In this paper we make use of Algorithm~1 of \citet{WGySz:NIPS15}. \todoc{Note the bug in the other paper.}
While this algorithm was proposed for stochastic bandits with Gaussian side observations, 
as noted in the above paper, the algorithm is also suitable for problems where the payoff distributions are subgaussian.
As Bernoulli random variables are $\sigma^2=1/4$-subgaussian (after centering),
the algorithm is also applicable in our case.

\begin{wrapfigure}{L}{0.55\textwidth}
\vspace{-0.4cm}
\begin{minipage}{0.54\textwidth}
\begin{algorithm}[H]
\caption{} %ALG1}
\label{alg:asym}
\begin{algorithmic}[1]
\STATE Inputs: $\alpha>0$ and $\beta: \N \to [0,\infty)$.
\STATE Play action $K$ and observe the sensor outputs $Y^1,\dots,Y^K$.
\STATE Set $\hgamma \set (0,\one{Y^1\ne Y^2},\dots,\one{Y^1\ne Y^K})$.
\STATE Initialize the exploration count: $n_e \set 0$.
\STATE Initialize the allocation counts: $N_i = \one{i=K}$, $i\in [K]$.
\FOR{$t=2,3,...$}
	\IF{$\frac{N}{4\alpha \log t}\in C(\hgamma)$} \label{alg:check}
		\STATE Set $I \set \argmin_{k\in [K]} c_k(\hgamma)$. \label{alg:greedy}
%		\STATE Set $n_e(t+1)=n_e(t)$.
	\ELSE
		\IF{$\min_{i\in \iset{K}}m_i(N)<\beta(n_e)/K$} \label{alg:starve}
			\STATE Set $I = \argmin_{i\in \iset{K}}m_i(N)$. \label{alg:forced}
		\ELSE
			\STATE Set $I$ to some $i$ for which \label{alg:plan} \\
			$\qquad$ $N_i(t)< u_i^*(\hgamma)4\alpha\log t$.
		\ENDIF
		\STATE Increment exploration count: $n_e \set n_e+1$.
	\ENDIF
	\STATE Play $I$ and observe the sensor outputs $Y^1,\dots,Y^{I}$.
	\STATE For $i\in [I_t]$, set\\
	$\qquad$ $\hgamma_i \set (1-1/t) \hgamma_i + 1/t \,\one{Y^1\ne Y^i}$.
\ENDFOR
\end{algorithmic}
\end{algorithm}
\end{minipage}
\vspace{-0.3cm}
\end{wrapfigure}

For the convenience of the reader, we give the algorithm resulting from applying the reduction to Algorithm~1 
of \citet{WGySz:NIPS15} in an explicit form.
For specifying the algorithm we need some extra notation.
Given a SAP instance $\theta = (P,c)$, we let $\gamma_k = \Prob{Y\ne Y^k}$ where $(Y,Y^1,\dots,Y^K)\sim P$ and $k\in [K]$.

Note that $d_k^*(\gamma) = d_k(\gamma)>0$ for any suboptimal
action $k$, while $d_k^*(\gamma) = d^*(\gamma)$ for the optimal action.

Define $C(\gamma) = \{ u\in [0,\infty)^K\,:\, u_1 + \dots + u_j \ge \frac{2\sigma^2}{(d_j^*(\gamma))^2}, j\in [K] \}$
as the set of ``sufficiently informative'' (fractional)  allocation counts 
(these ``counts'' guarantee that the observations are sufficient to distinguish all suboptimal actions from the optimal action; note that $\sigma^2=1/4$).
Further, let $n^*(\gamma)$
be the allocation count that minimizes the total expected excess cost over the set of sufficiently informative allocation counts:
In particular,  we let $u^*(\gamma) = \argmin_{u\in C(\gamma)} \ip{ u, d(\gamma) }$ 
with the understanding that for any optimal action $k$, $u_k^*(\gamma) = \min \{ u_k \,: u\in C(\gamma) \}$ (here, $\ip{x,y} = \sum_i x_i y_i$ is the standard inner product of vectors $x,y$).
For an allocation count $u\in [0,\infty)^K$ let $m(u)\in \N^K$ be the number observations made at each sensor:
$m_j(u) = \sum_{i=1}^j u_i$.

The idea of the algorithm shown as \cref{alg:asym} is as follows:
The algorithm keeps track of an estimate $\hgamma$ of $\gamma$, which is initialized by pulling arm $K$ as this arm
gives information about all the other arms.
In each round, the algorithm first checks whether given the current estimate $\hgamma$ and the current confidence level (where the confidence level is gradually increased over time), the current allocation count $N\in \N^K$
is sufficiently informative (cf. line \ref{alg:check}). If this holds, the action that is optimal under $\hgamma$ is chosen 
(cf. line \ref{alg:greedy}). If the check fails, we need to explore.
The idea of the exploration is that it tries to ensure that the ``optimal plan'' -- assuming $\hgamma$ is the ``correct'' parameter -- is followed (line \ref{alg:plan}). However, this is only reasonable, if all components of $\gamma$ are relatively well-estimated.
Thus, first the algorithm checks whether any of the components of $\gamma$ has a chance of being
extremely poorly estimated (line \ref{alg:starve}). Note that the requirement here is that a significant, but still altogether diminishing fraction of the \emph{exploration rounds} is spent on estimating each components: In the long run, the fraction of exploration rounds amongst all rounds itself is diminishing; hence the forced exploration of line \ref{alg:forced} overall has a small impact on the regret, while it allows to stabilize the algorithm.

\newcommand{\gap}{d}
\newcommand{\norm}[1]{\|#1\|}
For $\theta = (P,c)\in \TSD$, let $\gamma(\theta)$ be the error probabilities for the various sensors.
The following result follows from Theorem~6 of \cite{WGySz:NIPS15}:
\begin{thm}
Let $\epsilon>0$, $\alpha>2$ arbitrary and choose any non-decreasing $\beta(n)$ that satisfies $0\le \beta(n)\le n/2$ and $\beta(m+n)\le \beta(m)+\beta(n)$ for $m,n\in \N$.
Then, 
for any
$\theta = (P,c)\in \TSD$, letting $\gamma = \gamma(\theta)$
the expected regret of \cref{alg:asym} after $T$ steps satisfies 
\todoc{Actually, needs to be checked.. I also replaced $\gap_{\max}(\theta)$ with $1$.}
\begin{align*}
 R_T(\theta,c)
 & \le \big( 2K+2+4K/(\alpha-2) \big) + 4K\sum_{s=0}^T \exp \Big( -\frac{8\beta(s)\epsilon^2}{2K} \Big) \\
& + 2 \beta\Big( 4\alpha\log T\sum_{i\in \iset{K}} u_i^*(\gamma,\epsilon)+K \Big)
   + 4\alpha\log T \sum_{i\in \iset{K}} u_i^*(\gamma,\epsilon)\gap_i(\gamma) \,,
\end{align*}
where $u_i^*(\gamma,\epsilon) = \sup\{u_i^*(\gamma')\,:\, \norm{\gamma'-\gamma}_{\infty}\le \epsilon \}$.
\label{thm:alg1-ub1}
\end{thm}  
Further specifying $\beta(n)$ and using the continuity of $u^*(\cdot)$ at $\theta$, it immediately follows that Algorithm~\ref{alg:asym} achieves asymptotically optimal performance: \todoc{I just copy\&pasted this.
We don't actually have a lower bound..}
\begin{cor}
\label{cor:alg1-asym-opt}
 Suppose the conditions of Theorem~\ref{thm:alg1-ub1} hold. Assume, furthermore, that $\beta(n)$ satisfies $\beta(n) = o(n)$ and $\sum_{s=0}^\infty \exp \left( -\frac{\beta(s)\epsilon^2}{2K\sigma^2} \right)<\infty$ for any $\epsilon>0$, then for any $\theta$ such that $u^*(\theta)$ is unique, 
\[
\limsup_{T\rightarrow \infty} R_T(\theta,c)/\log T \le 4\alpha \inf_{u\in C_\theta} \ip{u,\gap(\gamma(\theta))}\,.
\]
\end{cor}
Note that any $\beta(n) = an^b$ with $a\in (0,\frac{1}{2}]$, $b\in (0,1)$ satisfies the requirements in Theorem~\ref{thm:alg1-ub1} and Corollary~\ref{cor:alg1-asym-opt}.
%Also note that the algorithms presented in \cite{CaKvLe12,BuErSh14} do not achieve this asymptotic bound.





\clearpage