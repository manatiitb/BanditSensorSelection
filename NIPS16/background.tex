%!TEX root =  main.tex

The purpose of this section is to present some necessary background material that will prove to be useful later.
In particular, we introduce a number of sequential decision making problems,
namely stochastic partial monitoring, bandits and bandits with side-observations, which we will build upon later.

First, a few words about our notation:
We will use upper case letters to denote random variables.
The set of real numbers is denoted by $\R$. For positive integer $n$, we let
$[n] = \{1,\dots,n\}$. % with $[n]=\emptyset$ if $n=0$.
We let $M_1(\X)$ to denote the set of probability distributions over some set $\X$.
When $\X$ is finite with a cardinality of $d \doteq |\X|$, 
$M_1(\X)$ can be identified with the $d$-dimensional probability simplex.

In a \emph{stochastic partial monitoring problem} a learner interacts with a stochastic environment in a sequential manner.
In round $t=1,2,\dots$ the learner chooses an action $A_t$ from an action set $\A$, and receives a feedback $Y_t\in \Y$
from a distribution $p$ which depends on the action chosen and also on the environment instance identified
with a ``parameter'' $\theta\in\Theta$:
$Y_t \sim p(\cdot;A_t,\theta)$. 
%The learner chooses $A_t$ based on the past feedbacks $Y_1,\dots,Y_{t-1}$. 
The learner also incurs a reward $R_t$, which is a function of the action chosen and the unknown parameter $\theta$:
$R_t = r(A_t,\theta)$. 
The reward may or may not be part of the feedback for round $t$.
The learner's goal is to maximize its total expected reward.
The family of distributions $(p(\cdot;a,\theta))_{a,\theta}$ and the family of rewards $(r(a,\theta))_{a,\theta}$
and the set of possible parameters $\Theta$ are known to the learner, who uses this knowledge to judiciously choose
its next action to reduce its uncertainty about $\theta$ so that it is able to eventually converge on choosing only an 
optimal action $a^*(\theta)$, achieving the best possible reward per round, $r^*(\theta) = \max_{a\in \A} r(a,\theta)$.
The quantification of the learning speed is given by the expected regret 
$\Regret_n = n r^*(\theta) - \EE{\sum_{t=1}^n R_t}$, which, for brevity and when it does not cause confusion, 
we will just call regret.
A sublinear expected regret, i.e., $\Regret_n/n \to 0$ as $n\to \infty$ means that the learner in the long run collects
almost as much reward on expectation as if the optimal action was known to it.
Such a learner is called Hannan consistent. \todoc{Not sure whether Hannan consistency is this, or when the random average regret converges to zero with probability one.}
In some cases it is more natural to define the problems in terms of costs as opposed to rewards;
in such cases the definition of regret is modified appropriately. 
Transforming between costs and rewards is trivial by flipping the sign of the rewards and costs.

A wide range of interesting sequential learning scenarios can be cast as partial monitoring.
One special case is bandit problems when $\Y$ is the set of real numbers and $r(a,\theta)$
is the mean of distribution $p(\cdot;a,\theta)$: 
Thus, in a bandit problem in evert round the learner chooses an action $A_t$ based on its past observations
and receives the noisy reward $Y_t \sim p(\cdot;A_t,\theta)$ as feedback. 
A bandit problem is special in that the observation $Y_t$ and the reward are directly tied.
Another special case is finite-armed  \emph{bandits with side-observations} \citet{MaSh11},
where each action $a\in \A$ is associated with a neighbor-set $\cN(a)\subset \A$ and
the set of neighborhoods is known to the learner from the beginning.
The learner upon choosing action $A_t\in \A$ receives noisy reward observations for each action in $\cN(A_t)$:
$Y_t  = (Y_{t,a})_{a\in N(A_t)}$, where $Y_{t,a} \sim p_r(\cdot;a,\theta)$, and $\EE{Y_{t,a}} = r(a,\theta)$.
(The action chosen may or may not be an element of $N(A_t)$.)
The reader can readily verify that this problem can also be cast as a partial monitoring problem
by defining $\Y$ as the set $\cup_{i=0}^K \R^i$ and defining 
the family of distributions $(p(\cdot;a,\theta))_{a,\theta}$ such that $Y_t \sim p(\cdot;A_t,\theta)$.
Finally, we note in passing that while we called $\Theta$ a parameter set, 
we have not equipped $\Theta$ with any structure. As such,
the framework is able to model both bona fide parametric settings (e.g., Bernoulli rewards) and 
the so-called non-parametric settings. For example, $K$-armed bandits with reward distributions supported
over $[0,1]$ can be modelled by choosing $\Theta$ as the set of all $K$-tuples 
$\theta:=(\theta_1,\dots,\theta_K)$ of distributions over $[0,1]$ and setting $p(\cdot;a,\theta) = \theta_a(\cdot)$.
More generally, we can identify $\Theta$ with set of instances $(p(\cdot;a,\theta),r(a,\theta))_{\theta\in \Theta}$.
In what follows, when convenient, we will use this identification and will view elements of $\Theta$ as a pair $p,r$ 
where $p(\cdot;a)$ is a probability distribution over $\Y$ for each $a\in \A$ and $r$ is a map from $\A$ to the reals.
